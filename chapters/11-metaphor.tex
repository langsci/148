\chapter{Metaphor}
\label{ch:metaphorandmetonymy}

The ease with which corpora are accessed via word forms is an advantage as long as it is our aim to investigate words, for example with respect to their relationship to other words, to their internal structure or to their distribution \is{distribution!conditional} across grammatical \is{grammar} structures and across texts and language varieties. \is{language variety} As we saw in \chapref{ch:grammar}, the difficulty of accessing corpora at levels of linguistic representation other than the word form is problematic where our aim is to investigate grammar in its own right, but since grammatical structures tend to be associated \is{association} with particular words and\slash or morphemes, \is{morphology} these difficulties can be overcome to some extent.

When it comes to investigating phenomena that are not lexical in nature, the word\hyp{}based nature of corpora is clearly a disadvantage and it may seem as though there is no alternative to a careful manual \is{manual analysis} search and\slash or a sophisticated annotation \is{annotation} (manual, semi\hyp{}manual or based on advanced natural\hyp{}language technology). However, corpus linguists have actually uncovered a number of relationships between words and linguistic phenomena beyond lexicon \is{lexicon} and grammar \is{grammar} without making use of such annotations. \is{annotation} In the final chapter of this book, we will discuss a number of case studies of one such phenomenon: \is{figurative language}\is{metaphor} metaphor.

\section{Studying metaphor in corpora}
\label{sec:studyingmetaphorincorpora}

Metaphor \is{figurative language}\is{metaphor} is traditionally defined as the transfer of a word from one referent (variously called vehicle, figure or source) to another (the tenor, ground or target) (cf. e.g. Aristotle, \textit{Poetics}, XXI). If metaphor were indeed located at the word level, it should be straightforwardly amenable to corpus\hyp{}linguistic analysis. Unfortunately, things are slightly more complicated. First, the transfer does not typically concern individual words but entire semantic \is{semantics} fields (or even conceptual \is{cognitive linguistics} domains, according to some theories). Second, as discussed in some detail in \chapref{ch:retrievalannotation}, there is nothing in the word itself that distinguishes its literal \is{literalness} and metaphorical uses. One way around this problem is manual \is{manual analysis} annotation, \is{annotation} and there are very detailed and sophisticated proposals for annotation procedures (most notably the Pragglejaz Metaphor Identification Procedure, cf., for example, \citet{low_pragglejaz_2010}).

However, as stressed in various places throughout this book, the manual \is{manual analysis} annotation \is{annotation} of corpora severely limits the amount of data that can be included in a research design; \is{research design} this does not invalidate manual annotation, but it makes alternatives highly desirable. Two broad alternatives have been proposed in corpus linguistics. Since these were discussed in some detail in \chapref{ch:retrievalannotation}, we will only repeat them briefly here before illustrating them in more detail in the case studies.

\section{Case studies}
\label{sec:metaphorcasestudies}

The first approach to extracting \is{retrieval} metaphors \is{figurative language}\is{metaphor} from corpora starts from a source domain, searching for individual words or sets of words (synonym \is{synonymy} sets, semantic \is{semantics} fields, discourse domains) and then identifying the metaphorical uses and the respective targets and underlying metaphors manually. \is{manual analysis} This approach is extensively demonstrated, for example, in \citet{deignan_corpus-based_1999, deignan_metaphorical_1999, deignan_metaphor_2005}. The three case studies in Section~\ref{sec:sourcedomains} use this approach. The second approach starts from a target domain, searching for abstract words describing (parts of) the target domain and then identifying those that occur in a grammatical \is{grammar} pattern together with items from a different semantic \is{semantics} domain (which will normally be a source domain). This approach has been taken by \citet{stefanowitsch_happiness_2004, stefanowitsch_words_2006} and others. The case studies in Section~\ref{sec:targetdomains} use this approach. A third approach has been suggested by \citet{wallington_metaphoricity_2003}: they attempt to identify words that are not themselves part of a metaphorical \is{figurative language}\is{metaphor} transfer but that point to a metaphorical transfer in the immediate context (the expression \textit{figuratively speaking} would be an obvious candidate). This approach has not been taken up widely, but it is very promising at least for the identification of certain types of metaphor, and of course the expressions in question are worthy of study in their own right, so one of the case studies in Section~\ref{sec:metaphorandtext} takes a closer look at it.

\subsection{Source domains}
\label{sec:sourcedomains}

Among other things, the corpus\hyp{}based study of (small set of) source domain words may provide insights into the systematicity of metaphor \is{figurative language}\is{metaphor} \citep[cf. esp.][]{deignan_metaphorical_1999}. In cognitive \is{cognitive linguistics} linguistics, it is claimed that metaphor is fundamentally a mapping from one conceptual \is{cognitive linguistics} domain to another, and that metaphorical expressions are essentially a reflex of such mappings. This suggests a high degree of isomorphism between literal \is{literalness} and metaphorical \is{figurative language}\is{metaphor} language: words should essentially display the same systemic and usage\hyp{}based behavior when they are used as the source domain of a metaphor as when they are used in their literal sense unless there is a specific reason in the semantics \is{semantics} of the target domain that precludes this \citep{lakoff_contemporary_1993}.

\subsubsection{Case study: Lexical relations and metaphorical mapping}
\label{sec:antonymymetaphor}

\citet{deignan_metaphorical_1999} tests the isomorphism between literal \is{literalness} and metaphorical \is{figurative language}\is{metaphor} uses of source\hyp{}domain vocabulary very straightforwardly by looking at synonymy \is{synonymy} and antonymy. \is{antonymy} Deignan argues that these lexical relations should be transferred to the target domain, such that, for example, metaphorical meanings \is{semantics} of \textit{cold} and \textit{hot} should also be found for \textit{cool} and \textit{warm} respectively, since their literal meanings are very similar. Likewise, metaphorical \is{figurative language}\is{metaphor} \textit{hot} and \textit{cold} should encode opposites in the same way they do in their literal \is{literalness} uses.

Let us replicate \is{replicability} Deignans study using the BNC Baby. \is{BNC Baby} To keep other factors equal, let us focus on attributive uses of adjectives \is{adjective} that modify target domain nouns \is{noun} (as in \textit{cold facts}), or nouns that are themselves used metaphorically (as in ``The project went into \textit{cold storage}'' meaning work on it ceased). Deignan focuses on the base forms of these adjectives, \is{adjective} let us do the same. She also excludes ``highly fixed collocations \is{collocation} and idioms'' \is{idiomaticity} because their potential metaphorical \is{figurative language}\is{metaphor} origin may no longer be transparent -- let us not follow her here, as we can always identify and discuss such cases after we have extracted \is{retrieval} and tabulated our data.

Deignan does not explicitly present an annotation \is{annotation} scheme, but she presents dictionary\hyp{}like \is{dictionary} definitions of her categories and extensive examples of her categorization \is{categorization} decisions that, taken together, serve the same function. Her categories differ in number (between four and ten) and semantic \is{semantics} granularity across the four words, let us design \is{research design} a stricter annotation \is{annotation} scheme with a minimal number of \is{categorization} categories.

Let us assume that survey of the dictionaries \is{dictionary} already used in Case Study \ref{sec:semanticdifferencesbetweenicandical} yields the following major metaphor \is{figurative language}\is{metaphor} categories:

\begin{enumerate}

\item \textvv{activity}, with the metaphors \textsc{high activity is heat} and \textsc{low activity is coldness}, as in \textit{cold\slash hot war}, \textit{hot pursuit}, \textit{hot topic}, etc.). This sense is not recognized by the dictionaries, \is{dictionary} except insofar as it is implicit in the definitions of \textit{cold war}, \textit{hot pursuit}, \textit{cold trail}, etc. It is understood here to include a sense of \textit{hot} described in dictionaries as ``currently popular'' or ``of immediate interest'' (e.g. \textit{hot topic}).

\item \textvv{affection}, with the metaphors \is{figurative language}\is{metaphor} \textsc{affection is heat} and \textsc{indifference is coldness}, as in \textit{cold stare}, \textit{warm welcome}, etc. This sense is recognized by all dictionaries, \is{dictionary} but we will interpret it to include sense connected with sexual attraction and (un)responsiveness, e.g. \textit{hot date}.

\item \textvv{temperament}, with the metaphors \is{figurative language}\is{metaphor} \textsc{emotional behavior is heat} and \textsc{rational behavior is coldness}, as in \textit{cool head}, \textit{cold facts}, \textit{hot temper}, etc. Most dictionaries \is{dictionary} recognize this sense as distinct from the previous one -- both are concerned with emotion \is{emotions} or its absence, but in case of the \textvv{affection}, the distinction is one between affectionate feelings and their absense, in the case of \textvv{temperament} the distinction is one between behavior based on any emotion \is{emotions} and behavior unaffected by emotion.

\item \textsc{synesthesia}, a category covering uses described in dictionaries \is{dictionary} as ``conveying or producing the impression of being hot, cold, etc.'' in some sensory domain other than temperature, i.e. \textit{warm color}, \textit{cold light}, \textit{cool voice}, etc.

\item \textsc{evaluation}, with the (potential) metaphor \textsc{positive things have a temperature}, as in \textit{a really cool movie}, \textit{a cool person}, \textit{a hot new idea}, etc. This may not be a metaphor \is{figurative language}\is{metaphor} at all, as both uses are very idiomatic; \is{idiomaticity} in fact, \textit{hot} in this sense could be included under \textit{activity} or \textit{affection}, and \textit{cool} in this sense is presumably derived from \textit{temperament}.

\end{enumerate}

\tabref{tab:temperaturemetaphors} lists the token \is{token (instance)} frequencies \is{frequency!token} of the four adjectives \is{adjective} with each of these broad metaphorical \is{figurative language}\is{metaphor} categories as well as all types \is{type (category)} instantiating the respective category. There is one category that does not show any significant deviations from the expected \is{frequency!expected} frequencies, namely the infrequently instantiated category \textvv{synesthesia}. For all other categories, there are clear differences that are unexpected from the perspective of conceptual \is{cognitive linguistics} metaphor theory.

\begin{sidewaystable}[!htbp]
\caption{\textsc{temperature} metaphors (BNC Baby)}
\label{tab:temperaturemetaphors}
\resizebox{0.9\textwidth}{!}{%
\begin{tabular}[t]{lccccr}
\lsptoprule
 & \multicolumn{4}{c}{\textvv{Adjective}} & \\
\textvv{Noun} & \textvv{cold} & \textvv{cool} & \textvv{warm} & \textvv{hot} & Total \\
\midrule
\textvv{\makecell[tl]{activity}}
% me: cold-activity
	& \makecell[t]{\begin{tabular}[t]{lS[table-format=2.2]}
		\small{\textit{Obs.:}} & 13 \\
		\small{\textit{Exp.:}} & 8.36 \\
		\small{\textit{$\chi^2$:}} & 2.57531100478469 \\
		\multicolumn{2}{l}{
			\begin{minipage}[t]{0.2\textwidth} \raggedright
			\footnotesize{\textit{storage}, \textit{turkey}, \textit{war}}
		\end{minipage}}
		\end{tabular}}
% me: cool-activity
	& \makecell[t]{\begin{tabular}[t]{lS[table-format=2.2]}
		\small{\textit{Obs.:}} & 0 \\
		\small{\textit{Exp.:}} & 4.62 \\
		\small{\textit{$\chi^2$:}} & 4.62 \\
		\multicolumn{2}{l}{
			\begin{minipage}[t]{0.2\textwidth} \raggedright
			\footnotesize{--}
		\end{minipage}}
		\end{tabular}}
% me: warm-activity
	& \makecell[t]{\begin{tabular}[t]{lS[table-format=2.2]}
		\small{\textit{Obs.:}} & 0 \\
		\small{\textit{Exp.:}} & 3.96 \\
		\small{\textit{$\chi^2$:}} & 3.96 \\
		\multicolumn{2}{l}{
			\begin{minipage}[t]{0.2\textwidth} \raggedright
			\footnotesize{--}
		\end{minipage}}
		\end{tabular}}
% me: hot-activity
	& \makecell[t]{\begin{tabular}[t]{lS[table-format=2.2]}
		\small{\textit{Obs.:}} & 9 \\
		\small{\textit{Exp.:}} & 5.06 \\
		\small{\textit{$\chi^2$:}} & 3.06790513833992 \\
		\multicolumn{2}{l}{
			\begin{minipage}[t]{0.2\textwidth} \raggedright
			\footnotesize{\textit{pursuit}, \textit{seat}, \textit{spot}, \textit{time}, \textit{war}}
		\end{minipage}}
		\end{tabular}}
	& 22 \\[1.6cm]
\textvv{\makecell[tl]{affection}}
% me: cold-affection
	& \makecell[t]{\begin{tabular}[t]{lS[table-format=2.2]}
		\small{\textit{Obs.:}} & 9 \\
		\small{\textit{Exp.:}} & 10.26 \\
		\small{\textit{$\chi^2$:}} & 0.154736842105263 \\
		\multicolumn{2}{l}{
			\begin{minipage}[t]{0.2\textwidth} \raggedright
			\footnotesize{\textit{atmosphere}, \textit{disapproval}, \textit{eyes}, \textit{note}, \textit{person}, \textit{response}, \textit{sarcasm}, \textit{savagery}, \textit{shoulder}}
		\end{minipage}}
		\end{tabular}}
% me: cool-affection
	& \makecell[t]{\begin{tabular}[t]{lS[table-format=2.2]}
		\small{\textit{Obs.:}} & 0 \\
		\small{\textit{Exp.:}} & 5.67 \\
		\small{\textit{$\chi^2$:}} & 5.67 \\
		\multicolumn{2}{l}{
			\begin{minipage}[t]{0.2\textwidth} \raggedright
			\footnotesize{--}
		\end{minipage}}
		\end{tabular}}
% me: warm-affection
	& \makecell[t]{\begin{tabular}[t]{lS[table-format=2.2]}
		\small{\textit{Obs.:}} & 16 \\
		\small{\textit{Exp.:}} & 4.86 \\
		\small{\textit{$\chi^2$:}} & 25.5348971193416 \\
		\multicolumn{2}{l}{
			\begin{minipage}[t]{0.2\textwidth} \raggedright
			\footnotesize{\textit{affinities}, \textit{approval}, \textit{embrace}, \textit{feeling}, \textit{glow}, \textit{kiss}, \textit{liking}, \textit{nest}, \textit{presence}, \textit{relief}, \textit{smile}, \textit{welcome}}
		\end{minipage}}
		\end{tabular}}
% me: hot-affection
	& \makecell[t]{\begin{tabular}[t]{lS[table-format=2.2]}
		\small{\textit{Obs.:}} & 2 \\
		\small{\textit{Exp.:}} & 6.21 \\
		\small{\textit{$\chi^2$:}} & 2.85412238325282 \\
		\multicolumn{2}{l}{
			\begin{minipage}[t]{0.2\textwidth} \raggedright
			\footnotesize{\textit{date}, \textit{pants}}
		\end{minipage}}
		\end{tabular}}
	& 27 \\[2.7cm]
\textvv{\makecell[tl]{temperament}}
% me: cold-temperament
	& \makecell[t]{\begin{tabular}[t]{lS[table-format=2.2]}
		\small{\textit{Obs.:}} & 14 \\
		\small{\textit{Exp.:}} & 10.64 \\
		\small{\textit{$\chi^2$:}} & 1.06105263157895 \\
		\multicolumn{2}{l}{
			\begin{minipage}[t]{0.2\textwidth} \raggedright
			\footnotesize{\textit{approach}, \textit{blood}, \textit{calculation}, \textit{clarity}, \textit{facts}, \textit{reality}, \textit{reminder}}
		\end{minipage}}
		\end{tabular}}
% me: cool-temperament
	& \makecell[t]{\begin{tabular}[t]{lS[table-format=2.2]}
		\small{\textit{Obs.:}} & 13 \\
		\small{\textit{Exp.:}} & 5.88 \\
		\small{\textit{$\chi^2$:}} & 8.62149659863946 \\
		\multicolumn{2}{l}{
			\begin{minipage}[t]{0.2\textwidth} \raggedright
			\footnotesize{\textit{analysis}, \textit{composure}, \textit{customer}, \textit{firmness}, \textit{head}, \textit{look}, \textit{response}, \textit{restraint}, \textit{Rose (prop. name)}, \textit{silence}, \textit{temper}}
		\end{minipage}}
		\end{tabular}}
% me: warm-temperament
	& \makecell[t]{\begin{tabular}[t]{lS[table-format=2.2]}
		\small{\textit{Obs.:}} & 0 \\
		\small{\textit{Exp.:}} & 5.04 \\
		\small{\textit{$\chi^2$:}} & 5.04 \\
		\multicolumn{2}{l}{
			\begin{minipage}[t]{0.2\textwidth} \raggedright
			\footnotesize{--}
		\end{minipage}}
		\end{tabular}}
% me: hot-temperament
	& \makecell[t]{\begin{tabular}[t]{lS[table-format=2.2]}
		\small{\textit{Obs.:}} & 1 \\
		\small{\textit{Exp.:}} & 6.44 \\
		\small{\textit{$\chi^2$:}} & 4.59527950310559 \\
		\multicolumn{2}{l}{
			\begin{minipage}[t]{0.2\textwidth} \raggedright
			\footnotesize{\textit{spirits}}
		\end{minipage}}
		\end{tabular}}
	& 28 \\[2.7cm]
\textvv{\makecell[tl]{synesthesia}}
% me: cold-synesthesia
	& \makecell[t]{\begin{tabular}[t]{lS[table-format=2.2]}
		\small{\textit{Obs.:}} & 2 \\
		\small{\textit{Exp.:}} & 1.9 \\
		\small{\textit{$\chi^2$:}} & 0.005263157894737 \\
		\multicolumn{2}{l}{
			\begin{minipage}[t]{0.2\textwidth} \raggedright
			\footnotesize{\textit{fire}, \textit{grey}}
		\end{minipage}}
		\end{tabular}}
% me: cool-synesthesia
	& \makecell[t]{\begin{tabular}[t]{lS[table-format=2.2]}
		\small{\textit{Obs.:}} & 1 \\
		\small{\textit{Exp.:}} & 1.05 \\
		\small{\textit{$\chi^2$:}} & 0.002380952380952 \\
		\multicolumn{2}{l}{
			\begin{minipage}[t]{0.2\textwidth} \raggedright
			\footnotesize{\textit{voice}}
		\end{minipage}}
		\end{tabular}}
% me: warm-synesthesia
	& \makecell[t]{\begin{tabular}[t]{lS[table-format=2.2]}
		\small{\textit{Obs.:}} & 2 \\
		\small{\textit{Exp.:}} & 0.9 \\
		\small{\textit{$\chi^2$:}} & 1.34444444444444 \\
		\multicolumn{2}{l}{
			\begin{minipage}[t]{0.2\textwidth} \raggedright
			\footnotesize{\textit{colour}, \textit{glow}}
		\end{minipage}}
		\end{tabular}}
% me: hot-synesthesia
	& \makecell[t]{\begin{tabular}[t]{lS[table-format=2.2]}
		\small{\textit{Obs.:}} & 0 \\
		\small{\textit{Exp.:}} & 1.15 \\
		\small{\textit{$\chi^2$:}} & 1.15 \\
		\multicolumn{2}{l}{
			\begin{minipage}[t]{0.2\textwidth} \raggedright
			\footnotesize{--}
		\end{minipage}}
		\end{tabular}}
	& 5 \\[1.6cm]
\textvv{\makecell[tl]{evaluation}}
% me: cold-evaluation
	& \makecell[t]{\begin{tabular}[t]{lS[table-format=2.2]}
		\small{\textit{Obs.:}} & 0 \\
		\small{\textit{Exp.:}} & 6.84 \\
		\small{\textit{$\chi^2$:}} & 6.84 \\
		\multicolumn{2}{l}{
			\begin{minipage}[t]{0.2\textwidth} \raggedright
			\footnotesize{--}
		\end{minipage}}
		\end{tabular}}
% me: cool-evaluation
	& \makecell[t]{\begin{tabular}[t]{lS[table-format=2.2]}
		\small{\textit{Obs.:}} & 7 \\
		\small{\textit{Exp.:}} & 3.78 \\
		\small{\textit{$\chi^2$:}} & 2.74296296296296 \\
		\multicolumn{2}{l}{
			\begin{minipage}[t]{0.2\textwidth} \raggedright
			\footnotesize{\textit{countess}, \textit{dude}, \textit{million}, \textit{thing}}
		\end{minipage}}
		\end{tabular}}
% me: warm-evaluation
	& \makecell[t]{\begin{tabular}[t]{lS[table-format=2.2]}
		\small{\textit{Obs.:}} & 0 \\
		\small{\textit{Exp.:}} & 3.24 \\
		\small{\textit{$\chi^2$:}} & 3.24 \\
		\multicolumn{2}{l}{
			\begin{minipage}[t]{0.2\textwidth} \raggedright
			\footnotesize{--}
		\end{minipage}}
		\end{tabular}}
% me: hot-evaluation
	& \makecell[t]{\begin{tabular}[t]{lS[table-format=2.2]}
		\small{\textit{Obs.:}} & 11 \\
		\small{\textit{Exp.:}} & 4.14 \\
		\small{\textit{$\chi^2$:}} & 11.3670531400966 \\
		\multicolumn{2}{l}{
			\begin{minipage}[t]{0.2\textwidth} \raggedright
			\footnotesize{\textit{favourite}, \textit{gunner}, \textit{handicap}, \textit{multimedia system}, \textit{shot}, \textit{stuff}, \textit{tips}}
		\end{minipage}}
		\end{tabular}}
	& 18 \\[1.6cm]
\midrule
Total
	& \makecell[t]{38}
	& \makecell[t]{21}
	& \makecell[t]{23}
	& \makecell[t]{18}
	& \makecell[t]{100} \\
\lspbottomrule
\multicolumn{6}{l}{\scriptsize{Supplementary Online Material: WLVF}} \\ %OSM
\end{tabular}}
\end{sidewaystable}

The category \textit{activity} is instantiated only for the words \textit{cold} and \textit{hot} and its absence for the other two words is significant. We can imagine (and, in a sufficiently large \is{corpus size} data set, find) uses for \textit{cool} and \textit{warm} that would fall into this category. For example, Frederick Pohl's 1981 novel \emph{The Cool War} describes a geopolitical situation in which political allies sabotage each other's economies, and it is occasionally used to refer to real\hyp{}life situations as well. But this seems to be a deliberate analogy rather than a systematic use, leaving us with an unexpected gap in the middle of the linguistic scale between hot and cold.

The category \textit{affection} is found with three of the four words, but its absence for the word \textit{cool} is statistically significant, as is its clear overrepresentation with \textit{warm}. This lack of systematicity is even more unexpected than the one observed with \textit{activity}: for the latter, we could argue that it reflects a binary distinction that uses only the extremes of the scale, for example because there is not enough of a potential conceptual difference between a \textit{cold war} and a \textit{cool war}. With \textvv{affection}, in contrast, this explanation is not adequate, as the entire scale is used. It remains unclear, therefore, why \textit{warm} should be so prominently used here, and why \textit{cool} is so rare (it is possible: the dictionaries \is{dictionary} list examples like \textit{a cool manner}, \textit{a cool reply}).

With \textit{temperament}, we find a partially complementary situation: again, three of the four words occur with this metaphor, \is{figurative language}\is{metaphor} including, again, the extreme points. However, in this case it is \textit{cool} that is significantly overrepresented and \textit{warm} that is significantly absent. A possible explanation would be that there is a potential for confusion between the metaphors \textit{affection is temperature} and \textit{temperament is temperature}, and so speakers divide up the continuum from cold to hot between them. However, this does not explain why \textit{cold} is frequently used in both metaphorical senses.

The gaps in the last category, \textvv{evaluation}, are less confusing. As mentioned above, this is probably not a single coherent category and we would not expect uses to be equally disributed across the four words.

This case study demonstrates the use of corpus data to evaluate claims about conceptual structure. Specifically, it shows how a central claim of conceptual metaphor \is{figurative language}\is{metaphor} theory can be investigated (see the much more detailed discussion in \citet{deignan_metaphorical_1999}.

\subsubsection{Case study: Word forms in metaphorical mappings}
\label{sec:flamevsflames}

Another area in which we might expect a large degree of isomorphism between literal \is{literalness} and metaphorical \is{figurative language}\is{metaphor} uses of a word is the quantitative and qualitative distribution \is{distribution!conditional} of word forms. In a highly intriguing study, \citet{stefanowitsch_grammar_2006} investigate the metaphors associated \is{association} with the source domain words \textit{flame} and \textit{flames} in terms of whether they occur in positively or negatively connoted \is{connotation} contexts.

Her study is generally deductive \is{deduction} in that she starts with an expectation (if not quite a full\hyp{}fledged hypothesis) that there are frequently differences between the singular \is{number} and the plural forms of a metaphorically \is{figurative language}\is{metaphor} used word with respect to \is{connotation} connotation.

A cursory look at a few relatively randomly \is{chance} selected examples appears to corroborate this expression. More precisely, it seems that the singular form \textit{flame} has positive connotations \is{connotation} more frequently than expected \is{frequency!expected} (cf. (\ref{ex:flamemet}), while the plural \is{number} form \textit{flames} has negative connotations \is{connotation} more frequently than expected (cf. (\ref{ex:flamesmet}):

\begin{exe}
\ex
\begin{xlist}
\label{ex:flamemet}
\ex $[$T$]$he flame of hope burns brightly here. (BNC AJD)
\ex Emilio Estevez, sitting on the sofa next to old flame Demi Moore... (BNC CH1)
\end{xlist}
\end{exe}

\begin{exe}
\ex
\begin{xlist}
\label{ex:flamesmet}
\ex ...the flames of civil war engulfed the central Yugoslav republic. (BNC AHX)
\ex The game was going OK and then it went up in flames. (BNC CBG)
\end{xlist}
\end{exe}

Deignan studies this potential difference systematically based on a sample of more than 1500 hits for \textit{flame/s} in the Bank of English (a proprietary, non\hyp{}accessible corpus owned by HarperCollins), from which she manually \is{manual analysis} extracts \is{retrieval} all 153 metaphorical uses. These are then categorized \is{categorization} according to their connotation. \is{connotation} Deignan's design \is{research design} thus has two nominal \is{nominal data} variables: \textsc{Word Form of Flame} (with the variables \textsc{singular} and \textsc{plural}) \is{number} and \textsc{Connotation of Metaphor} \is{figurative language}\is{metaphor} (with the values \textsc{positive} and \textsc{negative}. She does not provide an annotation \is{annotation} scheme for categorizing \is{categorization} the metaphorical expressions, but she provides a set of examples that are intuitively quite plausible. \tabref{tab:flamemetposneg} shows her results ($\chi^2 = 53.98, \df = 1, p < 0.001$).\is{chi-square test}

\begin{table}
\caption{Positive and negative metaphors with singular and plural forms of \textit{flame} \citep[117]{stefanowitsch_grammar_2006}}
\label{tab:flamemetposneg}
\begin{tabular}[t]{llccr}
\lsptoprule
 & & \multicolumn{2}{c}{\textvv{Word Form of Flame}} & \\
 & & \textvv{singular} & \textvv{plural} & Total \\
\midrule
\textvv{\makecell[lt]{Connotation}}
	& \textvv{positive}
		& \makecell[t]{\num{90}\\\small{(\num{70.78})}}
		& \makecell[t]{\num{24}\\\small{(\num{43.22})}}
		& \makecell[t]{\num{114}\\} \\
	& \textvv{negative}
		& \makecell[t]{\num{5}\\\small{(\num{24.22})}}
		& \makecell[t]{\num{34}\\\small{(\num{14.78})}}
		& \makecell[t]{\num{39}\\} \\
\midrule
	& Total
		& \makecell[t]{\num{95}}
		& \makecell[t]{\num{58}}
		& \makecell[t]{\num{153}} \\
\lspbottomrule
\end{tabular}
\end{table}
% me: chisq.test(matrix(c(90,5,24,34),ncol=2),corr=FALSE)

Clearly, metaphorical \is{figurative language}\is{metaphor} singular \textit{flame} is used in positive metaphorical contexts much more frequently than metaphorical plural \is{number} \textit{flames}. Deignan tentatively explains this in relation to the literal \is{literalness} uses of \textit{flame(s)}: a single flame is ``usually under control'', and it may ``be if use, as a candle or a burning match''. If there is more than one flame, we are essentially dealing with a fire -- ``flames are often undesired, out of control and very dangerous'' \citep[117]{stefanowitsch_grammar_2006}.

This explanation itself is of course a hypothesis about the literal \is{literalness} use of singular \is{number} and plural \textit{flame} that must be tested separately. Deignan does not provide such a test, so let us do it here. Let us select a sample of 20 hits each for literal uses the singular and plural of \textit{flame(s)} from the BNC \is{BNC} (as mentioned above, Deignan's corpus is not accessible, so we must hope that the BNC is roughly comparable). \figref{fig:flameconc} shows the sample for singular \is{number} (lines 1--20) and plural (21--40).

\begin{figure}
\caption{Concordance of \textit{flame(s)} (BNC, Sample)}
\label{fig:flameconc}
\hrulefill
\begin{fitverb}
 1 ford base . The jet crashed in a ball of [flame] , destroying 15 cars and damaging 10 mo
 2  face down , applying the cheroot to the [flame] . But his eyes never left the four men
 3 ls of a child 's that had passed through [flame] and were partially melted . They would
 4 ontainer next to him . An orange ball of [flame] ripped up into the sky , bathing the de
 5 went out of one door but then a sheet of [flame] came down and blocked me , so I had to
 6  . The fire burns evenly with a thin hot [flame] , as though there are no oils or resins
 7 ill-smouldering logs , fanning them into [flame] . He places some more logs from a pile
 8  of sherry to the momentary blue veil of [flame] on the pudding , been what she would ha
 9 truck one and cupped his hand around the [flame] . ` Cheers , ' said the man and dis
10 ed with the element , burning circles of [flame] round creatures she had demanded Ariel
11 the soft promise of the light burst into [flame] ; the vanguard of the islanders fell ba
12 arched for his lighter , and touched the [flame] to the tip to make contact with him . I
13  again , tighter this time , guiding the [flame] . She sucked , and the cigarette end gl
14   There , ' she shouted , pluming liquid [flame] from one claw , ` you 're not the onl
15 and steady to bring the cigarette to the [flame] and kept it for a few seconds longer th
16 ern horns outside their house , the weak [flame] of the candles fluttering in their prot
17 the upstairs windows , a sudden spurt of [flame] , and a part of the roof begin to sag o
18  however , disappear in a white sheet of [flame] . He just kept right on kicking Pikey ,
19 ue to finish . Do n't cook over a fierce [flame] . The outside of the food will cook bef
20 no cushion . Candle erm Church , steeple [Flame] . Steeple . Got it got it got it got it

21 it was winning its battle to put out the [flames] . He had to do it now , while it was s
22 ner , arm raised . Its back was to him , [flames] still glowing deep in its side . He ra
23  how the roof caved in before a sheet of [flames] spread across the fuselage , cutting h
24 control down a steep hill and burst into [flames] . The fully-laden truck careered throu
25 s spent more than two hours fighting the [flames] , police said . Bowbazaar , in the cen
26 h a shining chair by a fire with fragile [flames] . These images had what Alexander desi
27 lood rejected -- racks of fragile spiked [flames] of votive candles , elaborate china an
28 ce , looking into the authentic fake gas [flames] as he sipped his drink . He touched hi
29 ross to the fireplace , staring into the [flames] . ` There 's no reason why he should
30 the ground and died , no explosions , no [flames] reaching to the sky . It simply flippe
31 ack of her head , protected her from the [flames] and blocked out any further damage to
32 W went first , its roof torn open by the [flames] and blast as if by a giant unseen can
33 stunned and wearied by the water and the [flames] , the howling and frantic clangour of
34 annonballs , and caught the smell of the [flames] , of split flesh , and heard the howls
35  . And the best is yet to come . ' ` The [flames] of hell ? ' ` Exactly . Operatic
36 ds , then night crept back in around the [flames] . Trails of burning liquid spiderwebbe
37 e properly . Susan reeled away from us , [flames] springing up where she had been touche
38  , and his leg broke in two places . The [flames] were dying down . I could see his blac
39 mes and lower temperatures to reduce the [flames] . ` Eventually all the cooking was do
40 s . This is Crystal Palace going up in f [flames] . November the thirtieth nineteen thir
\end{fitverb}
\hrulefill
\end{figure}

It is difficult to determine which hits should be categorized \is{categorization} as positive and which as negative. Let us assume that any unwanted and\slash or destructive fire should be characterized as negative, and, on this basis, categorize lines 1, 3, 4, 5, 11, 14, 17, 18, 21, 22, 23, 24, 25, 30, 31, 33, 34, 35, 36, 37, 38 and 40 as \textvv{negative} and the rest as \textvv{positive}. This would give us the result in \tabref{tab:flamelitposneg} (if you disagree with the categorization, \is{categorization} come up with your own and do the corresponding calculations).

\begin{table}
\caption{Positive and negative contexts for literal uses of singular and plural forms of \textit{flame/s}}
\label{tab:flamelitposneg}
\begin{tabular}[t]{llccr}
\lsptoprule
 & & \multicolumn{2}{c}{\textvv{Word Form of Flame}} & \\
 & & \textvv{positive} & \textvv{negative} & Total \\
\midrule
\textvv{\makecell[lt]{Connotation}}
	& \textvv{singular}
		& \makecell[t]{\num{12}\\\small{(\num{9.00})}}
		& \makecell[t]{\num{8}\\\small{(\num{11.00})}}
		& \makecell[t]{\num{20}\\} \\
	& \textvv{plural}
		& \makecell[t]{\num{6}\\\small{(\num{9.00})}}
		& \makecell[t]{\num{14}\\\small{(\num{11.00})}}
		& \makecell[t]{\num{20}\\} \\
\midrule
	& Total
		& \makecell[t]{\num{18}}
		& \makecell[t]{\num{22}}
		& \makecell[t]{\num{40}} \\
\lspbottomrule
\end{tabular}
\end{table}
% me: chisq.test(matrix(c(12,6,8,14),ncol=2),corr=FALSE)

It does seem that negative connotations \is{connotation} are found more frequently with literal \is{literalness} uses of the plural form \textit{flames} than with literal uses of the singular form \textit{flame}. Despite the small size \is{corpus size} of the sample used here, this difference only just fails to reach statistical significance ($\chi^2 = 3.64, \df = 1, p = 0.0565$). The difference would likely become significant if we used a larger sample. However, it is nowhere near as pronounced as in the metaphorical \is{figurative language}\is{metaphor} uses presented by Deignan. A crucial difference between literal \is{literalness} and metaphorical uses may be that fire is inherently dangerous and so literal references to fire are more likely to be negative than metaphorical ones, that allow us to focus on other aspects of fire. Interestingly, however, most of the negative uses of singular \is{number} \textit{flame} occur in constructions like \textit{ball of flame}, \textit{sheet of flame} and \textit{spurt of flame}, where \textit{flame} could be argued to be a mass noun \is{noun} rather than a true singular form. If we remove these five uses, then the difference between singular and plural becomes very significant even in the now further reduced sample ($\chi^2 = 8.58, \df = 1, p < 0.01$).

Thus, Deignan's explanation appears to be generally correct, providing evidence for a substantial degree of isomorphism between literal \is{literalness} and figurative \is{figurative language} uses of (at least some) words. An analysis of more such cases could show whether this isomorphism between literal and metaphorical \is{figurative language}\is{metaphor} uses is a general principle (as the conceptual \is{cognitive linguistics} theory of metaphor as \citet{lakoff_contemporary_1993} suggests it should be.

This case study demonstrates first, how to approach the study of metaphor starting from source\hyp{}domain words, and, second, that such an approach may be applied not just descriptively, \is{description} but in the context of answering fundamental questions about the nature of \is{figurative language}\is{metaphor} metaphor.

\subsubsection{Case study: The impact of metaphorical expressions}
\label{sec:theimpactofmetaphoricalexpressions}

A slightly different example of a source\hyp{}domain oriented study is found in \citet{stefanowitsch_function_2005}, which investigates the relationship between metaphorical and literal \is{literalness} expressions hinted at at the end of the preceding case study. The aim of the study is to uncover evidence for the function of metaphorical \is{figurative language}\is{metaphor} expressions that have literal paraphrases, such as [\textit{dawn of} NP] in examples like (\ref{ex:dawnof}a), which is seemingly equivalent to the literal \is{literalness} [\textit{beginning of} NP] in (\ref{ex:dawnof}b):

\begin{exe}
\ex
\begin{xlist}
\label{ex:dawnof}
\ex $[$I$]$t has taken until the dawn of the 21st century to realise that the best methods of utilising . . . our woodlands are those employed a millennium ago. (BNC AHD)
\ex Communal life survived until the beginning of the nineteenth century and traditions peculiar to that way of life had lingered into the present. (BNC AEA)
\end{xlist}
\end{exe}

Other examples studied in \citet{stefanowitsch_function_2005} are \textit{in the center\slash heart of}, \textit{at the center\slash heart of} and \textit{a(n) increase\slash growth\slash rise in}. The studies have two nominal \is{nominal data} variables: the independent variable is \textvv{Metaphoricity of Pattern} (whose values are pairs of patterns like the one illustrated in examples (\ref{ex:dawnof}a, b)), the dependent variable is \textvv{Noun} (whose values are the nouns in the NP slot provided by these patterns. Methodologically, this corresponds to a differential collexeme \is{collostructional analysis}\is{collexeme!differential} analysis (\chapref{ch:grammar}, Case Study \ref{sec:ditransitiveandprepositionaldative}).

The studies are deductive \is{deduction} in that they aim to test the hypothesis that metaphorical \is{figurative language}\is{metaphor} language serves a cognitive \is{cognitive linguistics} function and that for each pair of patterns investigated, the metaphorical variant should be used with nouns \is{noun} referring to more complex entities. The construct \textsc{complexity} is operationalized \is{operationalization} in the form of axioms derived from gestalt psychology, \is{psychology} such as the following:

\begin{quote}
Concepts representing entities that have a simple shape and\slash or have a clear boundary are less complex than those representing entities with complex shapes or fuzzy boundaries (because they are more easily delineable). This follows from the gestalt principles of closure and simplicity \citep[170]{stefanowitsch_function_2005}.
\end{quote}

For each pair of expressions, the differential collexemes \is{collostructional analysis}\is{collexeme!differential} are identified and the resulting lists are compared against these axiomatic assumptions. Let us illustrate this using the pattern \textit{the dawn\slash beginning of NP}. A case insensitive query for the string \texttt{dawn} or \texttt{beginning}, followed by \textit{of}, followed by up to three words that are not a noun, \is{noun} followed by a noun yields the results shown in \tabref{tab:dawnofdifferential} (they are very similar to those based on a more careful manual extraction \is{retrieval} in \citealt{stefanowitsch_function_2005}).

\begin{table}
\caption{Differential collexemes of \textit{beginning of NP} and \textit{dawn of NP} (BNC)}
\label{tab:dawnofdifferential}
\resizebox{\textwidth}{!}{%
\begin{tabular}[t]{l *{2}{S[table-format=4]} *{2}{S[table-format=8]} S}
\lsptoprule
\multicolumn{1}{c}{\makecell[tc]{\textvv{Collexeme}}} & \multicolumn{1}{c}{\makecell[tc]{Frequency \\ with \textvv{dawn}}} & \multicolumn{1}{c}{\makecell[tc]{Frequency \\ with \textvv{beginning}}} & \multicolumn{1}{c}{\makecell[tc]{Other words \\ with \textvv{dawn}}} & \multicolumn{1}{c}{\makecell[tc]{Other words \\ with \textvv{beginning}}} & \multicolumn{1}{c}{\makecell[tc]{\emph{G}}} \\
\midrule
\multicolumn{6}{l}{Most strongly associated with \textvv{dawn}} \\
\midrule
\textit{civilisation} & 25 & 1 & 112 & 3584 & 161.385803778975 \\
\textit{history} & 9 & 12 & 128 & 3573 & 32.180697883425 \\
\textit{time} & 13 & 39 & 124 & 3546 & 31.2657527467307 \\
\textit{era} & 10 & 30 & 127 & 3555 & 23.872749059115 \\
\textit{dream} & 4 & 1 & 133 & 3584 & 21.598766197973 \\
\textit{mankind} & 3 & 0 & 134 & 3585 & 19.8759726284138 \\
\textit{day} & 9 & 33 & 128 & 3552 & 18.7007109334932 \\
\textit{age} & 5 & 8 & 132 & 3577 & 16.4543085057119 \\
\multicolumn{6}{l}{...} \\
\midrule
\multicolumn{6}{l}{Most strongly associated with \textvv{beginning}} \\
\midrule
\textit{year} & 1 & 317 & 136 & 3268 & 17.7844361675103 \\
\textit{century} & 4 & 382 & 133 & 3203 & 11.4120728157816 \\
\textit{chapter} & 0 & 100 & 137 & 3485 & 7.60512306421009 \\
\textit{end} & 0 & 94 & 137 & 3491 & 7.14281356737414 \\
\textit{war} & 0 & 75 & 137 & 3510 & 5.68395907143544 \\
\textit{week} & 0 & 61 & 137 & 3524 & 4.61396278618309 \\
\textit{month} & 0 & 60 & 137 & 3525 & 4.53769415111556 \\
\textit{period} & 0 & 55 & 137 & 3530 & 4.15666913330235 \\
\textit{term} & 0 & 51 & 137 & 3534 & 3.85223020091959 \\
\lspbottomrule
\end{tabular}}
\end{table}
% me: query: BNC; "dawn"%c "of"%c [pos!=".*NN.*"]{0,3} [pos=".*NN.*"]; and
% me: BNC; "beginning"%c "of"%c [pos!=".*NN.*"]{0,3} [pos=".*NN.*"];

Unsurprisingly, both expressions are associated \is{association} almost exclusively with words referring to events and time spans (or, in some cases, with entities that exist through time, like \textit{mankind}, or that we interact with through time, like \textit{chapter}). Crucially, most of the nouns \is{noun} associated with the literal \is{literalness} \textit{beginning of} refer to time spans with clear boundaries and a clearly defined duration (\textit{year}, \textit{century}, etc.), while those associated \is{association} with the metaphorical \is{figurative language}\is{metaphor} \textit{dawn of} refer to events and time spans without clear boundaries or a clear duration (\textit{civilization}, \textit{time}, \textit{history}, \textit{age}, \textit{era}, \textit{culture}). The one apparent exception is \textit{day}, but this occurs exclusively in literal \is{literalness} uses of \textit{dawn of}, such as \textit{It was the dawn of the fourth day since the murder} (BNC CAM). This (and similar results for other pairs of expressions) are presented in \citet{stefanowitsch_function_2005} as evidence for a cognitive \is{cognitive linguistics} function \is{figurative language}\is{metaphor} of metaphor.

In a short discussion of this study, \citet{liberman_what_2005} notes in passing that even individual decades and centuries may differ in the degree to which they prefer \textit{beginning of} or \textit{dawn of}: using internet search engines, he shows that \textit{dawn of the 1960s} is more probable than \textit{dawn of the 1980s} compared to \textit{beginning of the 1960s\slash 1980s}, and that \textit{dawn of the 21st century} is more probable than \textit{dawn of the 18th century} compared to \textit{beginning of the 18th\slash 21st century}. He rightly points out that this seems to call into question the properties of boundedness and well\hyp{}defined length that \citet{stefanowitsch_function_2005} appeals to, since obviously all decades\slash centuries are equally bounded.

Since search engine frequency data are notoriously unreliable, let us replicate \is{replicability} this observation in a large \is{corpus size} corpus, the 400+ million word Corpus of Contemporary American English (COCA). The names of decades (such as \textit{1960s} or \textit{sixties}) occur too infrequently with \textit{dawn of} in this corpus to say anything useful about them, but the names of centuries are frequent enough for a differential collexeme \is{collostructional analysis}\is{collexeme!differential} analysis.

\tabref{tab:dawnofcentury} shows the percentage of \textit{dawn of} for the past ten centuries (spelling) variants of the respective centuries, such as \textit{19th century}, \textit{nineteenth century}, etc.) as well as spelling errors were normalized to the spelling shown in this table.

\begin{table}
\caption{Differential collexemes of \textit{dawn\slash beginning of \_\_ century} (COCA)}
\label{tab:dawnofcentury}
\resizebox{\textwidth}{!}{%
\begin{tabular}[t]{l *{2}{S[table-format=4]} *{2}{S[table-format=8]} S}
\lsptoprule
\multicolumn{1}{c}{\makecell[tc]{\textvv{Collexeme}}} & \multicolumn{1}{c}{\makecell[tc]{Frequency \\ with \textvv{dawn}}} & \multicolumn{1}{c}{\makecell[tc]{Frequency \\ with \textvv{beginning}}} & \multicolumn{1}{c}{\makecell[tc]{Other words \\ with \textvv{dawn}}} & \multicolumn{1}{c}{\makecell[tc]{Other words \\ with \textvv{beginning}}} & \multicolumn{1}{c}{\makecell[tc]{\emph{G}}} \\
\midrule
\multicolumn{6}{l}{Most strongly associated with \textvv{dawn of \_\_ century}} \\
\midrule
\textit{the twenty-first} & 32 & 90 & 52 & 623 & 29.9285102625756 \\
\textit{a new} & 7 & 6 & 77 & 707 & 15.329855332433 \\
\textit{a} & 1 & 0 & 83 & 713 & 4.51077281590357 \\
\textit{America's} & 1 & 0 & 83 & 713 & 4.51077281590357 \\
\textit{an American} & 1 & 0 & 83 & 713 & 4.51077281590357 \\
\textit{an Asian} & 1 & 0 & 83 & 713 & 4.51077281590357 \\
\textit{our new} & 1 & 0 & 83 & 713 & 4.51077281590357 \\
\textit{that ancient} & 1 & 0 & 83 & 713 & 4.51077281590357 \\
\textit{the eleventh} & 1 & 0 & 83 & 713 & 4.51077281590357 \\
\textit{the incoming} & 1 & 0 & 83 & 713 & 4.51077281590357 \\
\midrule
\multicolumn{6}{l}{Most strongly associated with \textvv{beginning of \_\_ century}} \\
\midrule
\textit{the} & 2 & 94 & 82 & 619 & 11.4620118146882 \\
\textit{the nineteenth} & 2 & 68 & 82 & 645 & 6.39669989856688 \\
\textit{this} & 5 & 97 & 79 & 616 & 4.6940397471707 \\
\textit{the seventeenth} & 0 & 14 & 84 & 699 & 3.14778820208579 \\
\lspbottomrule
\end{tabular}}
\end{table}

There are clear differences between the centuries associated with \textvv{dawn} and those associated \is{association} with \textvv{beginning}: the literal \is{literalness} expression is associated with the past (\textit{nineteenth}, \textit{seventeenth} (just below significance)), while the metaphorical \is{figurative language}\is{metaphor} expression, as already observed by Liberman, is associated with the twenty\hyp{}first century, i.e., the future (the expressions \textit{a new}, \textit{our new} and \textit{the incoming} also support this). I would argue that this does point to a difference in boundedness and duration. While all centuries are objectively speaking, of the same length and have the same clear boundaries, it seems reasonable to assume that the past feels more bounded than the future because it is actually over, and we can imagine it in its entirety. In contrast, none of the speakers in the COCA will live to see the end of the 21st century, making it conceptually less bounded to them.

If this is true, then we should be able to observe the same effect in the past: When the twentieth century was still the future, it, too, should have been associated \is{association} with the metaphorical \is{figurative language}\is{metaphor} \textit{dawn of}. Let us test this hypothesis using the Corpus of Historical American English, which includes language from the early nineteenth to the very early twenty\hyp{}first century -- in a large part of the corpus, the twentieth century was thus entirely or partly in the future. \tabref{tab:dawnofdifferentialcoha} shows the differential collexemes \is{collostructional analysis}\is{collexeme!differential} of the two expressions in this corpus.

\begin{table}
\caption{Differential collexemes of \textit{dawn\slash beginning of \_\_ century} (COHA)}
\label{tab:dawnofdifferentialcoha}
\resizebox{\textwidth}{!}{%
\begin{tabular}[t]{l *{2}{S[table-format=4]} *{2}{S[table-format=8]} S}
\lsptoprule
\multicolumn{1}{c}{\makecell[tc]{\textvv{Collexeme}}} & \multicolumn{1}{c}{\makecell[tc]{Frequency \\ with \textvv{dawn}}} & \multicolumn{1}{c}{\makecell[tc]{Frequency \\ with \textvv{beginning}}} & \multicolumn{1}{c}{\makecell[tc]{Other words \\ with \textvv{dawn}}} & \multicolumn{1}{c}{\makecell[tc]{Other words \\ with \textvv{beginning}}} & \multicolumn{1}{c}{\makecell[tc]{\emph{G}}} \\
\midrule
\multicolumn{6}{l}{Most strongly associated with \textvv{dawn of \_\_ century}} \\
\midrule
\textit{the twenty\hyp{}first} & 7 & 14 & 23 & 908 & 24.0907472190799 \\
\textit{another} & 2 & 0 & 28 & 922 & 13.9616651831275 \\
\textit{america's} & 1 & 0 & 29 & 922 & 6.94739452091335 \\
\textit{that} & 1 & 0 & 29 & 922 & 6.94739452091335 \\
\textit{the twentieth} & 7 & 72 & 23 & 850 & 6.531541200179 \\
\midrule
\multicolumn{6}{l}{Most strongly associated with \textvv{beginning of \_\_ century}} \\
\midrule
\textit{this} & 0 & 108 & 30 & 814 & 7.34849897856036 \\
\lspbottomrule
\end{tabular}}
\end{table}

As predicted, the twentieth century is now associated \is{association} with the metaphorical \is{figurative language}\is{metaphor} expression (as is the twenty\hyp{}first). In addition, there is the expression \textit{America's century} in both corpora, and \textit{an American} and \textit{an Asian} in COCA. These, I would argue, do not refer to precise centuries but are to be understood as labels for eras. In sum, I would conclude that the idea of boundedness accounts for the apparent exceptions too, at least in the case of centuries, supporting a cognitive \is{cognitive linguistics} function of \is{figurative language}\is{metaphor} metaphor.

Even if we agree with this conclusion in general, however, it does not preclude a more literary, \is{literary language} rhetorical function for metaphor in addition: while some of the expression pairs investigated in \citet{stefanowitsch_function_2005} are fairly neutral with respect to genre \is{genre} or register, \is{register} metaphorical \is{figurative language}\is{metaphor} \textit{dawn of} intuitively has a distinctly literary flavor. To conclude this section, let us check the distribution \is{distribution!conditional} of the hits for the query outlined above across the text categories defined in the BNC. \is{BNC} \tabref{tab:dawnbeginninggenre} shows the results (note that the categories Spoken Conversation \is{conversation} and Spoken Other from the BNC have been collapsed into a single category here).

\begin{table}
\caption{The expressions \textit{dawn of} and \textit{beginning of} by text category (BNC)}
\label{tab:dawnbeginninggenre}
\resizebox{0.6\textwidth}{!}{%
\begin{tabular}[t]{lccr}
\lsptoprule
 & \multicolumn{2}{c}{\textvv{Expression}} & \\
\textvv{Text Category} & \textvv{dawn of} & \textvv{beginning of} & Total \\
\midrule
\textvv{\makecell[tl]{prose}}
	& \makecell[t]{\begin{tabular}[t]{lS[table-format=2.2]} \small{\textit{Obs.:}} & 54 \\ \small{\textit{Exp.:}} & 50.72 \\ \small{\textit{$\chi^2$:}} & 0.21 \end{tabular}}
	& \makecell[t]{\begin{tabular}[t]{lS[table-format=2.2]} \small{\textit{Obs.:}} & 1324 \\ \small{\textit{Exp.:}} & 1327.28 \\ \small{\textit{$\chi^2$:}} & 0.01 \end{tabular}}
	& 1378 \\[1.1cm]
\textvv{\makecell[tl]{miscellaneous \\ published}}
	& \makecell[t]{\begin{tabular}[t]{lS[table-format=2.2]} \small{\textit{Obs.:}} & 30 \\ \small{\textit{Exp.:}} & 25.14 \\ \small{\textit{$\chi^2$:}} & 0.94 \end{tabular}}
	& \makecell[t]{\begin{tabular}[t]{lS[table-format=2.2]} \small{\textit{Obs.:}} & 653 \\ \small{\textit{Exp.:}} & 657.86 \\ \small{\textit{$\chi^2$:}} & 0.04 \end{tabular}}
	& 683 \\[1.1cm]
\textvv{\makecell[tl]{fiction}}
	& \makecell[t]{\begin{tabular}[t]{lS[table-format=2.2]} \small{\textit{Obs.:}} & 28 \\ \small{\textit{Exp.:}} & 9.50 \\ \small{\textit{$\chi^2$:}} & 36.05 \end{tabular}}
	& \makecell[t]{\begin{tabular}[t]{lS[table-format=2.2]} \small{\textit{Obs.:}} & 230 \\ \small{\textit{Exp.:}} & 248.50 \\ \small{\textit{$\chi^2$:}} & 1.38 \end{tabular}}
	& 258 \\[1.1cm]
\textvv{\makecell[tl]{newspaper}}
	& \makecell[t]{\begin{tabular}[t]{lS[table-format=2.2]} \small{\textit{Obs.:}} & 13 \\ \small{\textit{Exp.:}} & 8.58 \\ \small{\textit{$\chi^2$:}} & 2.28 \end{tabular}}
	& \makecell[t]{\begin{tabular}[t]{lS[table-format=2.2]} \small{\textit{Obs.:}} & 220 \\ \small{\textit{Exp.:}} & 224.42 \\ \small{\textit{$\chi^2$:}} & 0.09 \end{tabular}}
	& 233 \\[1.1cm]
\textvv{\makecell[tl]{academic}}
	& \makecell[t]{\begin{tabular}[t]{lS[table-format=2.2]} \small{\textit{Obs.:}} & 11 \\ \small{\textit{Exp.:}} & 27.31 \\ \small{\textit{$\chi^2$:}} & 9.74 \end{tabular}}
	& \makecell[t]{\begin{tabular}[t]{lS[table-format=2.2]} \small{\textit{Obs.:}} & 731 \\ \small{\textit{Exp.:}} & 714.69 \\ \small{\textit{$\chi^2$:}} & 0.37 \end{tabular}}
	& 742 \\[1.1cm]
\textvv{\makecell[tl]{unpublished}}
	& \makecell[t]{\begin{tabular}[t]{lS[table-format=2.2]} \small{\textit{Obs.:}} & 1 \\ \small{\textit{Exp.:}} & 6.00 \\ \small{\textit{$\chi^2$:}} & 4.17 \end{tabular}}
	& \makecell[t]{\begin{tabular}[t]{lS[table-format=2.2]} \small{\textit{Obs.:}} & 162 \\ \small{\textit{Exp.:}} & 157.00 \\ \small{\textit{$\chi^2$:}} & 0.16 \end{tabular}}
	& 163 \\[1.1cm]
\textvv{\makecell[tl]{spoken \\ (all)}}
	& \makecell[t]{\begin{tabular}[t]{lS[table-format=2.2]} \small{\textit{Obs.:}} & 0 \\ \small{\textit{Exp.:}} & 9.75 \\ \small{\textit{$\chi^2$:}} & 9.75 \end{tabular}}
	& \makecell[t]{\begin{tabular}[t]{lS[table-format=2.2]} \small{\textit{Obs.:}} & 265 \\ \small{\textit{Exp.:}} & 255.24 \\ \small{\textit{$\chi^2$:}} & 0.37 \end{tabular}}
	& 265 \\[1.1cm]
\midrule
Total
	& \makecell[t]{137}
	& \makecell[t]{3585}
	& \makecell[t]{3722} \\
\lspbottomrule
\end{tabular}}
\end{table}
% layout: table shows up only after next section has started...

It is very obvious that the metaphorical \is{figurative language}\is{metaphor} expression \textit{the dawn of} is significantly overrepresented in the text category \textvv{fiction} \is{literary language} and underrepresented in the text categories \textvv{academic} and \textvv{spoken}, corroborating the intuition about the literaryness of the expression. Within this text category, of course, it may well have the cognitive \is{cognitive linguistics} function attributed to it in \citet{stefanowitsch_function_2005}.

This case study demonstrates use of the differential collexeme \is{collostructional analysis}\is{collexeme!differential} analysis (and thus of collocational \is{collocation} methods in general) that goes beyond associations \is{association} between words and other elements of structure and instead uses words and grammatical \is{grammar} patterns as ways of investigating semantic \is{semantics} associations. \is{association} Direct comparisons of literal \is{literalness} and metaphorical \is{figurative language}\is{metaphor} language are rare in the research literature, \is{literary language} so this remains a potentially interesting field of research. The study also demonstrates that the distribution \is{distribution!conditional} of particular metaphorical expressions across varieties, \is{language variety} which can easily be determined in corpora that contain the relevant metadata, \is{metadata} may shed light on the function of those expressions (and of metaphor \is{figurative language}\is{metaphor} in general).

\subsection{Target domains}
\label{sec:targetdomains}

As discussed in \chapref{ch:retrievalannotation}, there are two types of metaphorical \is{figurative language}\is{metaphor} utterances: those that could be interpreted literally \is{literalness} in their entirety (like the example \textit{I am burned up} from \citealt[203]{lakoff_cognitive_1987}), and those that contain vocabulary from both the source and the target domain (like \textit{He was filled with anger}). \is{emotions} \citet{stefanowitsch_happiness_2004, stefanowitsch_words_2006} refers to the latter as \textit{metaphorical patterns}, defined as follows:

\begin{quotation}
A metaphorical \is{figurative language}\is{metaphor} pattern is a multi\hyp{}word expression from a given source domain (SD) into which one or more specific lexical item from a given target domain (TD) have been inserted \citep[66]{stefanowitsch_words_2006}.
\end{quotation}

In the example just cited, the multi\hyp{}word source\hyp{}domain expression would be [NP\textsubscript{container} \textit{be filled with} NP\textsubscript{substance}], the source domain would be that of substances in containers. The target domain word that has been inserted in this expression is \textit{anger}, \is{emotions} yielding the metaphorical \is{figurative language}\is{metaphor} pattern [NP\textsubscript{container} \textit{be filled with} NP\textsubscript{emotion}]. The metaphors instantiated by this pattern include ``an emotion is a substance'' and ``experiencing an emotion is being filled with a substance''.

A metaphorical \is{figurative language}\is{metaphor} pattern analysis \is{metaphorical pattern analysis} of a given target domain (like `anger') \is{emotions} thus proceeds by selecting one or more words that refer to (or are inherently connected with) this domain (for example, the word \textit{anger}, or the set \textit{irritation}, \textit{annoyance}, \textit{anger}, \textit{rage}, \textit{fury}, etc.) and retrieve \is{retrieval} all instances of this word or set of words from a corpus. The next step consists in identifying all cases where the search term(s) occur in a multi\hyp{}word expression referring to some domain other than emotions. \is{emotions} Finally, the source domains of these expressions are identified, giving us the metaphor \is{figurative language}\is{metaphor} instantiated by each metaphorical pattern. The patterns can then be grouped into larger sets corresponding to metaphors like ``emotions are substances''.

% me: Expand discussion of MPA

\subsubsection{Case study: Happiness across cultures}
\label{sec:happinessculture}

\citet{stefanowitsch_happiness_2004} investigates differences in metaphorical \is{figurative language}\is{metaphor} patterns associated \is{association} with \textit{happiness} \is{emotions} in American \is{American English} English and its translation equivalent \textit{Glück} in German. The study finds, among other things, that the metaphors \is{figurative language}\is{metaphor} \textsc{the attempt to achieve happiness is a search\slash pursuit} and \textsc{causing happiness is a transaction} are instantiated more frequently in American English than in German. The question, raised but not addressed in \citet{stefanowitsch_happiness_2004}, is whether this is a linguistic difference or a cultural \is{culture} difference. The word \textit{Glück} is a close translation equivalent of \textit{happiness}, \is{emotions} but the meaning \is{semantics} of these two words is not identical. For example, as \citet{goddard_semantic_1998} argues and \citet{stefanowitsch_happiness_2004} shows empirically (cf. Case Study \ref{sec:fillintensity}), the German word describes a more intense emotion than the English one. This may have consequences for the metaphors \is{figurative language}\is{metaphor} a word is associated \is{association} with. Alternatively, the emotional state described by both \textit{Glück} and \textit{happiness} may play a different role in German vs. American \is{American English} culture, \is{culture} for example, in regard to beliefs about whether and how one can actively try to cause or achieve it. In order to answer this question, we need to compare metaphorical \is{figurative language}\is{metaphor} patterns associated \is{association} with \textit{happiness} \is{emotions} in different English\hyp{}speaking cultures (or patterns associated \is{association} with \textit{Glück} in different German\hyp{}speaking ones).

Let us attempt to do this, focusing on the two metaphors \is{figurative language}\is{metaphor} just mentioned but discussing others in passing. In order to introduce the method of metaphorical pattern analysis, \is{metaphorical pattern analysis} let us limit the study to small samples of language, which will allow us to study the relevant concordances \is{concordance} in detail. This will make it less probable that we will find statistically significant differences, so let us treat the following as an exploratory pilot study. Given how frequently we have compared British \is{British English} and American \is{American English} English in this book, these two varieties \is{language variety} may seem an obvious place to start, but the two cultures \is{culture} may be too similar, and the word \textit{happiness} \is{emotions} happens to be too infrequent in the BROWN \is{BROWN} corpus anyway. Let us compare British English (the LOB \is{LOB} corpus) and Indian \is{Indian English} English (the Kolhapur \is{KOLHAPUR} corpus constructed along the same categories) instead. \figref{fig:happinesslobconc} shows all hits of the query $\langle$ \texttt{[word="happiness"\%c]} $\rangle$.

\begin{figure}
\caption{Concordance of \textit{happiness} (LOB)}
\label{fig:happinesslobconc}
\hrulefill
\begin{fitverb}
 1 rences in the way of life and pursuit of [happiness] , differences in our social system and
 2  and laughter , he feels , engender more [happiness] than politics or philanthropy . at a me
 3 experiences the true meaning of love and [happiness] . ' X-certificate . Phillipe Lemarre ha
 4  used of experiences of life and death , [happiness] and sorrow ( cf Job 9.25 ; Ps 16.10 ; I
 5 or an ultimate goal to the merriment and [happiness] that life does contain in some of its s
 6  says about the relation of goodness and [happiness] . most people know Heine's brilliant je
 7 duty is not concerned with consequence : [happiness] is concerned with nothing else . here w
 8  about the supreme good - which includes [happiness] . A E Taylor has said that what disting
 9 ending improvement need not mean perfect [happiness] there any more than here . but after se
10 ew moral intuition . ` that goodness and [happiness] ought to go together , and the existenc
11 he seems to have overcome the dualism of [happiness] and duty but at a cost . he has been vi
12 dly meets the problem ` does Kant regard [happiness] as a good thing or not ? ' the answer w
13 we prove ourselves worthy or unworthy of [happiness] in the next . but in this life is it no
14 n this life is it not lawful to seek the [happiness] of others ? on stern Kantian grounds ,
15 ng attitudes , to a life of fulfilment , [happiness] and success . as each year passes the s
16 e and the car , will not bring increased [happiness] to our increased leisure . nor will the
17 ge of this fundamental truth - that real [happiness] and satisfaction is found in doing for
18 hich no one will read . '' sign here for [happiness] . Judith Simons meets a woman who share
19 hrough it - that moment when all hope of [happiness] seems lost for ever . they said they 'd
20 nts are able to provide tranquillity and [happiness] within the home itself , and in their d
21  not only in money but in the health and [happiness] of its people and the enhanced prestige
22 hey studied Richard Lucas' enquiry after [happiness] , Norris' sermons , Stephen's letters a
23 ctively in the sacrifice of her sister's [happiness] , or in consolidating her own usurpatio
24 he summertime , sent her into shrieks of [happiness] . she loved bright objects and pleasant
25 us beauty of the Latin liturgy ` a vital [happiness] ' . it was to him a means of mediation
26 ture . to all who have retired , we wish [happiness] and long life . research leaders honour
27 her told stories about the war a curious [happiness] came over him which the stories themsel
28 tomorrow afternoon ? ' he felt a glow of [happiness] steal over him . everything was all rig
29 ee you happy . ' ` there will n't be any [happiness] for me until I can prove him guilty . '
30 ith an undescribable expression of utter [happiness] . seeing Heather he came to her and dan
31  in turn with an expression of ineffable [happiness] on his flat face . quickly taking his c
32 ommand . Sirisec . '' he looked up , all [happiness] gone from his leathery features . ` oh
33 . though he never expected to attain the [happiness] he yearned for in a daughter-in-law and
34  West again . Barry had brought her more [happiness] than she had ever known was possible ,
35 ut there 'll be sons for you - aye , and [happiness] , too - when Helen 's gone from your si
36 y known before that there was no hope of [happiness] in the future for her and Gavin . if he
37 is own love for her , his desire for her [happiness] . far better that she should believe hi
38 e word . Nicholas , Philip ... where was [happiness] , or peace of mind ? Philip put out a h
39 ved Sandra too deeply to ruin her future [happiness] . had ever circumstances conspired so c
40 tood there staring at Julia with all the [happiness] draining out of her pretty little face
41 a burden to be endured and never never a [happiness] to be anticipated . now , her young mou
42 wards he had believed that she had found [happiness] with the bluff sailor and he 'd been ge
43 y nothing of this . it concerns Missie's [happiness] . '' so that was it ! someone was anxio
44 k . ' Mollie followed him , bemused with [happiness] . she moved on a cloud , floating effor
45  they sat for an hour , bemused by their [happiness] , feeling that all things were possible
46 on of Dorcas and Adrian Mallory , of the [happiness] of that girl on the eve of her marriage
47 change , even for a fortnight , the warm [happiness] of being with Neil , of sharing with hi
48 mented minute had been a tiny stretch of [happiness] . he leaned from the carriage window an
49 ery on a fast vanishing hope of ultimate [happiness] . Betty was right . Kay must not be for
50 ' ` yes , and we 'll drink to our future [happiness] , Bill ! ' she answered , raising her f
51 s of golden sunshine and music and utter [happiness] . the knowledge that she might never se
52 em , Tandy felt a private little glow of [happiness] . for so long , now , she 'd felt respo
53 re so selfish . " " The whole concept of [happiness] , mother dearest , is outdated . Your ph
\end{fitverb}
\hrulefill
\end{figure}

The very first line contains an example of one of the \textvv{search\slash pursuit} metaphor, \is{figurative language}\is{metaphor} namely the metaphorical pattern [\textit{pursuit of} NP\textsubscript{EMOT}]. If we go through the entire concordance, \is{concordance} we find three additional patterns instantiating this metaphor, namely [\textit{seek} NP\textsubscript{EMOT}] (line 14), [NP\textsubscript{EMOT} \textit{be found in} V\textsubscript{ing}] (line 17) and [NP\textsubscript{EXP} \textit{find} NP\textsubscript{EMOT}] (line 42), with NP\textsubscript{EXP} indicating the slot for the noun \is{noun} referring to the experiencer of the emotion. \is{emotions} Note that, as is typical in metaphorical \is{figurative language}\is{metaphor} pattern analysis, \is{metaphorical pattern analysis} the patterns are generalized with respect to the slot of the emotion noun (we could find the same patterns with other emotions), and they are relatively close in form to the actual citation. We could subsume the passive \is{passive voice} in line 17 under the same pattern as the active in line 42, of course, but there might be differences across emotion terms, varieties, \is{language variety} etc. concerning voice (and other formal aspects of the pattern), and there is little gained by discarding this information.

The \textvv{transfer} metaphor \is{figurative language}\is{metaphor} is also instantiated a number of times in the concordance, \is{concordance} namely as [NP\textsubscript{STIM} \textit{bring} NP\textsubscript{EMOT}] (lines 16 and 42), and [NP\textsubscript{STIM} \textit{provide} NP\textsubscript{EMOT}] (line 20).

Additional clear cases of metaphorical \is{figurative language}\is{metaphor} patterns are [\textit{glow of} NP\textsubscript{EMOT}] (lines 28 and 53) and [\textit{warm} NP\textsubscript{EMOT}] (line 47), which instantiate the metaphor \textsc{happiness \is{emotions} is warmth}, and [NP\textsubscript{EMOT} \textit{drain out of} NP\textsubscript{EXP}\textit{'s} \textit{face}] (line 40), which instantiates \textsc{happiness is a liquid filling the experiencer}. In other cases, it depends on our judgment (which we have to defend within a given research design) \is{research design} whether a hit constitutes a metaphorical \is{figurative language}\is{metaphor} pattern. For example, do we want to analyze [NP\textsubscript{STIM} \textit{'s} NP\textsubscript{EMOT}] (lines 23 and 43) and [PRON.POSS.\textsubscript{STIM} NP\textsubscript{EMOT}] (lines 37, 39, 45, 50) as \textsc{happiness is a possessed object}, or do we consider the possessive \is{possessive} construction to be too abstract semantically \is{semantics} to be analyzed as metaphorical? Similarly, do we analyze [NP\textsubscript{STIM} \textit{engender} NP\textsubscript{EMOT}] (line 2) as an instance of \textsc{happiness \is{emotions} is an organism}, based on the etymology of \textit{engender}, which comes from Latin \textit{generare} \textit{beget} and was still used for organisms in Middle English (cf. Chaucer's \textit{...swich licour, / Of which vertu engendred is the flour})? You might want to think about these and other cases in the concordance, \is{concordance} to get a sense of the kind of annotation \is{annotation} scheme you would need to make such decisions on a principled, replicable \is{replicability} basis.

For now, let us turn to Indian \is{Indian English} English. \tabref{fig:happinesskolhapur} \is{emotions} shows the hits for the query $\langle$ \texttt{[word="happiness"\%c]} $\rangle$ in the Kolhapur \is{KOLHAPUR} corpus. Here, 35 hits for the phrase \textit{harmonious happiness} have been removed, because they all came from one text extolling the virtues of the \textit{principle of harmonious happiness} (17 hits), \textit{(moral) standard of harmonious happiness} (15 hits), \textit{(moral) good of harmonious happiness)} (2 hits) or \textit{nor of harmonious happiness} \is{emotions} (1 hit). This text is obviously very much an outlier, as it contains almost as many hits as the entire rest of the corpus combined, and as the hits are extremely restricted in their linguistic behavior. To discard them might not seem ideal, but to include them would be even less so.

\begin{figure}
\caption{Concordance of \textit{happiness} (Kolhapur)}
\label{fig:happinesskolhapur}
\hrulefill
\begin{fitverb}
 1 ove . Dr . Patwardhan expresses both her [happiness] at seeing the growth of Anandgram and he
 2 with mature understanding the search for [happiness] of an actress . The Shyam Benegal and Bl
 3 nd lives on her earnings makes Usha seek [happiness] elsewhere . The search for happiness of
 4 eek happiness elsewhere . The search for [happiness] of this intensely sensitive girl leads h
 5  are seeking something , seeking peace , [happiness] , seeking a nobler way of life , seeking
 6 e occasion , the Buddha himself bringing [happiness] to a doomed city , and accordingly , the
 7  their suffering and obtain security and [happiness] is by seeking to change and transform so
 8 e and notoriety , censure and praise and [happiness] and misery . Just as the stalk gives bir
 9 tute , consoled the stricken and brought [happiness] to the miserable . He did not run away f
10 th the physical body is another name for [happiness] . Finally , when the mind is stilled ( i
11 d through such purity of mind to achieve [happiness] . It also says that if one acts or speak
12 or control of mind which is conducive to [happiness] because it flits and floats all over and
13 ual cooperation , the key to success and [happiness] , are at a discount . Even people who ar
14 edas there are many prayers for wealth , [happiness] and glory . " We call on Thee for prospe
15 from sin and full of wealth , leading to [happiness] day by day . " ( Rig ) " May I be glorio
16  I am confident that I can sing to bring [happiness] to my listeners and fulfilment to myself
17 se wanderers together again and there is [happiness] . When they all return to Jaipur they di
18  his old position . Thus there is double [happiness] for all . This plot will give an idea of
19 s possible at the cost of the people ' s [happiness] . The freedom fighter for India against
20 is this ennobling vision of the world of [happiness] and contentment which I have always born
21 eace alone there is human fulfilment and [happiness] . But even if the goal appears distant ,
22 with the peace and prosperity , life and [happiness] of the society ? The only answer to this
23 of you . I wish you every prosperity and [happiness] in the coming years . I have served you
24 ull of vegetation , trees , orchards and [happiness] . But she could not do that for she was
25 ould make her if you did . Learn to give [happiness] to people , all you modern children are
26  didn ' t wish to come in the way of our [happiness] , at which she , my wife , pretended to
27 hree children . They did not know of the [happiness] we shared -- the exciting excursions , t
28 for breakfast with us . It gave us great [happiness] , though he was not his cheerful old sel
29 t more could she desire ? The goddess of [happiness] and mirth had visited her . Forty-five m
30 ure of health . Nobody was bursting with [happiness] ; there was no expectation of sharing in
31  pronounced my son as completely cured . [Happiness] flooded my heart . Silently I held my wi
32 l . He was filled with a kind of childsh [happiness] . He wanted to scamper over the rocks ,
33  said Joan . Janaki ' s face beamed with [happiness] at the comparison . From that day , Joan
34  had a warm sniff of its steam . But his [happiness] drove him into one of those sudden snooz
35 on ' s ? Have I always placed Dinesh ' s [happiness] above mine ? Am I not selfish and posses
36 me in ample measure for the pleasure and [happiness] my stories and novels have brought into
37 tting together , bit by bit . Moments of [happiness] are such fleeting things . Maybe they al
38 leeting things . Maybe they always are . [Happiness] - - maybe it ' s just the burden of a bi
39 is door had played a stellar role in his [happiness] . Day after day , he had sat there like
40 rtainly put me on the highway to eternal [happiness] but it could do nothing about my immedia
41 rs will travel through life in peace and [happiness] in spite of delays , discomfort and suff
\end{fitverb}
\hrulefill
\end{figure}

Again, the \textvv{search} metaphor \is{figurative language}\is{metaphor} is instantiated several times in the concordance: \is{concordance} we find [\textit{search for} NP\textsubscript{EMOT}] (lines 2 and 4) and [NP\textsubscript{EXP} \textit{seek} NP\textsubscript{EMOT}] (lines 3 and 5). They all seem to be from the same text, so similar considerations apply as in the case of the \textit{principle\slash standard of harmonious happiness}. \is{emotions} We also find the transfer metaphor quite strongly represented: [NP\textsubscript{STIM} \textit{bring} NP\textsubscript{EMOT}] (lines 6, 9, 16), [NP\textsubscript{EXP} \textit{obtain} NP\textsubscript{EMOT}] (line 7), [NP\textsubscript{STIM} \textit{give} NP\textsubscript{EMOT}] (lines 25, 28) and [NP\textsubscript{EXP} \textit{share} NP\textsubscript{EMOT}] (line 27).

Again, there are other clear cases of metaphor, \is{figurative language}\is{metaphor} such as [NP\textsubscript{EXP} \textit{burst with} NP\textsubscript{EMOT}] (line 30), [NP\textsubscript{EMOT} \textit{flood} NP\textsubscript{EXP}\textit{'s} \textit{heart}] (line 31), and [NP\textsubscript{EXP} \textit{be filled with} NP\textsubscript{EMOT}] (line 32) (again, they seem to be from the same text).

Comparing the metaphors \is{figurative language}\is{metaphor} we set out to investigate, we see that the \textvv{pursuit} and \textvv{search} metaphors are fairly evenly distributed \is{distribution!conditional} across the two varieties, \is{language variety} with no significant difference even on the horizon. The \textvv{transfer} metaphor, in contrast, shows clear differences, and as \tabref{tab:transferlobkolhapur} shows, this difference is almost significant even in our small sample ($\chi^2 = 3.17, \df = 1, p = 0.075$). This would be an interesting difference to look at in a larger study, especially since the two varieties \is{language variety} differ not only in the token \is{token (instance)} frequency \is{frequency!token} of this metaphor, \is{figurative language}\is{metaphor} but also in the type \is{type (category)} frequency \is{frequency!type} -- the LOB \is{LOB} corpus contains only two different patterns instantiating this metaphor, the Kolhapur \is{KOLHAPUR} corpus contains four.

\begin{table}
\caption{\textvv{causing happiness is a transfer} in two corpora (LOB, Kolhapur)}
\label{tab:transferlobkolhapur}
\begin{tabular}[t]{llccr}
\lsptoprule
 & & \multicolumn{2}{c}{\textvv{Corpus}} & \\
 & & \textvv{lob} & \textvv{kolhapur} & Total \\
\midrule
\textvv{\makecell[lt]{Type}}
	& \textvv{transfer}
		& \makecell[t]{\num{3}\\\small{(\num{5.64})}}
		& \makecell[t]{\num{7}\\\small{(\num{4.36})}}
		& \makecell[t]{\num{10}\\} \\
	& \textvv{$\neg$transfer}
		& \makecell[t]{\num{50}\\\small{(\num{47.36})}}
		& \makecell[t]{\num{34}\\\small{(\num{36.64})}}
		& \makecell[t]{\num{84}\\} \\
\midrule
	& Total
		& \makecell[t]{\num{53}}
		& \makecell[t]{\num{41}}
		& \makecell[t]{\num{94}} \\
\lspbottomrule
\end{tabular}
\end{table}
% me: chisq.test(matrix(c(3,50,7,34),ncol=2),corr=FALSE)

This case study demonstrates the basic procedure of Metaphorical \is{figurative language}\is{metaphor} Pattern Analysis \is{metaphorical pattern analysis} and some of the questions raised by the need to categorize \is{categorization} corpus hits. It also shows that even a small\hyp{}scale study of such patterns may provide interesting results on which we can build hypotheses to be tested on larger \is{corpus size} data sets. Finally, it shows that there may well be differences between cultures \is{culture} in the metaphorical \is{figurative language}\is{metaphor} patterns and the overarching metaphors they instantiate (cf. e.g. \citealt{rojo_lopez_metaphorical_2010}, \citealt{rojo_lopez_distinguishing_2013} and \citealt{ogarkova_emotion_2014} for cross\hyp{}linguistic comparisons and \citealt{diaz-vera_exploring_2013}, \citealt{diaz-vera_love_2015} and \citealt{guldenring_emotion_2017} for comparisons across different varieties \is{language variety} of English; cf. also \citealt{tissari_lovescapes:_2003, tissari_english_2010} for comparisons across time periods within one language).

\subsubsection{Case study: Intensity of emotions}
\label{sec:fillintensity}

Whether we start from the source domain or from the target domain, the extraction of metaphorical \is{figurative language}\is{metaphor} patterns from large \is{corpus size} corpora typically requires time\hyp{}consuming manual \is{manual analysis} annotation. \is{annotation} However, if we are interested in specific metaphors, we can speed up the extraction \is{retrieval} significantly, by looking for utterances containing vocabulary from the source and target domains we are interested in. For example, \citet{martin_corpus-based_2006} compiles lists of words from two domains (such as \textvv{war} and \textvv{commercial activity} and then searches a large \is{corpus size} corpus for instances where items from both lists co\hyp{}occur in a particular \is{span} span.

We can take this basic idea and apply it in a linguistically slightly more conservative way to target\hyp{}item oriented versions of metaphorical \is{figurative language}\is{metaphor} pattern analysis. \is{metaphorical pattern analysis} Instead of searching for co\hyp{}occurrence in a span, \is{span} let us construct a set of structured queries \is{query} that would find metaphorical patterns instantiating a given metaphor. As mentioned in Case Study \ref{sec:happinessculture}, the word \textit{happiness} and its German translation equivalent \textit{Glück} may differ in the intensity of the emotion \is{emotions} they refer to. In \citet{stefanowitsch_happiness_2004}, this hypothesis is tested by investigating metaphors \is{figurative language}\is{metaphor} that plausibly reflect such a difference in intensity. Specifically, it is shown that while both \textit{happiness} and \textit{Glück} are found with metaphors describing an emotion as a substance in a container (the experiencer), \textit{Glück} is found significantly more frequently with metaphors \is{figurative language}\is{metaphor} describing the experiencer as a container that disintegrates because it cannot withstand the pressure of the substance. The same difference is found within English between the words \textit{happiness} \is{emotions} and \textit{joy}.

Let us investigate these metaphors \is{figurative language}\is{metaphor} with respect to frequent basic emotion terms in English, to see whether there are general differences between emotions with respect to metaphorical intensity. \citet{stefanowitsch_words_2006} lists the metaphorical patterns for five English emotion terms, categorized \is{categorization} into general metaphors. \is{figurative language}\is{metaphor} Combining all patterns occurring with at least one emotion \is{emotions} noun, \is{noun} the patterns in (\ref{ex:emotionliquid}) are identified or the metaphor \textvv{emotions are a substance in a container} with \textit{an experiencer is a container} (instead of the experiencer themselves, the patterns may also refer to their \textit{heart}, \textit{face}, \textit{eyes}, \textit{mind}, \textit{voice}, etc.):

\begin{exe}
\ex
\begin{xlist}
\label{ex:emotionliquid}
\ex NP\textsubscript{EMOT} \textit{fill} NP\textsubscript{EXP}
\ex NP\textsubscript{STIM} \textit{fill} NP\textsubscript{EXP} with NP\textsubscript{EMOT}
\ex NP\textsubscript{EXP} \textit{fill} (up) with NP\textsubscript{EMOT}
\ex NP\textsubscript{EXP} \textit{be filled with} NP\textsubscript{EMOT}
\ex NP\textsubscript{EXP} \textit{be full of} NP\textsubscript{EMOT}
\ex NP\textsubscript{EXP} \textit{be(come)} filled with NP\textsubscript{EMOT}
\ex NP\textsubscript{EMOT} \textit{seep into} NP\textsubscript{EXP}
\ex NP\textsubscript{EMOT} \textit{spill into} NP\textsubscript{EXP}
\ex NP\textsubscript{EMOT} \textit{flood through} NP\textsubscript{EXP}
\end{xlist}
\end{exe}

Let us ignore the patterns in (\ref{ex:emotionliquid}g--i), \is{emotions} which occurred only once each in \citet{stefanowitsch_words_2006}. The rest then form a set of expressions with relatively simple structures that we can extract \is{retrieval} using the following queries (shown here for the noun \is{noun} \textit{happiness}):

\begin{exe}
\ex
\begin{xlist}
\label{ex:emotionliquidquery}
\ex \texttt{[hw="full"] [hw="of"] []\{0,2\} [word="happiness"\%c]}
\ex \texttt{[hw="(fill|filled)"] [hw="with"] []\{0,2\} [word="happiness"\%c]}
\ex \begin{minipage}[t]{0.85\textwidth} \raggedright \texttt{[word="happiness"\%c] [pos=".*(VB|VD|VH|VM).*"]} \texttt{[pos=".*AV0.*"] [hw="fill" \& pos=".*VV.*"]} \end{minipage}
\end{xlist}
\end{exe}
% me: query: (([hw="full"][word="of"%c]|[hw="(fill|filled)"][]?[word="with"]) []{0,2} @[word="(fear|desire|anger|pride|shame|happiness|sadness|disgust)"%c & pos=".*NN.*"] | @[word="(fear|desire|anger|pride|shame|happiness|sadness|disgust)"%c & pos=".*NN.*"][pos=".*(VB|VD|VH|VM).*"]?[pos=".*AV0.*"]?[hw="fill" & pos=".*VV.*"])

The paper also lists a number of metaphorical \is{figurative language}\is{metaphor} patterns that describe an increasing pressure, e.g. [NP\textsubscript{EMOT} \textit{build inside} NP\textsubscript{EXP}] or an overflowing, e.g. [NP\textsubscript{EXP} \textit{brim over with} NP\textsubscript{EMOT}], but let us focus on those patterns that describe a sudden failure to contain the substance. These are listed in \is{emotions} (\ref{ex:emotionpressure}):

\begin{exe}
\ex
\begin{xlist}
\label{ex:emotionpressure}
\ex \textit{burst\slash outburst\slash explosion of} NP\textsubscript{EMOT}
\ex NP\textsubscript{EMOT} \textit{burst in\slash through} NP\textsubscript{EXP}
\ex NP\textsubscript{EMOT} \textit{burst (out)}
\ex NP\textsubscript{EMOT} \textit{erupt\slash explode}
\ex NP\textsubscript{EMOT} \textit{blow up}
\ex NP\textsubscript{EXP} \textit{burst\slash erupt\slash explode with} NP\textsubscript{EMOT}
\ex \textit{explosive} NP\textsubscript{EMOT}
\ex \textit{volcanic} NP\textsubscript{EMOT}
\end{xlist}
\end{exe}

We can extract \is{retrieval} these patterns using the following queries (shown, again, for \is{emotions} \textit{happiness}):

\begin{exe}
\ex
\begin{xlist}
\label{ex:emotionburstingquery}
\ex \begin{minipage}[t]{0.85\textwidth} \raggedright \texttt{[word="happiness"\%c \& pos=".*NN.*"]} \texttt{[hw="(burst|\allowbreak erupt|\allowbreak explode)" \& pos=".*VV.*"]} \end{minipage}
\ex \begin{minipage}[t]{0.85\textwidth} \raggedright \texttt{[word="happiness"\%c \& pos=".*NN.*"]} \texttt{[hw="(blow)" \& pos=".*VV.*"][word="up"\%c]} \end{minipage}
\ex \begin{minipage}[t]{0.85\textwidth} \raggedright \texttt{[hw="(explosive|\allowbreak eruptive|\allowbreak volcanic)"]} \texttt{[word="happiness"\%c \& pos=".*NN.*"]} \end{minipage}
\ex \begin{minipage}[t]{0.85\textwidth} \raggedright \texttt{[hw="(burst|\allowbreak erupt|\allowbreak explode)" \& pos=".*VV.*"] [word="(in|\allowbreak with)"\%c]} \texttt{[word="happiness"\%c \& pos=".*NN.*"]} \end{minipage}
\ex \begin{minipage}[t]{0.85\textwidth} \raggedright \texttt{[hw="(outburst|\allowbreak burst|\allowbreak eruption|\allowbreak explosion)"] [word="of"\%c]} \texttt{[word="happiness"\%c \& pos=".*NN.*"]} \end{minipage}
\end{xlist}
\end{exe}
% me: query: ( @[word="(fear|desire|anger|pride|shame|happiness|sadness|disgust)"%c & pos=".*NN.*"] ( [hw="(burst|erupt|explode)" & pos=".*VV.*"] | [hw="(blow)" & pos=".*VV.*"][word="up"%c] ) | ( [hw="(explosive|eruptive|volcanic)"] | [hw="(burst|erupt|explode)" & pos=".*VV.*"] [word="(in|with)"%c] | [hw="(outburst|burst|eruption|explosion)"][word="of"%c] ) @[word="(fear|desire|anger|pride|shame|happiness|sadness|disgust)"%c & pos=".*NN.*"])

We are now searching for combinations of a target\hyp{}domain item with specific source domain items in specific structural configurations. In doing so, we will reduce recall \is{recall} compared to a manual \is{manual analysis} extraction, \is{retrieval} but the precision \is{precision} is very high, allowing us to query large \is{corpus size} corpora and to work with the results without annotating \is{annotation} them manually.

Let us apply both sets of queries to the basic emotion \is{emotions} nouns \is{noun} \textit{anger}, \textit{desire}, \textit{disgust}, \textit{fear}, \textit{happiness}, \textit{pride}, \textit{sadness}, \textit{shame}. \tabref{tab:fillemotions} shows the tabulated results for the queries in (\ref{ex:emotionliquidquery}) compared against the overall frequency \is{frequency} of the respective nouns in the \is{BNC} BNC.

\begin{table}
\caption{The metaphor \textvv{emotions are a substance filling the experiencer} (BNC)}
\label{tab:fillemotions}
\resizebox{0.8\textwidth}{!}{%
\begin{tabular}[t]{lccr}
\lsptoprule
 & \multicolumn{2}{c}{\textvv{filling\slash fullness} metaphors} & \\
\textvv{Emotion} & \textvv{yes} & \textvv{no} & Total \\
\midrule
\textvv{\makecell[tl]{\textvv{anger}}}
	& \makecell[t]{\begin{tabular}[t]{lS[table-format=4.2]} \small{\textit{Obs.:}} & 29 \\ \small{\textit{Exp.:}} & 21.07 \\ \small{\textit{$\chi^2$:}} & 2.99 \end{tabular}}
	& \makecell[t]{\begin{tabular}[t]{lS[table-format=4.2]} \small{\textit{Obs.:}} & 3512 \\ \small{\textit{Exp.:}} & 3519.93 \\ \small{\textit{$\chi^2$:}} & 0.02 \end{tabular}}
	& 3541 \\[1.1cm]

\textvv{\makecell[tl]{\textvv{desire}}}
	& \makecell[t]{\begin{tabular}[t]{lS[table-format=4.2]} \small{\textit{Obs.:}} & 7 \\ \small{\textit{Exp.:}} & 30.12 \\ \small{\textit{$\chi^2$:}} & 17.74 \end{tabular}}
	& \makecell[t]{\begin{tabular}[t]{lS[table-format=4.2]} \small{\textit{Obs.:}} & 5055 \\ \small{\textit{Exp.:}} & 5031.88 \\ \small{\textit{$\chi^2$:}} & 0.11 \end{tabular}}
	& 5062 \\[1.1cm]

\textvv{\makecell[tl]{\textvv{disgust}}}
	& \makecell[t]{\begin{tabular}[t]{lS[table-format=4.2]} \small{\textit{Obs.:}} & 6 \\ \small{\textit{Exp.:}} & 3.63 \\ \small{\textit{$\chi^2$:}} & 1.55 \end{tabular}}
	& \makecell[t]{\begin{tabular}[t]{lS[table-format=4.2]} \small{\textit{Obs.:}} & 604 \\ \small{\textit{Exp.:}} & 606.37 \\ \small{\textit{$\chi^2$:}} & 0.01 \end{tabular}}
	& 610 \\[1.1cm]

\textvv{\makecell[tl]{\textvv{fear}}}
	& \makecell[t]{\begin{tabular}[t]{lS[table-format=4.2]} \small{\textit{Obs.:}} & 47 \\ \small{\textit{Exp.:}} & 42.62 \\ \small{\textit{$\chi^2$:}} & 0.45 \end{tabular}}
	& \makecell[t]{\begin{tabular}[t]{lS[table-format=4.2]} \small{\textit{Obs.:}} & 7117 \\ \small{\textit{Exp.:}} & 7121.38 \\ \small{\textit{$\chi^2$:}} & 0.00 \end{tabular}}
	& 7164 \\[1.1cm]

\textvv{\makecell[tl]{\textvv{happiness}}}
	& \makecell[t]{\begin{tabular}[t]{lS[table-format=4.2]} \small{\textit{Obs.:}} & 15 \\ \small{\textit{Exp.:}} & 9.61 \\ \small{\textit{$\chi^2$:}} & 3.02 \end{tabular}}
	& \makecell[t]{\begin{tabular}[t]{lS[table-format=4.2]} \small{\textit{Obs.:}} & 1601 \\ \small{\textit{Exp.:}} & 1606.39 \\ \small{\textit{$\chi^2$:}} & 0.02 \end{tabular}}
	& 1616 \\[1.1cm]

\textvv{\makecell[tl]{\textvv{pride}}}
	& \makecell[t]{\begin{tabular}[t]{lS[table-format=4.2]} \small{\textit{Obs.:}} & 11 \\ \small{\textit{Exp.:}} & 16.21 \\ \small{\textit{$\chi^2$:}} & 1.68 \end{tabular}}
	& \makecell[t]{\begin{tabular}[t]{lS[table-format=4.2]} \small{\textit{Obs.:}} & 2714 \\ \small{\textit{Exp.:}} & 2708.79 \\ \small{\textit{$\chi^2$:}} & 0.01 \end{tabular}}
	& 2725 \\[1.1cm]

\textvv{\makecell[tl]{\textvv{sadness}}}
	& \makecell[t]{\begin{tabular}[t]{lS[table-format=4.2]} \small{\textit{Obs.:}} & 13 \\ \small{\textit{Exp.:}} & 4.47 \\ \small{\textit{$\chi^2$:}} & 16.29 \end{tabular}}
	& \makecell[t]{\begin{tabular}[t]{lS[table-format=4.2]} \small{\textit{Obs.:}} & 738 \\ \small{\textit{Exp.:}} & 746.53 \\ \small{\textit{$\chi^2$:}} & 0.10 \end{tabular}}
	& 751 \\[1.1cm]

\textvv{\makecell[tl]{\textvv{shame}}}
	& \makecell[t]{\begin{tabular}[t]{lS[table-format=4.2]} \small{\textit{Obs.:}} & 11 \\ \small{\textit{Exp.:}} & 11.27 \\ \small{\textit{$\chi^2$:}} & 0.01 \end{tabular}}
	& \makecell[t]{\begin{tabular}[t]{lS[table-format=4.2]} \small{\textit{Obs.:}} & 1883 \\ \small{\textit{Exp.:}} & 1882.73 \\ \small{\textit{$\chi^2$:}} & 0.00 \end{tabular}}
	& 1894 \\[1.1cm]
\midrule
Total
	& \makecell[tS]{139}
	& \makecell[tS]{23224}
	& \makecell[tS]{23363} \\
\lspbottomrule
\end{tabular}}
\end{table}
% me: query: (([hw="full"][word="of"%c]|[hw="(fill|filled)"][]?[word="with"]) []{0,2} @[word="(fear|desire|anger|pride|shame|happiness|sadness|disgust)"%c & pos=".*NN.*"] | @[word="(fear|desire|anger|pride|shame|happiness|sadness|disgust)"%c & pos=".*NN.*"][pos=".*(VB|VD|VH|VM).*"]?[pos=".*AV0.*"]?[hw="fill" & pos=".*VV.*"])

There are two emotion \is{emotions} nouns \is{noun} whose frequency \is{frequency!observed} in these metaphorical \is{figurative language}\is{metaphor} patterns deviates from the expected \is{frequency!expected} frequency: \textit{desire} is described as a substance in a container less frequently than expected, and \textit{sadness} is more frequently than expected. It is an interesting question why this should be the case, but it is not plausibly related to the intensity of the respective \is{emotions} emotions.

\tabref{tab:burstemotions} shows the tabulated results for the queries in (\ref{ex:emotionburstingquery}) compared against the overall frequency of the respective nouns \is{noun} in the \is{BNC} BNC.

\begin{table}
\caption{The metaphor \textvv{emotions are a substance bursting the experiencer} (BNC)}
\label{tab:burstemotions}
%\resizebox*{!}{\textheight}{%
\begin{tabular}[t]{lccr}
\lsptoprule
 & \multicolumn{2}{c}{\textvv{bursting\slash exploding} metaphors} & \\
\textvv{Emotion} & \textvv{yes} & \textvv{no} & Total \\
\midrule

\textvv{\makecell[tl]{\textvv{anger}}}
	& \makecell[t]{\begin{tabular}[t]{lS[table-format=4.2]} \small{\textit{Obs.:}} & 42 \\ \small{\textit{Exp.:}} & 9.40 \\ \small{\textit{$\chi^2$:}} & 113.12 \end{tabular}}
	& \makecell[t]{\begin{tabular}[t]{lS[table-format=4.2]} \small{\textit{Obs.:}} & 3499 \\ \small{\textit{Exp.:}} & 3531.60 \\ \small{\textit{$\chi^2$:}} & 0.30 \end{tabular}}
	& 3541 \\[1.1cm]

\textvv{\makecell[tl]{\textvv{desire}}}
	& \makecell[t]{\begin{tabular}[t]{lS[table-format=4.2]} \small{\textit{Obs.:}} & 4 \\ \small{\textit{Exp.:}} & 13.43 \\ \small{\textit{$\chi^2$:}} & 6.62 \end{tabular}}
	& \makecell[t]{\begin{tabular}[t]{lS[table-format=4.2]} \small{\textit{Obs.:}} & 5058 \\ \small{\textit{Exp.:}} & 5048.57 \\ \small{\textit{$\chi^2$:}} & 0.02 \end{tabular}}
	& 5062 \\[1.1cm]

\textvv{\makecell[tl]{\textvv{disgust}}}
	& \makecell[t]{\begin{tabular}[t]{lS[table-format=4.2]} \small{\textit{Obs.:}} & 1 \\ \small{\textit{Exp.:}} & 1.62 \\ \small{\textit{$\chi^2$:}} & 0.24 \end{tabular}}
	& \makecell[t]{\begin{tabular}[t]{lS[table-format=4.2]} \small{\textit{Obs.:}} & 609 \\ \small{\textit{Exp.:}} & 608.38 \\ \small{\textit{$\chi^2$:}} & 0.00 \end{tabular}}
	& 610 \\[1.1cm]

\textvv{\makecell[tl]{\textvv{fear}}}
	& \makecell[t]{\begin{tabular}[t]{lS[table-format=4.2]} \small{\textit{Obs.:}} & 1 \\ \small{\textit{Exp.:}} & 19.01 \\ \small{\textit{$\chi^2$:}} & 17.06 \end{tabular}}
	& \makecell[t]{\begin{tabular}[t]{lS[table-format=4.2]} \small{\textit{Obs.:}} & 7163 \\ \small{\textit{Exp.:}} & 7144.99 \\ \small{\textit{$\chi^2$:}} & 0.05 \end{tabular}}
	& 7164 \\[1.1cm]

\textvv{\makecell[tl]{\textvv{happiness}}}
	& \makecell[t]{\begin{tabular}[t]{lS[table-format=4.2]} \small{\textit{Obs.:}} & 2 \\ \small{\textit{Exp.:}} & 4.29 \\ \small{\textit{$\chi^2$:}} & 1.22 \end{tabular}}
	& \makecell[t]{\begin{tabular}[t]{lS[table-format=4.2]} \small{\textit{Obs.:}} & 1614 \\ \small{\textit{Exp.:}} & 1611.71 \\ \small{\textit{$\chi^2$:}} & 0.00 \end{tabular}}
	& 1616 \\[1.1cm]

\textvv{\makecell[tl]{\textvv{pride}}}
	& \makecell[t]{\begin{tabular}[t]{lS[table-format=4.2]} \small{\textit{Obs.:}} & 12 \\ \small{\textit{Exp.:}} & 7.23 \\ \small{\textit{$\chi^2$:}} & 3.14 \end{tabular}}
	& \makecell[t]{\begin{tabular}[t]{lS[table-format=4.2]} \small{\textit{Obs.:}} & 2713 \\ \small{\textit{Exp.:}} & 2717.77 \\ \small{\textit{$\chi^2$:}} & 0.01 \end{tabular}}
	& 2725 \\[1.1cm]

\textvv{\makecell[tl]{\textvv{sadness}}}
	& \makecell[t]{\begin{tabular}[t]{lS[table-format=4.2]} \small{\textit{Obs.:}} & 0 \\ \small{\textit{Exp.:}} & 1.99 \\ \small{\textit{$\chi^2$:}} & 1.99 \end{tabular}}
	& \makecell[t]{\begin{tabular}[t]{lS[table-format=4.2]} \small{\textit{Obs.:}} & 751 \\ \small{\textit{Exp.:}} & 749.01 \\ \small{\textit{$\chi^2$:}} & 0.01 \end{tabular}}
	& 751 \\[1.1cm]

\textvv{\makecell[tl]{\textvv{shame}}}
	& \makecell[t]{\begin{tabular}[t]{lS[table-format=4.2]} \small{\textit{Obs.:}} & 0 \\ \small{\textit{Exp.:}} & 5.03 \\ \small{\textit{$\chi^2$:}} & 5.03 \end{tabular}}
	& \makecell[t]{\begin{tabular}[t]{lS[table-format=4.2]} \small{\textit{Obs.:}} & 1894 \\ \small{\textit{Exp.:}} & 1888.97 \\ \small{\textit{$\chi^2$:}} & 0.01 \end{tabular}}
	& 1894 \\[1.1cm]
\midrule
Total
	& \makecell[tS]{62}
	& \makecell[tS]{23301}
	& \makecell[tS]{23363} \\
\lspbottomrule
\end{tabular}%}
\end{table}
% me: query: ( @[word="(fear|desire|anger|pride|shame|happiness|sadness|disgust)"%c & pos=".*NN.*"] ( [hw="(burst|erupt|explode)" & pos=".*VV.*"] | [hw="(blow)" & pos=".*VV.*"][word="up"%c] ) | ( [hw="(explosive|eruptive|volcanic)"] | [hw="(burst|erupt|explode)" & pos=".*VV.*"] [word="(in|with)"%c] | [hw="(outburst|burst|eruption|explosion)"][word="of"%c] ) @[word="(fear|desire|anger|pride|shame|happiness|sadness|disgust)"%c & pos=".*NN.*"])

Two emotion \is{emotions} nouns \is{noun} occur with the \textvv{bursting} metaphors \is{figurative language}\is{metaphor} noticeably more frequently than expected, \is{frequency!expected} namely \textit{anger} and \textit{pride} (the latter marginally significantly); three occur noticeably less frequently, \textit{desire}, \textit{fear} and \textit{shame}. Again, this does not seem to be related to the intensity of the emotion -- fear or desire can be just as intense as anger or pride. Instead, it seems to be related to the likelihood that the emotion will be outwardly visible, for example, by resulting in a particular kind of behavior toward others. Again, \textit{desire} is an exception, it may be that it does not participate in \textit{container} metaphors \is{figurative language}\is{metaphor} at all.

This case study demonstrates how central metaphors for a given target domain can be identified by searching for combinations of words describing (aspects of) the source and the target domain in question. It also shows that these metaphors \is{figurative language}\is{metaphor} can be associated \is{association} to different degrees with different words within a given target domain (cf. e.g. \citealt{stefanowitsch_words_2006}, \citealt{turkkila_near-synonyms_2014}). It is unclear to what extent this lexeme specificity of metaphorical mappings supports or contradicts current cognitive \is{cognitive linguistics} theories of metaphor, \is{figurative language}\is{metaphor} so it is potentially a very interesting area of research.

\subsection{Metaphor and text}
\label{sec:metaphorandtext}

\subsubsection{Case study: Identifying potential source domains}
\label{sec:identifyingpotentialsourcedomains}

If we are interested in questions relating to a particular source domain, as in Section~\ref{sec:sourcedomains}, or to a particular target domain, as in Section~\ref{sec:targetdomains}, our first task is to define a representative \is{representativeness} set of lexical items to query. \is{query} This set may be dictated by the hypothesis we are planning to test, or we may, in more exploratory studies, assemble it on the basis of thesauri or words and patterns identified in previous studies. If, on the other hand, we are interested in source domains associated \is{association} with a particular target domain, matters are more complicated. We can start by selecting a set of target\hyp{}domain items and then identify all source domains in the metaphorical \is{figurative language}\is{metaphor} patterns of these items, but while this will tell us something about these items, it does not tell us much about the target domain as a whole. Or we can identify the source domains manually, \is{manual analysis} which restricts the amount of data we can reasonably process.

\citet{partington_patterns_1998} suggests a promising solution to this problem: If we apply a keyword \is{keyword analysis} analysis (cf. \chapref{ch:text}) to a thematic corpus dealing with the target domain in question, then the dominant source domains in that target domain should be visibly represented among the keywords. Thus, searching the results for items that do not, in their literal \is{literalness} meaning, \is{semantics} belong to the target domain should identify items that ar used metaphorically \is{figurative language}\is{metaphor} in the target domain.

To demonstrate this method, let us define a subcorpus for the domain \textvv{economy} in the BNC Baby. \is{BNC Baby} This is easiest done by using the meta\hyp{}information supplied by the corpus makers, which includes the category ``commerce'' as a subcategory of ``newspaper'' \is{newspaper language} (cf. \tabref{tab:bncbabycomposition} in \chapref{ch:corpuslinguistics}). Let us determine the keywords \is{keyword analysis} in this subcorpus compared to the rest of the \textvv{newspaper} subcorpus (excluding the \textvv{spoken}, \is{medium} \textvv{fiction} \is{literary language} and \textvv{academic} \is{academic language} subcorpora, in order to reduce the number of keywords that are typical for newspaper language in general). \tabref{tab:econkey} shows selected results of a keyword \is{keyword analysis} analysis of these files.

\begin{table}
\caption{Selected keywords from the newspaper subcategory \textvv{commerce} (BNC Baby)}
\label{tab:econkey}
\resizebox{.9\textwidth}{!}{%
\begin{tabular}[t]{r l *{2}{S[table-format=4]} *{2}{S[table-format=8]} S}
\lsptoprule
Rank & \multicolumn{1}{c}{\makecell[tc]{\textvv{Keyword}}} & \multicolumn{1}{c}{\makecell[tc]{Frequency in \\ \textit{commerce}}} & \multicolumn{1}{c}{\makecell[tc]{Frequency in \\ \textit{other}}} & \multicolumn{1}{c}{\makecell[tc]{Other words \\ in \textit{commerce}}} & \multicolumn{1}{c}{\makecell[tc]{Other words \\ in \textit{other}}} & \multicolumn{1}{c}{\makecell[tc]{\emph{G}}} \\
\midrule
\multicolumn{6}{l}{Top 15} \\
\midrule
1 & \textit{share} & 576 & 173 & 170065 & 913288 & 1381.10884190827 \\
2 & \textit{profit} & 443 & 54 & 170198 & 913407 & 1315.96334647513 \\
3 & \textit{company} & 561 & 409 & 170080 & 913052 & 894.985577370924 \\
4 & \textit{market} & 403 & 246 & 170238 & 913215 & 713.797585448087 \\
5 & \textit{p.c.} & 175 & 0 & 170466 & 913461 & 647.282149703015 \\
6 & \textit{group} & 434 & 394 & 170207 & 913067 & 594.566176335946 \\
7 & \textit{business} & 372 & 287 & 170269 & 913174 & 571.849622580703 \\
8 & \textit{Middlesbrough} & 203 & 39 & 170438 & 913422 & 550.493409710853 \\
9 & \textit{investor} & 161 & 6 & 170480 & 913455 & 545.845248241068 \\
10 & \textit{rate} & 245 & 154 & 170396 & 913307 & 426.771066393312 \\
11 & \textit{dividend} & 115 & 3 & 170526 & 913458 & 398.394204301664 \\
12 & \textit{price} & 262 & 211 & 170379 & 913250 & 391.159091460529 \\
13 & \textit{investment} & 164 & 52 & 170477 & 913409 & 385.94864847718 \\
14 & \textit{rise} & 221 & 146 & 170420 & 913315 & 374.094562655005 \\
15 & \textit{property} & 168 & 66 & 170473 & 913395 & 365.56857622936 \\
\midrule
\multicolumn{6}{l}{Additional words relating to \textvv{verticality} in Top 200} \\
\midrule
31 & \textit{fall} & 191 & 238 & 170450 & 913223 & 198.365368418608 \\
48 & \textit{jump} & 80 & 42 & 170561 & 913419 & 153.150735309968 \\
61 & \textit{up} & 548 & 1630 & 170093 & 911831 & 127.874429542027 \\
96 & \textit{low} & 111 & 167 & 170530 & 913294 & 93.6686639626192 \\
123 & \textit{slump} & 42 & 22 & 170599 & 913439 & 80.4873096048154 \\
156 & \textit{plunge} & 36 & 20 & 170605 & 913441 & 66.9831337818917 \\
174 & \textit{below} & 47 & 46 & 170594 & 913415 & 60.6500503451627 \\
186 & \textit{high} & 179 & 474 & 170462 & 912987 & 57.2925185336427 \\
188 & \textit{leap} & 33 & 21 & 170608 & 913440 & 57.0569288214017 \\
192 & \textit{surge} & 29 & 15 & 170612 & 913446 & 55.9161930382788 \\
200 & \textit{soar} & 27 & 13 & 170614 & 913448 & 53.8524788891024 \\
\midrule
\multicolumn{6}{l}{Other potential metaphors in Top 200} \\
\midrule
81 & \textit{cut} & 146 & 246 & 170495 & 913215 & 106.571616612729 \\
82 & \textit{inflation} & 51 & 23 & 170590 & 913438 & 104.758774339549 \\
84 & \textit{stake} & 80 & 78 & 170561 & 913383 & 103.560794211787 \\
89 & \textit{growth} & 63 & 47 & 170578 & 913414 & 98.9238607886036 \\
105 & \textit{chain} & 45 & 22 & 170596 & 913439 & 89.1257486535752 \\
133 & \textit{expansion} & 30 & 8 & 170611 & 913453 & 74.5673578463827 \\
138 & \textit{fixed} & 27 & 5 & 170614 & 913456 & 73.8215414858273 \\
157 & \textit{recovery} & 51 & 49 & 170590 & 913412 & 66.7957651616921 \\
181 & \textit{close} & 136 & 316 & 170505 & 913145 & 58.2854034590598 \\
183 & \textit{float} & 27 & 11 & 170614 & 913450 & 57.8863247734166 \\
184 & \textit{flotation} & 19 & 2 & 170622 & 913459 & 57.7380300163353 \\
193 & \textit{outlet} & 24 & 8 & 170617 & 913453 & 55.5026936443676 \\
\lspbottomrule
\multicolumn{6}{l}{\scriptsize{Supplementary Online Material: UH9B}} \\ %OSM
\end{tabular}}
\end{table}

As expected, most of the strongest keywords \is{keyword analysis} for the subcorpus are directly related to the domain of economics. Among the Top 15 (shown in the first part of the table), only two are not directly related to this domain: the proper name \textit{Middlesbrough} and the word \textit{rise}. The keyness \is{keyness} of the former is due to the fact that one of the files in the BNC Baby \is{BNC Baby} \textvv{commercial} subcorpus is from the Northern Echo, a regional newspaper \is{newspaper language} covering County Durham and Teesside -- Middlesbrough is the largest town in this region and is thus mentioned frequently, but it is not generally an important town, so it is hardly mentioned outside of this text. The keyness \is{keyness} of the latter is more interesting, as it is the kind of word we are looking for: its literal \is{literalness} meaning, \is{semantics} `motion from a lower to a higher position', would not be expected to be particularly central to the domain of economics. More interestingly, there are 11 additional words from the domain of `vertical motion' among the Top 200 keywords, \is{keyword analysis} making it the largest single semantic field other than `economy' (see second part of the table). There are only 12 other words whose literal meaning \is{semantics} is not directly related to economy, listed in the third part of the table, from source domains such as `increase in size' (\textit{inflation}, \textit{growth}, \textit{expansion}), `health' (\textit{recovery}) and `bodies of water' (\textit{float(ation)}, \textit{outlet}).

This suggests that `vertical motion' is a central source domain in the domain of economics, which we can now study by querying the respective keywords \is{keyword analysis} as well as other words from this domain (which we could get from a thesaurus). As an example, consider the concordance \is{concordance} of the lemma \is{lemma} \textit{rise} in \figref{fig:riseconc} (a random sample of 20 hits from the 221 hits in the \textvv{commerce} section in the BNC \is{BNC Baby} Baby).

\begin{figure}
\caption{A concordance of the wordform \textit{rise} in \textvv{commerce} texts (BNC Baby, sample)}
\label{fig:riseconc}
\hrulefill
\begin{fitverb}
 1 ts for the year to March . The price has [risen] by 119p since Caradon announced a possi
 2   was more than accounted for by a £1.6m [rise] in its Channel 4 subscription and by £
 3 aid that the top rate of income tax will [rise] to 50 p.c. on income , after allowances
 4 ay . The Pearl Investor Confidence Index [rose] by 1.2 p.c. last month -- its largest
 5  December , this figure was said to have [risen] to 17,000 a month at $25,000 ( £14,200
 6 by shoppers APPLICATIONS for credit have [risen] sharply in the wake of the Tory 's elec
 7  quarter of the year , despite a 15 p.c. [rise] in sales to $373m ( £214m ) . City : Q
 8 rman of United Biscuits , saw his salary [rise] from £233,000 to £425,000 last year .
 9 nce of the quality food retailers with a [rise] of almost 25% . Liz Dolan 's Surrey bui
10 s . They gave themselves an average 1992 [rise] of 5% , says the IoD , and a lot got ev
11  75p , while Fuller Smith , with profits [rising] to £3.75m ( £3.6m ) at midway , climb
12 LJ from SmithKline Beecham helped shares [rise] 23p to 248p . Merrydown is financing th
13 op up at Wessex WESSEX Water saw profits [rise] 11.3% to £44.3m in the first half help
14 s interims tomorrow , with a 10% profits [rise] to £101m expected . Northumbrian Water
15 es added 2p to 270p after a 10% dividend [rise] to 3.3p . Property divi slide GREAT Por
16 es on the dairy industry . Welcoming the [rise] in retail sales figures for October , C
17 isation of Petroleum Exporting Countries [rose] 100,000 barrels per day to 24.2 million
18   £13 barrier but Glaxo failed to hold a [rise] above £8 . To a degree some of the ris
19 he jobs queue since unemployment started [rising] in April 1990 . Employment Secretary Gi
20  parts of the country , with the biggest [rises] again in London and the SouthEast , fol
\end{fitverb}
\hrulefill
\end{figure}
% me: BNC Baby; A = [hw="rise"]:: match.text_genre="W:newsp.*commerce.*";
% me: randomize 9
% me: reduce A to 20

First, the concordance \is{concordance} corroborates our suspicion that \textit{rise} is not used literally \is{literalness} in the domain of economics. All 20 hits refer not to vertical motion, but to an increase in quantity, i.e., they instantiate the metaphor \is{figurative language}\is{metaphor} \textsc{more is up}. This is true both of verbal \is{verb} uses (in lines 1, 3, 4, 5, 6, 8, 11, 12, 13, 17, and 19) and to the nominal \is{noun} uses in the remaining lines. Second, the nouns in the surrounding context show where this metaphor is applied, namely overwhelmingly to genuinely economic concepts like \textit{prices}, \textit{tax rates}, \textit{salaries}, \textit{sales}, \textit{dividends}, \textit{profits}, \textit{shares}, etc.

The results of such a keyword \is{keyword analysis} analysis can now be used as a basis for all kinds of studies. For example, we may simply be interested in describing frequent metaphorical \is{figurative language}\is{metaphor} patterns in the data (say, in the context of teaching English for Special Purposes); some very noticeable examples are [\textit{rise}\textsubscript{N} \textit{in} NP] (lines 2, 7, 16) and [\textit{see} NP \textit{rise}\textsubscript{V}] (lines 8, 13), or [\textit{hold a} \textit{rise}\textsubscript{N}] (line 18).

Or we may be interested in the kind of research question discussed in Case Study \ref{sec:antonymymetaphor}, i.e. in whether literal \is{literalness} synonyms \is{synonymy} and antonyms \is{antonymy} of \textit{rise} are mapped isomorphically onto the domain of economics (the fact that \textit{jump}, \textit{surge} and \textit{soar} as well as \textit{fall}, \textit{slump} and \textit{plunge} are among the top 200 keywords \is{keyword analysis} certainly suggests they are.

Or we may be interested in the kind of research question discussed in Case Study \ref{sec:theimpactofmetaphoricalexpressions}, i.e. in what, if any, differences there are between the metaphorical pattern [\textit{rise in} NP] and its literal \is{literalness} equivalent [\textit{increase in} NP]. For example, a query of the BNC \is{BNC} for (\ref{ex:risingcost}) results in 84 hits for \textit{rising cost(s)} as opposed to 34 hits for \textit{increasing cost(s)} and 7 hits for \textit{rising profit(s)} as opposed to 12 hits for \textit{increasing profit(s)}:

\begin{exe}
\ex \texttt{[word="(rising|increasing)"\%c] [word="(cost|profit)s?"\%c]}
\label{ex:risingcost}
\end{exe}


It is left as an exercise for the reader to test this distribution \is{distribution!conditional} for significance using the $\chi^2$ \is{chi-square test} test or a similar test (but use a separate sheet of paper, as the margin of this page will be too small to contain your calculations). If more such differences can be found, this might suggest that the metaphor \is{figurative language}\is{metaphor} ``increase in quantity is upward motion'' is associated \is{association} more strongly with spending money than with making money.

This case study demonstrates that central metaphors \is{figurative language}\is{metaphor} for a given target domain can be identified by applying a keyword \is{keyword analysis} analysis to a specialized corpus of texts from that domain. The case study does not discuss a particular research question, but obviously, the method is useful in the context of many different research designs. \is{research design} Of course, it requires specialized corpora for the target domain under investigation. Such corpora are not available (and in some cases not imaginable) for all target domains, so the method works better for some target domains (such as \textvv{economics}) than for others (like \is{emotions} \textit{emotions}).

\subsubsection{Case study: Metaphoricity signals}
\label{sec:metaphoricitysignals}

Although metaphorical \is{figurative language}\is{metaphor} expressions are pervasive in all language varieties \is{language variety} and speakers do not generally seem to draw special attention to their occurrence, there is a wide range of devices that mark non\hyp{}literal language more or less explicitly (as in \textit{metaphorically\slash figuratively speaking}, \textit{picture} NP \textit{as} NP, \textit{so to speak\slash say}):

\begin{exe}
\ex
\begin{xlist}
\label{ex:metaphoricitysingnals}
\ex ...the princess held a gun to Charles's head, figuratively speaking... (BNC CBF)
\ex He pictures eternity as a filthy Russian bathhouse... (BNC A18)
\ex ...the only way they can deal with crime is to fight fire, so to speak, with fire. (BNC ABJ)
\end{xlist}
\end{exe}

\citet{wallington_metaphoricity_2003} investigate the extent to which these devices, which they call \textit{metaphoricity signals}, correlate \is{correlation} systematically with the occurrence of metaphorical \is{figurative language}\is{metaphor} expressions in language use. They find no strong correlation, but as they note, this may well be due to various aspects of their design. \is{research design} First, they adopt a very broad view of what constitutes a metaphoricity \is{metaphoricity signal}\is{figurative language}\is{metaphor} signal, including expressions like \textit{a kind\slash type\slash sort of}, \textit{not so much} NP \textit{as} NP and even prefixes \is{affix} like \textit{super}-, \textit{mini}-, etc. While some or all of these signals may have an affinity to certain kinds of non\hyp{}literal language, one would not really consider them to be \textit{metaphoricity} signals in the same way as those in (\ref{ex:metaphoricitysingnals}a--c). Second, they investigate a carefully annotated, \is{annotation} but very small corpus. Third, they do not distinguish between strongly conventionalized \is{conventionality} metaphors, \is{figurative language}\is{metaphor} which are found in almost every utterance and are thus unlikely to be explicitly signaled, and weakly conventionalized metaphors, which seems more likely to be signaled explicitly \textit{a priori}).

More restricted case studies are needed to determine whether the idea of metaphoricity \is{metaphoricity signal}\is{figurative language}\is{metaphor} signals \is{metaphoricity signal} is, in principle, plausible. Let us look at what is intuitively the clearest case of such a signal on Wallington et al.'s list: the sentence adverbials \is{adverb} \textit{metaphorically speaking} and \textit{figuratively speaking}. As a control, let us use the roughly equally frequent sentence adverbial \is{adverb} \textit{technically speaking}, which does not signal metaphoricity \is{figurative language}\is{metaphor} but which can, of course, co\hyp{}occur with (conventionalized) \is{conventionality} metaphors and which can thus serve as a baseline.

There are 22 cases of \textit{technically speaking} in the \is{BNC} BNC:

\begin{exe}
\ex
\begin{xlist}
\label{ex:technicallyspeaking}
\ex Do you mind if, \textit{technically speaking}, I resign rather than you sack me? (BNC A0F) %a
\ex \textit{Technically speaking} as long as nobody was hurt, no injuries, no damage to the other vehicle, this is not an accident. (BNC A5Y) %b
\ex \textit{$[$T$]$echnically speaking}, $[$...$]$ if you put her out into the road she would have no roof over her head and we should have to take her in. \footnotesize{(BNC AC7)} %c
\ex You will have to be the builder, \textit{technically speaking}. (BNC AM5) %d
\ex \textit{Technically speaking}, the only difference between VHS and VHS-C is in the length of the tape in the cassette $[$...$]$. (BNC CBP) %e
\ex $[$U$]$nlike financial controllers, directors can, \textit{technically speaking}, be held liable for negligence and consequently sued. (BNC CBU) %f
\ex \textit{Technically speaking} $[$...$]$, the unique formula penetrates the hair, enters the cortex and strengthens the hair bonds. (BNC CFS) %g
\ex As novelists, however, Orwell and Waugh evolve not towards each other but, \textit{technically speaking}, in opposite directions. (BNC CKN) %h
\ex Under the Net $[$...$]$ is \textit{technically speaking} a memoir-novel $[$...$]$, being composed as autobiography in the first person $[$...$]$.(BNC CKN) %i
\ex $[$T$]$he greater part of the works of art in the trade are \textit{technically speaking} `second-hand goods'. (BNC EBU) %j
\ex $[$T$]$he listener feels uncomfortably voyeuristic at times (yes, yes, I know that \textit{technically speaking}, a listener can't be voyeuristic. (BNC ED7) %k
\ex Adulterers. That's what they both were, \textit{technically speaking}. \footnotesize{(BNC F9C)} %l
\ex \textit{Technically speaking}, this $[$walk$]$ will certainly lead to the semi\hyp{}recumbent stone circle of Strichen in the district of Banff and Buchan $[$...$]$. (BNC G1Y) %m
\ex Richard was quite correct, as \textit{technically speaking} they were all in harbour, in addressing them by the names of their craft. (BNC H0R) %n
\ex `What's to stop you simply saying that my designs aren't suitable $[$...$]$' `Nothing, I suppose, \textit{technically speaking}.' (BNC H97) %o
\ex At least I was still a virgin, \textit{technically speaking}. (BNC HJC) %p
\ex The mike concealed in the head of the figure is only medium-quality, \textit{technically speaking} $[$...$]$. (BNC HTT) %q
\ex `Well, \textit{technically speaking} $[$...$]$ you are no longer in a position to provide him with employment.' (BNC HWN) %r
\ex Getting it right -- \textit{technically speaking} (BNC HX4) %s
\ex \textit{Technically speaking}, of course, she was off duty now and one of the night sisters had responsibility for the unit $[$...$]$. (BNC JYB) %t
\ex \textit{$[$T$]$echnically speaking} I suppose it is burnt but well done $[$...$]$ \footnotesize{(BNC KBP)} %u
\ex $[$Speaker A:$]$ And they class it as the south. Bloody ridiculous. $[$Speaker B:$]$ Well it is, \textit{technically speaking}, south of a ...(BNC KDD) %v
\end{xlist}
\end{exe}
% me: query: [word="technically"%c] [word="speaking"]

Taking a generous view, four of these are part of a clause that arguably contains a metaphor: \is{figurative language}\is{metaphor} (\ref{ex:technicallyspeaking}f) uses \textit{hold} as part of the phrase \textit{hold liable}, instantiating a metaphor like ``believing something about someone is holding them'' (cf. also \textit{hold s.o. responsible}\slash \textit{accountable}, \textit{hold in high esteem}); (\ref{ex:technicallyspeaking}h) uses the verb \is{verb} \textit{evolve} metaphorically to refer to a non\hyp{}evolutionary development and then uses the spatial expressions \textit{towards} and \textit{opposite direction} metaphorically \is{figurative language}\is{metaphor} to describe the quality of the development; (\ref{ex:technicallyspeaking}r) uses \textit{provide} as part of the phrase \textit{provide employment}, which instantiates a metaphor like ``causing someone to be in a state is transferring an object to them'' (cf. also \textit{provide s.o. with an opportunity\slash insight\slash power...}); (\ref{ex:technicallyspeaking}t) contains the spatial preposition \is{adposition} \textit{off} as part of the phrase \textit{off duty}, which could be said to instantiate the metaphor \is{figurative language}\is{metaphor} ``a situation is a location''. Note that all four expressions involve highly conventionalized \is{conventionality} metaphors, that would hardly be noticed as such by speakers.

There are 7 hits for the sentence adverbial \is{adverb} \textit{metaphorically speaking} in the \is{BNC} BNC:

\begin{exe}
\ex
\begin{xlist}
\label{ex:metaphoricallyspeaking}
\ex A convicted mass murderer has, for the second time, bloodied the nose, \textit{metaphorically speaking}, of Malcolm Rifkind, the Secretary of State for Scotland, by successfully pursuing a claim for damages. (BNC A3G) %a
\ex Yet, when I was seven years old, I should have thought him a very silly little boy indeed not to have understood about \textit{metaphorically speaking}, even if he had never heard of it, and it does seem that what he possessed in the way of scientific approach he lacked in common sense. (BNC AC7) %b
\ex Good caddies have good temperaments. Just watch Ian Wright getting a lambasting from Seve Ballesteros and see if Ian ever answers back, or, indeed, reacts in any way other than to quietly stand and take it on the chin, \textit{metaphorically speaking} of course. (BNC ASA) %c
\ex Family [are] a safe investment, but in love you can make a killing overnight. \textit{Metaphorically speaking}, I hasten to add. (BNC BMR) %d
\ex \textit{Metaphorically speaking}, the research front is a frozen moment in time $[$...$]$. (BNC HPN) %e
\ex Gregory put the boot in... \textit{metaphorically speaking}! (BNC K25) %f
\ex Mr Allenby are you ready to burst into song? \textit{Metaphorically speaking}. (BNC KM7) %g
\end{xlist}
\end{exe}
% me: query: [word="metaphorically"%c][word="speaking"%c]

In clear contrast to \textit{technically speaking}, six of these seven hits occur in clauses that contain a metaphor: \is{figurative language}\is{metaphor} \textit{bloody the nose of sb} in (\ref{ex:metaphoricallyspeaking}a) means `be successful in court against sb', instantiating the metaphor \textsc{legal fight is physical fight}; \textit{take it on the chin} in (\ref{ex:metaphoricallyspeaking}c) means `endure being criticized', instantiating the metaphor \textsc{argument is physical fight}; \textit{make a killing} in (\ref{ex:metaphoricallyspeaking}d) means `be financially successful', instantiating the metaphor \textsc{commercial activity is a hunt}; \textit{a frozen moment in time} in (\ref{ex:metaphoricallyspeaking}e) means `documentation of a particular state', instantiating the metaphor \textsc{time is a flowing body of water}; \textit{put the boot in} in (\ref{ex:metaphoricallyspeaking}f) means `treat sb cruelly', instantiating the metaphor \textsc{life (or sports) is physical fight}; \textit{burst into song} in (\ref{ex:metaphoricallyspeaking}g) means ``take one's turn speaking'', instantiating the metaphor \textsc{speaking is singing}. The only exception is (\ref{ex:metaphoricallyspeaking}b); this is a meta-linguistic use, indicating that someone did not understand that an utterance was meant metaphorically, \is{figurative language}\is{metaphor} rather than marking an utterance as metaphorical.

There are 13 hits for \textit{figuratively speaking} in the \is{BNC} BNC:

\begin{exe}
\ex
\begin{xlist}
\label{ex:figurativelyspeaking}
\ex The darts, the lumps of poison and the raw materials from which it is extracted all provide a challenge for others with a taste (\textit{figuratively speaking}) for excitement. (BNC AC9) %a
\ex Alternatively, you could select spiky, upright plants like agaves or yuccas to transport you across the world, \textit{figuratively speaking}, to the great deserts of North America. (BNC ACX) %b
\ex Palladium, statue of the goddess Pallas (Minerva) at Troy on which the city's safety was said to depend, hence, \textit{figuratively speaking}, the Bar seen as a bulwark of society. (BNC B0Y) %c
\ex \textit{Figuratively speaking}, who would not give their right arm to find such a love? (BNC B21) %d
\ex $[$I$]$t is surprising to me that this process was ever permitted on this site at all (being \textit{figuratively speaking} within arms length of the dwellings). (BNC B2D) %e
\ex \textit{Figuratively speaking}, we also make the law of value serve our aims. (BNC BMA) %f
\ex This schlocky international movie, photographed in eye-straining colour, cashing in (\textit{figuratively speaking}) on the craze for James Bond pictures $[$...$]$. (BNC C9U) %g
\ex He said: `I'm not sure if the princess held a gun to Charles's head, \textit{figuratively speaking}, but it seems if she wanted something said.' (BNC CBF) %h
\ex Let's pick someone completely at random, now we've had Tracey \textit{figuratively speaking}! (BNC F7U) %i
\ex `\textit{Figuratively speaking}', he declared, `in case of need, Soviet artillerymen can support the Cuban people with their rocket fire $[$...$]$'. (BNC G1R) %j
\ex $[$Customer talking to a clerk about a coat.$]$ `You told me it it was guaranteed waterproof.' `I didn't! I've never seen you before in my life!' $[$...$]$ `\textit{Figuratively speaking}, I meant. I bought it a few months ago and I was assured they were waterproof.' (BNC HGY) %k
\ex $[$T$]$he superego $[$...$]$ possesses the notable psychological property of being -- \textit{figuratively speaking} -- partly soluble in alcohol! (BNC HTP) %l
\ex Joan Daniels has now been appointed Honorary Treasurer of the Medau Society and we wish her the best of luck in balancing the books -- \textit{figuratively speaking}! (BNC KAE) %m
\end{xlist}
\end{exe}

Here, we would expect to find not just metaphors \is{figurative language}\is{metaphor} but also other kinds of non\hyp{}literal language -- which we do in almost all cases. The one exception is (\ref{ex:figurativelyspeaking}i), where the context (even enlarged beyond what is shown here) does not contain anything that could be a metaphor \is{figurative language}\is{metaphor} (it might be a metalinguistic use, indicating that the person called Tracey has been speaking figuratively). All other cases are clearly figurative: \is{figurative language} \textit{a taste for excitement} in (\ref{ex:figurativelyspeaking}a) means `an experience', instantiating the metaphor \textvv{experience is taste}; \textit{transport sb. across the world} in (\ref{ex:figurativelyspeaking}b) means `make sb think of a distant location', instantiating the metaphor \is{figurative language}\is{metaphor} \textvv{imaginary distance is physical distance}, \textit{bulwark of society} in (\ref{ex:figurativelyspeaking}c) means `defender of society', instantiating the metaphor \textvv{defense is a wall}, \textit{give one's right arm to do sth} in (\ref{ex:figurativelyspeaking}d) means `want sth very much', instantiating the metonymy \textvv{body part for personal value}; \textit{be within arm's length} in (\ref{ex:figurativelyspeaking}e) means `be in close proximity', instantiating the metonymy \textvv{arm's length for short distance}; \textit{make sth serve one's aims} in (\ref{ex:figurativelyspeaking}f) means `put sth to use in achieving sth', instantiating the metaphor \is{figurative language}\is{metaphor} \textvv{to be used is to serve}; \textit{cash in} in (\ref{ex:figurativelyspeaking}g) means `be successful', instantiating the metaphor \textvv{life is commercial transaction};\textit{hold gun to sb's head} in (\ref{ex:figurativelyspeaking}h) means `coerce sb to act', instantiating the metaphor \is{figurative language}\is{metaphor} \textvv{power is physical force}; (\ref{ex:figurativelyspeaking}j) is from a speech by the Soviet head of state Nikita Khrushchev in which he uses \textit{artillery} to refer metonymically about nuclear missiles; \textit{you told me} in (\ref{ex:figurativelyspeaking}k) means `your co\hyp{}employee told me', instantiating the metonymy \textvv{employee for company}; \textit{the superego is soluble in alcohol} in (\ref{ex:figurativelyspeaking}l) means `self\hyp{}control disappears when drunk ', instantiating the metaphor \is{figurative language}\is{metaphor} \textvv{character is a physical substance}; \textit{balance the books} in (\ref{ex:figurativelyspeaking}m) means `make sure debits and credits match', instantiating the metaphor \textvv{abstract entities are physical entities}.

We can now compare the literal \is{literalness} and metaphorical \is{figurative language}\is{metaphor} contexts in which the expressions \textit{technically speaking} and \textit{metaphorically\slash figuraltively speaking} occur. If the latter are a metaphoricity \is{metaphoricity signal}\is{figurative language}\is{metaphor} signal, they should occur significantly more frequently in metaphorical contexts than the former. \tabref{tab:literallytechnicallyspeaking} shows the tabulated results from the discussion above, subsuming metaphors \is{figurative language}\is{metaphor} and metonymies \is{figurative language}\is{metonymy} under \textvv{figurative}. The expected difference between contexts is clearly there, and statistically highly significant ($\chi^2 = 21.66, \df = 1 , p < 0.001, \phi = 0.7182$).\is{chi-square test}

\begin{table}
\caption{Literal and figurative utterances containing the sentence adverbials \textit{metaphorically\slash figuratively speaking} and \textit{technically speaking} (BNC)}
\label{tab:literallytechnicallyspeaking}
\begin{tabular}[t]{llccr}
\lsptoprule
 & & \multicolumn{2}{c}{\textvv{Sentence adverbial}} & \\
 & & \textvv{met./fig.} & \textvv{technically} & Total \\
\midrule
\textvv{\makecell[lt]{Utterance}}
	& \textvv{figurative}
		& \makecell[t]{\num{18}\\\small{(\num{10.48})}}
		& \makecell[t]{\num{4}\\\small{(\num{11.52})}}
		& \makecell[t]{\num{22}\\} \\
	& \textvv{$\neg$figurative}
		& \makecell[t]{\num{2}\\\small{(\num{9.52})}}
		& \makecell[t]{\num{18}\\\small{(\num{10.48})}}
		& \makecell[t]{\num{20}\\} \\
\midrule
	& Total
		& \makecell[t]{\num{20}}
		& \makecell[t]{\num{22}}
		& \makecell[t]{\num{42}} \\
\lspbottomrule
\end{tabular}
\end{table}
% me: chisq.test(matrix(c(18,2,4,18),ncol=2),corr=FALSE)

Of course, the question remains, \textit{why} some metaphors \is{figurative language}\is{metaphor} should be explicitly signaled while the majority is not. For example, we might suspect that metaphorical expressions are more likely to be explicitly signaled in contexts in which they might be interpreted literally. \is{literalness} This may be the case for \textit{put the boot in} in (\ref{ex:metaphoricallyspeaking}f), which occurs in a description of a rugby game where one could potentially misread it for a statement that someone was actually kicked. Alternatively (or additionally), a metaphor may be signaled explicitly if its specific phrasing is more likely to be used in literal \is{literalness} contexts. This may be the case with \textit{hold a gun to sb's head} in (\ref{ex:figurativelyspeaking}f): there are ten hits for this phrase in the BNC, \is{BNC} only one of which is metaphorical. Again, which of these hypotheses (if any of them) is correct would have to be studied more systematically.

This case study found a clear effect where the authors of the study it is based on did not. This demonstrates the need to formulate specific predictions concerning the behavior of specific linguistic items in such a way that they can be tested systematically and the results be evaluated statistically. The study also shows that the area of metaphoricity \is{metaphoricity signal}\is{figurative language}\is{metaphor} signals \is{metaphoricity signal} is worthy of further investigation.

\subsubsection{Case study: Metaphor and ideology}
\label{sec:metaphorandideology}

Regardless of whether metaphor is a rhetorical device (as has traditionally been assumed) or a cognitive \is{cognitive linguistics} device (as seems to be the majority view today), it is clear that it can serve an ideological \is{ideology} function, allowing authors suggest a particular perspective on a given topic. Thus, an analysis of the metaphors \is{figurative language}\is{metaphor} used in texts manifesting a particular ideology should allow us to uncover those perspectives.

For example, \citet{charteris-black_politicians_2005} investigates a corpus of ``right\hyp{}wing communication and media reporting'' on immigration, containing speeches, political manifestos and articles from the conservative newspapers \is{newspaper language} Daily Mail and Daily Telegraph. He finds, among other things, that the metaphor \is{figurative language}\is{metaphor} ``immigration is a flood'' is used heavily, arguing that this allows the right to portray immigration as a disaster that must be contained, citing examples like \textit{a flood of refugees}, \textit{the tide of immigration}, and \textit{the trickle of applicants has become a flood}.

Charteris\hyp{}Black's findings are intriguing, but since he does not compare the findings from his corpus of right\hyp{}wing materials to a neutral or a corresponding left\hyp{}wing corpus, it remains an open question whether the use of these metaphors \is{figurative language}\is{metaphor} indicates a specifically right\hyp{}wing perspective on immigration. Let us therefore replicate \is{replicability} his analysis more systematically. The BNC \is{BNC} contains 1~232~966 words from the Daily Telegraph (all files whose names begin with AH, AJ and AK), which will serve as our right\hyp{}wing corpus, and 918~159 words from the Guardian (all files whose names begin with A8, A9 or AA, except file AAY), which will serve as our corresponding left\hyp{}wing (or at least left\hyp{}leaning) corpus. Since Charteris\hyp{}Black's examples all involve reference to target\hyp{}domain items such as \textit{refugee} and \textit{immigration}, a metaphorical \is{figurative language}\is{metaphor} pattern analysis (cf. Section~\ref{sec:targetdomains} above) suggests itself. \figref{fig:liquidrefugee} shows all concordance \is{concordance} lines for the words \textit{migrant(s)}, \textit{immigrant(s)} and \textit{refugee(s)} containing metaphorical patterns instantiating the metaphor ``immigration is a mass of water''.

\begin{figure}
\caption{Selected \textsc{Liquid} metaphors with \textit{migrant(s)}, \textit{immigrant(s)},\textit{refugee(s)}}
\label{fig:liquidrefugee}
\hrulefill \\
\textsc{telegraph} (n=68)
\begin{fitverb}
Britain would be ` swamped with [[immigrants]] ' under a Labour Government
s country would be swamped with [[immigrants]] of every colour and race ,
support was due to the flood of [[migrants]] and would-be asylum seekers .
t increasing levels of economic [[migrants]] and asylum seekers entering B
nges to the constitution if the [[refugee]] influx was to be curbed . Herr
d open up Britain to a flood of [[immigrants]] and permit the rise of fasc
 the Gulf war and the influx of [[refugees]] from Afghanistan and Iraq . T
 for help to deal with flood of [[refugees]] By Philip Sherwell in Tuzla T
 Sherwell in Tuzla THE FLOOD of [[refugees]] fleeing the escalating confli
an militia groups . Most of the [[refugees]] flowing into Tuzla are escapi
gling to cope with the flood of [[refugees]] and have appealed to the inte
\end{fitverb}
\textsc{guardian} (n=136)
\begin{fitverb}
mirroring this year 's flood of [[refugees]] . Watched by a demonstration
litary security , migration and [[refugee]] flows on an vast scale . As We
 cope with the current surge of [[refugees]] , her Foreign Secretary , Mr
nd other aspects of large-scale [[immigrant]] absorption . The bureaucracy
he need to control an influx of [[immigrants]] . Rebel troops end siege of
control and manage the flows of [[migrants]] that wars , disasters , and ,
greed that the flow of economic [[migrants]] from Vietnam should be stoppe
e form of an influx of Romanian [[refugees]] . In one case in 1988 a Roman
control and manage the flows of [[migrants]] that wars , disasters , and ,
greed that the flow of economic [[migrants]] from Vietnam should be stoppe
\end{fitverb}
\hrulefill
\end{figure}

In terms of absolute frequencies, \is{frequency} there is no great difference between the two subcorpora (10 vs. 11), but the overall number of hits for the words in question differs drastically: there are 136 instances of these words in the Guardian subcorpus but only half as many (68) in the Telegraph subcorpus. This means that relatively speaking, in the domain of migration liquid metaphor \is{figurative language}\is{metaphor} are more frequent than expected \is{frequency!expected} in the Telegraph and less frequent than expected in the Guardian (see \tabref{tab:liquidrefugeefreq}), which suggests that such metaphors are indeed typical of right\hyp{}wing discourse. The difference just misses statistical significance, however, so a larger \is{corpus size} corpus would be required to corroborate the result ($\chi^2 = 3.822, \df = 1, p = 0.0506, \phi = 0.1369$).

\begin{table}
\caption{\textsc{Liquid} patterns with the words \textit{migrant(s)}, \textit{immigrant(s)}, \textit{refugee(s)}}
\label{tab:liquidrefugeefreq}
\begin{tabular}[t]{llccr}
\lsptoprule
 & & \multicolumn{2}{c}{\textvv{Newspaper}} & \\
 & & \textvv{guardian} & \textvv{$\neg$guardian} & Total \\
\midrule
\textvv{\makecell[lt]{Pattern}}
	& \textvv{liquid met.}
		& \makecell[t]{\num{10}\\\small{(\num{14.00})}}
		& \makecell[t]{\num{11}\\\small{(\num{7.00})}}
		& \makecell[t]{\num{21}\\} \\
	& \textvv{$\neg$liquid met.}
		& \makecell[t]{\num{126}\\\small{(\num{122.00})}}
		& \makecell[t]{\num{57}\\\small{(\num{61.00})}}
		& \makecell[t]{\num{183}\\} \\
\midrule
	& Table
		& \makecell[t]{\num{136}}
		& \makecell[t]{\num{68}}
		& \makecell[t]{\num{204}} \\
\lspbottomrule
\end{tabular}
\end{table}
% me: chisq.test(matrix(c(10,126,11,57),ncol=2),corr=FALSE)

This case study demonstrates that even general metaphors \is{figurative language}\is{metaphor} such as ``immigration is a mass of water'' may be associated \is{association} with particular political ideologies. \is{ideology} There is a large research literature on the role of metaphor \is{figurative language}\is{metaphor} in political discourse (see, for example, \citealt{koller_metaphor_2004}, \citealt{charteris-black_corpus_2004}, cf. also \citealt{musolff_study_2012}), although at least part of this literature is not as systematic and quantitative \is{quantitative research} as it should be, so this remains a promising area of research). The case study also demonstrates the need to include a control sample in corpus\hyp{}linguistic designs \is{research design} (in case that this still needed to be demonstrated at this point).

\subsection{Metonymy}
\label{sec:metonymy}

\subsubsection{Case study: Subjects of the verb \textit{bomb}}
\label{sec:subjectsoftheverbbomb}

This chapter was concerned with metaphor, \is{figurative language}\is{metaphor} but touched upon metonymy \is{figurative language}\is{metonymy} in Case Study \ref{sec:metaphoricitysignals}. While metaphor and metonymy are different phenomena, they are related by virtue of the fact that both of them are cases of non\hyp{}literal language, and they tend to be of interest to the same groups of researchers, so let us finish the chapter with a short case study of metonymy, if only to see to what extent the methods introduced above can be transferred to this phenomenon.

Following \citet[35]{lakoff_metaphors_1980}, metonymy \is{figurative language}\is{metonymy} is defined in a broad sense here as ``using one entity to refer to another that is related to it'' (this includes what is often called \textit{synecdoche}, see \citet{panther_distinguishing_1999} for critical discussion). Text book examples are the following from \citet[35, 39]{lakoff_metaphors_1980}:

\begin{exe}
\ex
\begin{xlist}
\label{ex:nixonsandwich}
\ex The ham sandwich is waiting for his check
\ex Nixon bombed Hanoi.
\end{xlist}
\end{exe}

In (\ref{ex:nixonsandwich}a), the metonym \is{figurative language}\is{metonymy} \textit{ham sandwich} stands for the target expression `the person who ordered the ham sandwich', in (\ref{ex:nixonsandwich}b) the metonym \textit{Nixon} stands for the target expression `the air\hyp{}force pilots controlled by Nixon' (at least at first glance).

Thus, metonymy \is{figurative language}\is{metonymy} differs from metaphor \is{figurative language}\is{metaphor} in that it does not mix vocabulary from two domains, which has consequences for a transfer of the methods introduced for the study of metaphor in Section~\ref{sec:studyingmetaphorincorpora}.

The source\hyp{}domain oriented approach can be transferred relatively straightforwardly -- we can query an item (or set of items) that we suspect may be used as metonyms \is{figurative language}\is{metonymy} then identify the actual metonymic uses. The main difficulty with this approach is choosing promising items for investigation. For example, the word \textit{sandwich} occurs almost 900 times in the BNC, \is{BNC} but unless I have overlooked one, it is not used as a metonym even once.

A straightforward analogue to the target\hyp{}domain oriented approach (i.e., metaphorical \is{figurative language}\is{metaphor} pattern analysis) \is{metaphorical pattern analysis} is more difficult to devise, as metonymies \is{figurative language}\is{metonymy} do not combine vocabulary from different semantic \is{semantics} domains. One possibility would be to search for verbs \is{verb} that we know or suspect to be used with metonymic subjects and\slash or objects. For example, a Google search for $\langle$ \texttt{"is waiting for (his|her|their|the) check"} $\rangle$ turns up about 20 unique hits; most of these have people as subjects and none of them have meals as subjects, but there are three cases that have \textit{table} as subject, as in (\ref{ex:tablecheck}):

\begin{exe}
\ex Table 12 is waiting for their check. (articles.baltimoresun.com)
\label{ex:tablecheck}
\end{exe}

Let us focus on the \textit{source\hyp{}domain} oriented perspective here, and let us use the famous example sentence in (\ref{ex:nixonsandwich}) as a starting point, loosely replicating \is{replicability} the study in \citet{stefanowitsch_metonymies_2015}. According to Lakoff and Johnson, this sentence instantiates what they call the ``controller for controlled'' metonymy, \is{figurative language}\is{metonymy} i.e. \textit{Nixon} would be a metonym for \textit{the air force pilots controlled by Nixon}.\footnote{Alternatively, as argued by \citet{stallard_two_1993}, it is the predicate rather than the subject that is used metonymically \is{figurative language}\is{metonymy} in this sentence, which would make this a metonym\hyp{}oriented case study.} Thus, searching a corpus for sequences of a noun \is{noun} followed by the verb \is{verb} \textit{bomb} should allow us to asses, for example, the importance of this metonymy \is{figurative language}\is{metonymy} in relation to other metonymies and literal \is{literalness} uses.

Querying the BNC \is{BNC} for $\langle$ \texttt{[pos=".*NN.*"] [lemma="bomb" \& pos=".*VB.*"]} $\rangle$ yields 31 hits referring to the dropping of bombs. Of these, only a single one has the ultimate decision maker as a subject (cf. \ref{ex:bombmetyonymy}a). Somewhat more frequent in subject position are countries or inhabitants of countries (5 cases) (cf. \ref{ex:bombmetyonymy}b, c). Even more frequently, the organization responsible for carrying out the bombing -- e.g. an air force, or part of an air force -- is chosen as the subject (9 cases) (cf. \ref{ex:bombmetyonymy}d,e). The most frequent case (14 hits) mentions the aircraft carrying the bombs in subject position, often accompanied by an adjective \is{adjective} referring to the country whose military operates the planes (cf \ref{ex:bombmetyonymy}f) or some other responsible group (cf. \ref{ex:bombmetyonymy}g). Finally, there are two cases where the bombs themselves occupy the subject position (cf. \ref{ex:bombmetyonymy}h).

\begin{exe}
\ex
\begin{xlist}
\label{ex:bombmetyonymy}
\ex Mussolini bombed and gassed the Abyssinians into subjection.
\ex On the day on which Iraq bombed Larak...
\ex Seven years after the Americans bombed Libya...
\ex $[$T$]$he school was blasted by an explosion, louder than anything heard there since the Luftwaffe bombed it in 1944.
\ex ...Germany, whose Condor Legion bombed the working classes in Guernica...
\ex ... on Jan. 24 French aircraft bombed Iraq for the first time ...
\ex ... Rebel jets bombed the Miraflores presidential palace ...
\ex ... Watching our bombs bomb your people ...
\end{xlist}
\end{exe}

Cases with pronouns \is{pronoun} in subject position -- resulting from the query $\langle$ \texttt{[pos=".*PNP.*"] [lemma="bomb" \& pos=".*VB.*"]} $\rangle$ -- have a similar distribution, \is{distribution!conditional} again, there is only one hit with a human \is{animacy} controller in subject position. All hits (whether with pronouns, common nouns \is{noun} or proper names), interestingly, have metonymic \is{figurative language}\is{metonymy} subjects -- i.e., not a single example has the bomber pilot in the subject position. This is unexpected, since literal \is{literalness} uses should be more frequent than figurative \is{figurative language} uses (it leads \citet{stefanowitsch_metonymies_2015} to reject an analysis of such sentences as metonymies altogether). On the other hand, there are cases that are plausibly analyzed as metonymies \is{figurative language}\is{metonymy} here, such as examples (\ref{ex:bombmetyonymy}d--e), which seem to instantiate a metonymy like \textvv{military unit for member of unit} (i.e. \textvv{whole for part}) and (\ref{ex:bombmetyonymy}f--h), which instantiate \textvv{plane for pilot} (i.e. \textvv{instrument for controller}).

More systematic study of such metonymies \is{figurative language}\is{metonymy} by target domain could uncover more such facts as well as contributing to a general picture of how important particular metonymies are in a particular language.

This case study sketches a potential target\hyp{}oriented approach to the corpus\hyp{}based study of metonymy, \is{figurative language}\is{metonymy} along with some general questions that we might investigate using it (most obviously, the question of how central a given metonymy is in the language under investigation). Again, metonymy is a vastly under\hyp{}researched area in corpus linguistics, so much work remains to be done.