\chapter{Morphology}
\label{ch:morphology}

We saw in Chapter \ref{ch:grammar} that the wordform-centeredness of most corpora and corpus-access tools requires a certain degree of ingenuity when studying structures larger than the word. It does not pose particular problems for corpus-based morphology, which studies structures smaller than the word. Corpus morphology is mostly concerned with the distribution of affixes, and retrieving all occurrences of an affix plausibly starts with the retrieval of all strings potentially containing this affix. We could retrieve all occurrences of \textit{-ness}, for example, with a query like $\langle$ \texttt{[word=".+ness(es)?"\%c]} $\rangle$. The recall of this query will be close to 100 percent, as all words containing the suffix \textit{-ness} end in the string \texttt{ness}, optionally followed by the string \texttt{es} in the case of plurals. Depending on the tokenization of the corpus, this query might miss cases where the word containing the suffix \textit{-ness} is the first part of a hyphenated compound, such as \textit{usefulness-rating} or \textit{consciousness-altering}; we could alter the query to something like $\langle$ \texttt{[word=".+ness(es)?(--.+=)?"\%c]} $\rangle$ if we believe that including these cases in our sample is crucial. The precision of such a query will not usually be 100 percent, as it will also retrieve words that accidentally happen to end with the string specified in our query -- in the case of \textit{-ness}, these would be words like \textit{witness}, \textit{governess} or place names like \textit{Inverness}. The degree of precision will depend on how unique the string in our query is for the affix in question; for \textit{-ness} and \textit{-ity} it is fairly high, as there are only a few words that share the same string accidentally (examples like those just mentioned for \textit{-ness} and words like \textit{city} and \textit{pity} for \textit{-ity}), for a suffix like \textit{-ess} (``female animate entity'') it is quite low, as a query like $\langle$ \texttt{[word=".+ess(es)?"\%c]} $\rangle$ will also retrieve all words with the suffixes \textit{-ness} and \textit{-less}, as well as many words whose stem ends in \textit{ess}, like \textit{process}, \textit{success}, \textit{press}, \textit{access}, \textit{address}, \textit{dress}, \textit{guess} and many more.

However, once we have extracted and -- if necessary -- manually cleaned up our data set, we are faced with a problem that does not present itself when studying lexis or grammar: the very fact that affixes do not occur independently but always as parts of words, some of which (like \textit{wordform-centeredness} in the first sentence of this chapter) have been created productively on the fly for a specific purpose, while others (like \textit{ingenuity} in the same sentence) are conventionalized lexical items that are listed in dictionaries, even though they are theoretically the result of attaching an affix to a known stem (like \textit{ingen-}, also found in \textit{ingenious} and, confusingly, its almost-antonym \textit{ingenuous}). We have to keep the difference between these two types of words in mind when constructing morphological research designs; since the two types are not always clearly distinguishable, this is more difficult than it sounds. Also, the fact that affixes always occur as parts of words has consequences for the way we can, and should, count them; in quantitative corpus-linguistics, this is a crucial point, so I will discuss it in quite some detail before we turn to our case studies.

\section{Quantifying morphological phenomena}
\label{sec:quantifyingmorphologicalphenomena}

\subsection{Counting morphemes: types, tokens and hapax legomena}
\label{sec:countingmorphemes}

Determining the frequency of a linguistic phenomenon in a corpus or under a particular condition seems a straightforward task: we simply count the number of instances of this phenomenon in the corpus or under that condition. However, this sounds straightforward (in fact, tautological) only because we have made tacit assumptions about what it means to be an ``instance'' of a particular phenomenon.

When we are interested in the frequency of occurrence of a particular word, it seems obvious that every occurrence of the word counts as an instance. In other words, if we know how often the word occurs in our data, we know how many instances there are in our data. For example, in order to determine the number of instances of the definite article in the BNC, we construct a query that will retrieve the string \texttt{the} in all combinations of upper and lower case letters, i.e. at least \textit{the}, \textit{The}, and \textit{THE}, but perhaps also \textit{tHe}, \textit{ThE}, \textit{THe}, \textit{tHE} and \textit{thE}, just to be sure). We then count the hits (since this string corresponds uniquely to the word \textit{the}, we don't even have to clean up the results manually). The query will yield \num{6041234} hits, so there are \num{6041234} instances of the word \textit{the} in the BNC.

When searching for grammatical structures (for example in Chapters \ref{ch:quantifyingresearch} and \ref{ch:significancetesting}), we have simply transferred this way of counting occurrences. For example, in order to determine the frequency of the \textit{s}-possessive in the BNC, we would define a reasonable query or set of queries (which, as discussed in various places in this book, can be tricky) and again simply count the hits. Let us assume that the query $\langle$ \texttt{[pos="(POS|DPS)"] [pos=".*AJ.*"]? [pos=".*NN.*"]} $\rangle$ is a reasonable approximation: it retrieves all instances of the possessive clitic (tagged \texttt{POS} in the BNC) or a possessive determiner (\texttt{DPS}), optionally followed by a word tagged as an adjective (\textit{AJ0}, \textit{AJC} or \textit{AJS}, even if it is part of an ambiguity tag), followed by a word tagged as a noun (\textit{NN0}, \textit{NN1} or \textit{NN2}, even if it is part of an ambiguity tag). This query will retrieve \num{1651908} hits, so it seems that there are \num{1651908} instances of the \textit{s}-possessive in the BNC.

However, there is a crucial difference between the two situations: in the case of the word \textit{the}, every instance is identical to all others (if we ignore upper and lower case). This is not the case for the \textit{s}-possessive. Of course, here, too, many instances are identical to other instances: there are exact repetitions of proper names, like \textit{King's Cross} (322 hits) or \textit{People's revolutionary party} (47), of (parts of) idiomatic expressions, like \textit{arm's length} (216) or \textit{heaven's sake} (187) or non-idiomatic but nevertheless fixed phrases like \textit{its present form} (107) or \textit{child's best interest} (26), and also of many free combinations of words that recur because they are simply communicatively useful in many situations, like \textit{her head} (5105), \textit{his younger brother} (112), \textit{people's lives} (224) and \textit{body's immune system} (29).

This means that there are two different ways to count occurrences of the \textit{s}-possessive. First, we could simply count all instances without paying any attention to whether they recur in identical form or not. When looking at occurrences of a linguistic item or structure in this way, they are referred to as \textit{tokens}, so \num{1651908} is the \textit{token frequency} of the possessive. Second, we could exclude repetitions and count only the number of instances that are different from each other, for example, we would count \textit{King's Cross} only the first time we encounter it, disregarding the other 321 occurrences. When looking at occurrences of linguistic items in this way, they are referred to as \textit{types}; the type frequency of the \textit{s}-possessive in the BNC is \num{268450} (again, ignoring upper and lower case). The type frequency of \textit{the}, of course, is 1. 

Let us look at one more example of the type/token distinction before we move on. Consider the following famous line from the theme song of the classic television series ``Mister Ed'':

\begin{exe}
\ex A horse is a horse, of course, of course...
\label{ex:horseisahorse}
\end{exe}

At the word level, it consists of nine tokens (if we ignore punctuation): \textit{a}, \textit{horse}, \textit{is}, \textit{a}, \textit{horse}, \textit{of}, \textit{course}, \textit{of}, and \textit{course}, but only of five types: \textit{a}, \textit{horse}, \textit{is},\textit{of}, and \textit{course}. Four of these types occur twice, one (\textit{is}) occurs only once. At the level of phrase structure, it consists of seven tokens: the NPs \textit{a horse}, \textit{a horse}, \textit{course}, and \textit{course}, the PPs \textit{of course} and \textit{of course}, and the VP \textit{is a horse}, but only of three types: VP, NP and PP.

In other words, we can count ``instances'' at the level of types or at the level of tokens. Which of the two levels is relevant in the context of a particular research design depends both on the kind of phenomenon we are counting and on our research question. When studying words, we will normally be interested in how often they are used under a particular condition, so it is their token frequency that is relevant to us; but we could imagine designs where we are mainly interested in whether a word occurs at all, in which case all that is relevant is whether its type frequency is one or zero. When studying grammatical structures, we will also mainly be interested in how frequently a particular grammatical structure is used under a certain condition, regardless of the words that fill this structure. Again, it is the token frequency that is relevant to us. However, note that we can (to some extent) ignore the specific words filling our structure only because we are assuming that the structure and the words are, in some meaningful sense, independent of each other; i.e., that the same words could have been used in a different structure (say, an \textit{of}-possessive instead of an \textit{s}-possessive) or that the same structure could have been used with different words (e.g. \textit{John's spouse} instead of \textit{his wife}). Recall that in our case studies in Chapter \ref{ch:significancetesting} we excluded all instances where this assumption does not hold (such as proper names and fixed expressions); since there is no (or very little) choice with these cases, including them, let alone counting repeated occurrences of them, would have added nothing (we did, of course, include repetitions of free combinations, of which there were four in our sample: \textit{his staff}, \textit{his mouth}, \textit{his work} and \textit{his head} occurred twice each).

Obviously, instances of morphemes (whether inflectional or derivational) can be counted in the same two ways. Take the following passage from William Shakespeare's play Julius Cesar:

\begin{exe}
\ex CINNA: ... Am I a married man, or a bachelor? Then, to answer every man directly and briefly, wisely and truly: wisely I say, I am a bachelor.
\label{ex:shakspearecinna}
\end{exe}

Let us count the occurrences of the adverbial suffix \textit{-ly}. There are five word tokens that contain this suffix (\textit{directly}, \textit{briefly}, \textit{wisely}, \textit{truly}, and \textit{wisely}), so its token frequency is five; however, there are only four types, since \textit{wisely} occurs twice, so its type frequency in this passage is four.

Again, whether type or token frequency is the more relevant or useful measure depends on the research design, but the issue is more complicated than in the case of words and grammatical structures. Let us begin to address this problem by looking at the diminutive affixes \textit{-icle} (as in \textit{cubicle}, \textit{icicle}) and \textit{mini-} (as in \textit{minivan}, \textit{mini-cassette}).

\textbf{a. Token frequency}. First, let us count the tokens of both affixes in the BNC. This is relatively easy in the case of \textit{-icle}, since the string \texttt{icle} is relatively unique to this morpheme (the name \textit{Pericles} is one of the few false hits that the query $\langle$ \texttt{[word=".+icles?"\%c]} $\rangle$ will retrieve. It is more difficult in the case of \textit{mini-}, since there are words like \textit{minimal}, \textit{minister}, \textit{ministry}, \textit{miniature} and others that start with the string \texttt{mini} but do not contain the prefix \textit{mini-}. Once we have cleaned up our concordances (available in the Supplementary Online Materials), we will find that \textit{-icle} has a token frequency of \num{20772} -- more than ten times that of \textit{mini-}, which occurs only \num{1702} times. We might thus be tempted to conclude that \textit{-icle} is much more important in the English language than \textit{mini-}, and that, if we are interested in English diminutives, we should focus on \textit{-icle}. However, this conclusion would be misleading, or at least premature, for reasons related to the problems introduced above.

Recall that affixes do not occur by themselves, but always as parts of words (this is what makes them affixes in the first place). This means that their token frequency can reflect situations that are both quantitatively and qualitatively very different. Specifically, a high token frequency of an affix may be due to the fact that it is used in a small number of very frequent words, or in a large number of very infrequent words (or something in between). The first case holds for \textit{-icle}: the three most frequent words it occurs in (\textit{article}, \textit{vehicle} and \textit{particle}) account for \num{19195} hits (i.e., 92.41 percent of all occurrences). In contrast, the three most frequent words with \textit{mini-} (\textit{mini-bus}, \textit{mini-bar} and \textit{mini-computer}) account for only 557 hits, i.e. 32.73 percent of all occurrences. To get to 92.4 percent, we would have to include the 253 most frequent words (roughly two thirds of all types).

In other words, the high token frequency of \textit{-icle} tells us nothing (or at least very little) about the importance of the affix, but rather about the importance of some of the words containing it. This is true regardless of whether we are looking at its token frequency in the corpus as a whole or under specific conditions; if its token frequency turned out to be higher, under one condition than under the other, this could point to the association between that condition and one or more of the words containing the affix, rather than between the condition and the affix itself.

For example, the token frequency of the suffix \textit{-icle} is higher in the BROWN corpus (269 tokens) than in the LOB corpus (225 tokens). However, as Table \ref{tab:iclewords} shows, this is exclusively due to the words \textit{article} and \textit{vehicle}: the former is more frequent than expected in British English and less so in American English, the latter is more frequent than expected in American English and less so in British English.

\begin{table}[!htbp]
\caption{Words containing \textit{-icle} in two corpora}
\label{tab:iclewords}
\resizebox*{!}{\textheight}{%
\begin{tabular}[t]{lccr}
\lsptoprule
 & \multicolumn{2}{c}{\textvv{Corpus}} & \\
\textvv{Word} & \textvv{lob} & \textvv{brown} & Total \\
\midrule
\textit{\makecell[tl]{article}}
	& \makecell[t]{\begin{tabular}[t]{lS[table-format=2.2]} \small{\textit{Obs.:}} & 126 \\ \small{\textit{Exp.:}} & 102.48 \\ \small{\textit{$\chi^2$:}} & 5.40 \end{tabular}}
	& \makecell[t]{\begin{tabular}[t]{lS[table-format=2.2]} \small{\textit{Obs.:}} & 99 \\ \small{\textit{Exp.:}} & 122.52 \\ \small{\textit{$\chi^2$:}} & 4.52 \end{tabular}}
	& 225 \\[1.1cm]
\textit{\makecell[tl]{particle}}
	& \makecell[t]{\begin{tabular}[t]{lS[table-format=2.2]} \small{\textit{Obs.:}} & 38 \\ \small{\textit{Exp.:}} & 46.46 \\ \small{\textit{$\chi^2$:}} & 1.54 \end{tabular}}
	& \makecell[t]{\begin{tabular}[t]{lS[table-format=2.2]} \small{\textit{Obs.:}} & 64 \\ \small{\textit{Exp.:}} & 55.54 \\ \small{\textit{$\chi^2$:}} & 1.29 \end{tabular}}
	& 102 \\[1.1cm]
\textit{\makecell[tl]{vehicle}}
	& \makecell[t]{\begin{tabular}[t]{lS[table-format=2.2]} \small{\textit{Obs.:}} & 39 \\ \small{\textit{Exp.:}} & 57.84 \\ \small{\textit{$\chi^2$:}} & 6.14 \end{tabular}}
	& \makecell[t]{\begin{tabular}[t]{lS[table-format=2.2]} \small{\textit{Obs.:}} & 88 \\ \small{\textit{Exp.:}} & 69.16 \\ \small{\textit{$\chi^2$:}} & 5.13 \end{tabular}}
	& 127 \\[1.1cm]
\textit{\makecell[tl]{chronicle}}
	& \makecell[t]{\begin{tabular}[t]{lS[table-format=2.2]} \small{\textit{Obs.:}} & 7 \\ \small{\textit{Exp.:}} & 6.38 \\ \small{\textit{$\chi^2$:}} & 0.06 \end{tabular}}
	& \makecell[t]{\begin{tabular}[t]{lS[table-format=2.2]} \small{\textit{Obs.:}} & 7 \\ \small{\textit{Exp.:}} & 7.62 \\ \small{\textit{$\chi^2$:}} & 0.05 \end{tabular}}
	& 14 \\[1.1cm]
\textit{\makecell[tl]{ventricle}}
	& \makecell[t]{\begin{tabular}[t]{lS[table-format=2.2]} \small{\textit{Obs.:}} & 8 \\ \small{\textit{Exp.:}} & 5.47 \\ \small{\textit{$\chi^2$:}} & 1.18 \end{tabular}}
	& \makecell[t]{\begin{tabular}[t]{lS[table-format=2.2]} \small{\textit{Obs.:}} & 4 \\ \small{\textit{Exp.:}} & 6.53 \\ \small{\textit{$\chi^2$:}} & 0.98 \end{tabular}}
	& 12 \\[1.1cm]
\textit{\makecell[tl]{auricle}}
	& \makecell[t]{\begin{tabular}[t]{lS[table-format=2.2]} \small{\textit{Obs.:}} & 5 \\ \small{\textit{Exp.:}} & 2.28 \\ \small{\textit{$\chi^2$:}} & 3.26 \end{tabular}}
	& \makecell[t]{\begin{tabular}[t]{lS[table-format=2.2]} \small{\textit{Obs.:}} & 0 \\ \small{\textit{Exp.:}} & 2.72 \\ \small{\textit{$\chi^2$:}} & 2.72 \end{tabular}}
	& 5 \\[1.1cm]
\textit{\makecell[tl]{fascicle}}
	& \makecell[t]{\begin{tabular}[t]{lS[table-format=2.2]} \small{\textit{Obs.:}} & 0 \\ \small{\textit{Exp.:}} & 1.37 \\ \small{\textit{$\chi^2$:}} & 1.37 \end{tabular}}
	& \makecell[t]{\begin{tabular}[t]{lS[table-format=2.2]} \small{\textit{Obs.:}} & 3 \\ \small{\textit{Exp.:}} & 1.63 \\ \small{\textit{$\chi^2$:}} & 1.14 \end{tabular}}
	& 3 \\[1.1cm]
\textit{\makecell[tl]{testicle}}
	& \makecell[t]{\begin{tabular}[t]{lS[table-format=2.2]} \small{\textit{Obs.:}} & 0 \\ \small{\textit{Exp.:}} & 0.91 \\ \small{\textit{$\chi^2$:}} & 0.91 \end{tabular}}
	& \makecell[t]{\begin{tabular}[t]{lS[table-format=2.2]} \small{\textit{Obs.:}} & 2 \\ \small{\textit{Exp.:}} & 1.09 \\ \small{\textit{$\chi^2$:}} & 0.76 \end{tabular}}
	& 2 \\[1.1cm]
\textit{\makecell[tl]{conventicle}}
	& \makecell[t]{\begin{tabular}[t]{lS[table-format=2.2]} \small{\textit{Obs.:}} & 1 \\ \small{\textit{Exp.:}} & 0.46 \\ \small{\textit{$\chi^2$:}} & 0.65 \end{tabular}}
	& \makecell[t]{\begin{tabular}[t]{lS[table-format=2.2]} \small{\textit{Obs.:}} & 0 \\ \small{\textit{Exp.:}} & 0.54 \\ \small{\textit{$\chi^2$:}} & 0.54 \end{tabular}}
	& 1 \\[1.1cm]
\textit{\makecell[tl]{cuticle}}
	& \makecell[t]{\begin{tabular}[t]{lS[table-format=2.2]} \small{\textit{Obs.:}} & 1 \\ \small{\textit{Exp.:}} & 0.46 \\ \small{\textit{$\chi^2$:}} & 0.65 \end{tabular}}
	& \makecell[t]{\begin{tabular}[t]{lS[table-format=2.2]} \small{\textit{Obs.:}} & 0 \\ \small{\textit{Exp.:}} & 0.54 \\ \small{\textit{$\chi^2$:}} & 0.54 \end{tabular}}
	& 1 \\[1.1cm]
\textit{\makecell[tl]{canticle}}
	& \makecell[t]{\begin{tabular}[t]{lS[table-format=2.2]} \small{\textit{Obs.:}} & 0 \\ \small{\textit{Exp.:}} & 0.46 \\ \small{\textit{$\chi^2$:}} & 0.46 \end{tabular}}
	& \makecell[t]{\begin{tabular}[t]{lS[table-format=2.2]} \small{\textit{Obs.:}} & 1 \\ \small{\textit{Exp.:}} & 0.54 \\ \small{\textit{$\chi^2$:}} & 0.38 \end{tabular}}
	& 1 \\[1.1cm]
\textit{\makecell[tl]{icicle}}
	& \makecell[t]{\begin{tabular}[t]{lS[table-format=2.2]} \small{\textit{Obs.:}} & 0 \\ \small{\textit{Exp.:}} & 0.46 \\ \small{\textit{$\chi^2$:}} & 0.46 \end{tabular}}
	& \makecell[t]{\begin{tabular}[t]{lS[table-format=2.2]} \small{\textit{Obs.:}} & 1 \\ \small{\textit{Exp.:}} & 0.54 \\ \small{\textit{$\chi^2$:}} & 0.38 \end{tabular}}
	& 1 \\[1.1cm]
\midrule
Total
	& \makecell[t]{225}
	& \makecell[t]{269}
	& \makecell[t]{494} \\
\lspbottomrule
\end{tabular}}
\end{table}
% query: [word=".+icles?"%c];
% false hits: LOB/BROWN: Pericles, BROWN: sticle
% lemmatized: a-particle, wave-particle

Even if all words containing a particular affix were more frequent under one condition (e.g. in one variety) and under another, this would tell us nothing about the affix itself: while such a difference in frequency could be due to the affix (as in the case of the adverbial suffix \textit{-ly}, which is disappearing from American English, but not from British English), it could also be due to the words containing the affix.

This is not to say that the token frequencies of affixes can never play a useful role; they may be of interest, for example, in cases of morphological alternation (i.e. two suffixes competing for the same stems, such as \textit{-ic} and \textit{-ical} in words like \textit{electric}/\textit{al}); here, we may be interested in the quantitative association between particular stems and one or the other of the affix variants, essentially giving us a collocation-type research design based on token frequencies. But for most research questions, designs based (exclusively) on the distribution of token frequencies under different conditions will give us meaningless results.

\textbf{b. Type frequency}. In contrast, the type frequency of an affix is a fairly direct reflection of the importance of the affix for the lexicon of a language: obviously an affix that occurs in many different words is more important than one that occurs only in a few words. Note that order to compare type frequencies, we have to correct for the size of the sample: all else being equal, a larger sample will contain more types than a smaller one simply because it offers more opportunities for different types to occur (a point we will return to in more detail in the next subsection). A simple way of doing this is to divide the number of types by the number of tokens; the resulting measure is referred to very transparently as the ``type/token ratio'' (or TTR):

\begin{exe}
\ex $\displaystyle{TTR = \frac{n \left( types \right) }{n \left( tokens \right)}}$ 
\label{ex:ttrformula}
\end{exe}

The TTR is the percentage of types in a sample are different from each other; or, put differently, it is the mean probability that we will encounter a new type if we go through the sample item by item.

For example, the affix \textit{-icle} occurs in just 31 different words in the BNC, so its TTR is $\nicefrac{31}{20772} = 0.0015$. In other words, 0.15 percent its tokens the BNC are different from each other, the vast remainder consists of repetitions. Put differently, if we went through the occurrences of \textit{-icle} in the BNC item by item, the probability that the next item instantiating this suffix will be a type we have not seen before is 0.15 percent, so we will encounter a new type on average once every 670 words. For \textit{mini-}, the type-token ratio is much higher: it occurs in 382 different words, so its TTR is $\nicefrac{382}{1702} = 0.2244$. In other words, almost a quarter of all occurrences of \textit{mini-} are different from each other. Put differently, if we went through our sample word by word, the probability that the next instance of \textit{mini-} is a a new type would be 22.4 percent, so we will encounter a new type about every four to five hits. The differences in their TTRs suggests that \textit{mini-}, in its own right is much more central in the English lexicon than \textit{-icle}, even though the latter has a much higher token frequency. Note that this is a statement only about the affixes; it does not mean that the \textit{words} containing \textit{mini-} are individually or collectively more important than those containing \textit{-icle} (on the contrary: words like \textit{vehicle}, \textit{article} and \textit{particle} are arguably much more important than words like \textit{minibus}, \textit{minicomputer} and \textit{minibar}).

Likewise, observing the type frequency (i.e., the TTR) of an affix under different conditions provides information about the relationship between these conditions and the affix itself, albeit one that is mediated by the lexicon: it tells us how important the suffix in question is for the subparts of the lexicon that are relevant under those conditions. For example, there are 7 types and 9 tokens for \textit{mini-} in the 1991 British FLOB corpus (two token each for \textit{mini-bus} and \textit{mini-series} and one each for \textit{mini-charter}, \textit{mini-disc}, \textit{mini-maestro}, \textit{mini-roll} and \textit{mini-submarine}), so the TTR is $\nicefrac{7}{9} = 0.7779$. In contrast, in the 1991 US-American FROWN corpus, there are 11 types and 12 tokens (two tokens for \textit{mini-jack}, and one token each for \textit{mini-cavalry}, \textit{mini-cooper}, \textit{mini-major}, \textit{mini-retrospective}, \textit{mini-version}, \textit{mini-boom}, \textit{mini-camp}, \textit{mini-grinder}, \textit{mini-series}, and \textit{mini-skirt}), so the TTR is $\nicefrac{11}{12} = 0.9167$. This suggests that the prefix \textit{mini-} was more important to the US-English lexicon than to the British English lexicon in the 1990s, although, of course, the samples and the difference between them are both rather small, so we would not want to draw that conclusion without consulting larger corpora and, possibly, testing for significance first (a point I will return to in the next subsection).

\textbf{c. Hapax legomena}. While type frequency is a useful (and, in my view, insufficiently valued) way of measuring the importance of affixes in general or under specific conditions, it has one drawback: it does not tell us whether the affix plays a productive role in a language at the time from which we take our samples (i.e., whether speakers at that time made use of it when coining new words). An affix may have a high TTR because it was productively used at the time of the sample, or because it was productively uses at some earlier period in the history of the language in question. In fact, an affix can have a high TTR even if it was never productively used, for example, because speakers at some point borrowed a large number of words containing it; this is the case for a number of Romance affixes in English, occurring in words borrowed from Norman French but never (or very rarely) used to coin new words. An example is the suffix
\textit{-ence}/\textit{-ance} occurring in many Latin and French loanwords (such as \textit{appearance}, \textit{difference}, \textit{existence}, \textit{influence}, \textit{nuisance}, \textit{providence}, \textit{resistance}, \textit{significance}, \textit{vigilance}, etc.), but only in a handful of words formed in English (e.g. \textit{abidance}, \textit{forbearance}, \textit{furtherance}, \textit{hinderance}, and \textit{riddance}).

In order to determine the productivity (and thus the current importance) of affixes at a particular point in time, Harald Baayen (cf. e.g. \citet{baayen_41._2009} for an overview) has suggested that we should focus on types that only occur once in the corpus, so-called \textit{hapax legomena} (Greek for `said once'). The assumption is that productive uses of an affix (or other linguistic rule) should result in one-off coinages (some of which may subsequently spread through the speech community while others will not).

Of course, not all hapax legomena are the result of productive rule-application: the words \textit{wordform-centeredness} and \textit{ingenuity} that I used in the first sentence of this chapter are both hapax legomena in this book (or would be, if I did not keep mentioning them). However, \textit{wordform-centeredness} is a word I coined productively and which is (at the time of writing) not documented anywhere outside of this book; in fact, I coined it for the sole reason that I knew I needed a good example of a hapax legomenon later). In contrast, \textit{ingenuity} has been part of the English language for more than four-hundred years (the OED first records it in 1598); it occurs only once in this book for the simple reason that I only needed it once (or pretended to need it, to have another example of a hapax legomenon). So a word may be a hapax legomenon because it is a productive coinage, or because it is infrequently-needed (in larger corpora, the category of hapaxes typically also contains misspelled or incorrectly tokenized words which will have to be cleaned up manualy -- for example, the token \textit{manualy} is a hapax legomenon in this book because I just misspelled it intentionally, but the word \textit{manually} occurs dozens of times in this book).

Baayen's idea is quite straightforwardly to use the phenomenon of ``hapax legomenon'' as an operationalization of the construct ``productive application of a rule'' in the hope that the correlation between the two notions (in a large enough corpus) will be substantial enough for this operationalization to make sense.\footnote{Note also that the productive application of a suffix does not necessarily result in a hapax legomenon: two or more speakers may arrive at the same coinage, or a single speaker may like their own coinage so much that they use it again; some researchers therefore suggest that we should also pay attention to ``dis legomena'' (words occurring twice) or even ``tris legomena'' (words occurring three times). We will stick with the mainstream here and use only hapax legomena.}

Like the number of types, the number of hapax legomena is dependent on sample size (although the relationship is not as straightforward as in the case of types, see next subsection); it is useful, therefore, to divide the number of hapax legomena by the number of tokens to correct for sample size:

\begin{exe}
\ex $\displaystyle{HTR = \frac{n \left( hapax legomena \right)}{n \left( tokens \right)}}$ 
\label{ex:htrformula}
\end{exe}

We will refer to this measure as the hapax-token ratio (or HTR) by analogy with the term \textit{type-token ratio}. Note, however, that in the literature this measure is referred to as \textit{P} for ``Productivity'' (following Baayen, who first suggested the measure); I depart from this nomenclature here to avoid confusion with \textit{p} for ``probability (of error)''.

Let us apply this measure to our two diminutive affixes. The suffix \textit{-icle} has just five hapax legomena in the BNC (\textit{auricle}, \textit{denticle}, \textit{pedicle}, \textit{pellicle} and \textit{tunicle}). This means that its HTR is $\nicefrac{5}{20772} = 0.0002$ -- 0.02 percent of its tokens are hapax legomena. In contrast, there are 247 hapax legomena for \textit{mini-} in the BNC (including, for example, \textit{mini-earthquake}, \textit{mini-daffodil}, \textit{mini-gasometer}, \textit{mini-cow} and \textit{mini-wurlitzer}). This means that its HTR is $\nicefrac{247}{1702} = 0.1451$ -- 14.5 percent of its tokens are hapax legomena). Thus, we can assume that \textit{mini-} is much more productive than \textit{-icle} -- or was, at the time that the BNC was assembled --, which presumably matches the intuition of most speakers of English.

\subsection{Statistical evaluation}
\label{sec:statisticalevaluation}

As pointed out in connection with the comparison of the TTRs for \textit{mini-} in the FLOB and the FROWN corpus, we would like to be able to test differences between two (or more) TTRs (and, of course, also two or more HTRs) for statistical significance. Theoretically, this could be done very easily. Take the TTR: if we interpret it as the probability of encountering a new type as we move through our samples, we are treating it like a nominal variable \textsc{Type}, with the values \textsc{new} and \textsc{seen before}. One appropriate statistical test for a distribution nominal values under different conditions is the chi-square test, which we are already more than familiar with. For example, if we wanted to test whether the TTRs of \textit{-icle} and \textit{mini-} in the BNC differ significantly, we might construct a table like that in Table \ref{tab:iclemini}.

\begin{table}[!htbp]
\caption{Type/token ratios of \textit{-icle} and mini- in the BNC}
\label{tab:iclemini}
\begin{tabular}[t]{llccr}
\lsptoprule
 & & \multicolumn{2}{c}{\textvv{Affix}} & \\
 & & \textvv{-icle} & \textvv{mini-} & Total \\
\midrule
\textvv{\makecell[lt]{Type}}
	& \textvv{new} 
		& \makecell[t]{\num{31}\\\small{(\num{381.72})}}
		& \makecell[t]{\num{382}\\\small{(\num{31.28})}}
		& \makecell[t]{\num{413}\\} \\
	& \textvv{seen before}
		& \makecell[t]{\num{20741}\\\small{(\num{20390.28})}}
		& \makecell[t]{\num{1320}\\\small{(\num{1670.72})}}
		& \makecell[t]{\num{22061}\\} \\
\midrule
	& Total
		& \makecell[t]{\num{20772}}
		& \makecell[t]{\num{1702}}
		& \makecell[t]{\num{22474}} \\
\lspbottomrule
\end{tabular}
\end{table}
% chisq.test(matrix(c(31,20741,382,1320),ncol=2),corr=FALSE)

The chi-square test would tell us that the difference is highly significant with a respectable effect size ($\chi^2$ = 4334.67, df = 1, p < 0.001, $\phi$ = 0.4392). For HTRs, we could follow a similar procedure: in this case we are dealing with a nominal variable \textsc{Type} with the variables \textsc{occurs only once} and \textsc{occurs more than once}, so we could construct the corresponding table and perform the chi-square test.

However, while the logic behind this procedure may seem plausible in theory both for HTRs and for TTRs, in practice, matters are much more complicated. The reason for this is that, as mentioned above, type-token ratios and hapax-token ratios are dependent on sample size.

In order to understand why and how this is the case and how to deal with it, let us leave the domain of morphology for a moment and look at the relationship between tokens and types or hapax legomena in texts. Consider the opening sentences of Jane Austen's novel \textit{Pride and Prejudice} (the novel is freely available from Project Gutenberg):

\begin{exe}
\ex It is a truth universally acknowledged, that a\textsubscript{2/-1} single man in possession of a\textsubscript{3} good fortune, must be in\textsubscript{2/-1} want of\textsubscript{2/-1} a\textsubscript{4} wife. However little known the feelings or views of\textsubscript{3} such a\textsubscript{5} man\textsubscript{2/-1} may be\textsubscript{2/-1} on his first entering a\textsubscript{6} neighbourhood, this truth\textsubscript{2/-1} is\textsubscript{2/-1} so well fixed in\textsubscript{3} the\textsubscript{2/-1} minds of\textsubscript{4} the surrounding families, that\textsubscript{2/-1} he is\textsubscript{3} considered the rightful property of\textsubscript{5} some one or\textsubscript{2/-1} other of\textsubscript{6} their daughters.
\label{ex:prideandprejudicesample}
\end{exe}

All words without a subscript are new types and hapax legomena at the point at which they appear in the text; if a word has a subscript, it means that it is a repetition of a previously mentioned word, the subscript is its token frequency at this point in the text. The first repetition of a word is additionally marked by a subscript reading -1, indicating that it ceases to be hapax legomenon at this point, decreasing the overall count of hapaxes by one.

As we move through the text word by word, initially all words are new types and hapaxes, so the type- and hapax-counts rise at the same rate as the token counts. However, it only takes eight tokens before we reach the first repetition (the word \textit{a}), so while the token frequency rises to 8, the type count remains constant at seven and the hapax count falls to six. Six words later, there is another occurrence of a, so type and hapax counts remain at 12 and 11 respectively as the token count rises to 14, and so on. In other words, while the numbers of types and hapaxes generally increases as the number of tokens in a sample increases, they do not increase at a steady rate. The more types have already occurred, the more types are there to be re-used (put simply, speakers will encounter fewer and fewer communicative situations that require a new type), which makes it less and less probable that new types (including new hapaxes) will occur. Figure \ref{fig:typehapaxausten} shows how type and hapax counts develop in the first 100 words of \textit{Pride and Prejudice} (on the right) and in the whole novel (on the left).

\begin{figure}
\caption{TTR and HTR in Jane Austen's \textit{Pride and Prejudice}}
\label{fig:typehapaxausten}
\begin{minipage}{.5\textwidth}
  \centering
  \includegraphics[width=\textwidth]{figures/prideandprejudiceonehundred}
\end{minipage}%
\begin{minipage}{.5\textwidth}
  \centering
  \includegraphics[width=\textwidth]{figures/prideandprejudiceall}
\end{minipage}
\end{figure}

As we can see by looking at the first 100 words, type and hapax counts fall below the token counts fairly quickly: after 20 tokens, the TTR is $\nicefrac{18}{20} = 0.9$ and the HTR is $\nicefrac{17}{20} = 0.85$, after 40 tokens the TTR is $\nicefrac{31}{40} = 0.775$ and the HTR is $\nicefrac{26}{40} = 0.65$, after 60 tokens the HTR is $\nicefrac{42}{60} = 0.7$ and the TTR is $\nicefrac{33}{60} = 0.55$, and so on (note also how the hapax-token ratio sometimes drops before it rises again, as words that were hapaxes up to a particular point in the text re-occur and cease to be counted as hapaxes. If we zoom out and look at the entire novel, we see that the growth in hapaxes slows considerably, to the extent that it has almost stopped by the time we reach the end of the novel. The growth in types also slows, although not as much as in the case of the hapaxes. In both cases this means that the ratios will continue to fall as the number of tokens increases.

Now imagine we wanted to use the TTR and the HTR as measures of Jane Austen's overall lexical productivity (referred to as ``lexical richness'' in computational stylistics and in second-language teaching): if we chose a small sample of her writing, the TTR and the HTR would be larger than if we chose a large sample, to the extent that the scores derived from the two samples would differ significantly. Table \ref{tab:austenttr} shows what would happen if we compared the TTR of the first chapter with the TTR of the entire rest of the novel.

\begin{table}[!htbp]
\caption{Type/token ratios in Pride and Prejudice}
\label{tab:austenttr}
\begin{tabular}[t]{llccr}
\lsptoprule
 & & \multicolumn{2}{c}{\textvv{Type}} & \\
 & & \textvv{new} & \textvv{$\neg$new} & Total \\
\midrule
\textvv{\makecell[lt]{Text Sample}}
	& \textvv{first chapter} 
		& \makecell[t]{\num{321}\\\small{(\num{47.29})}}
		& \makecell[t]{\num{6829}\\\small{(\num{7102.71})}}
		& \makecell[t]{\num{7150}\\} \\
	& \textvv{$\neg$first chapter}
		& \makecell[t]{\num{528}\\\small{(\num{801.71})}}
		& \makecell[t]{\num{120679}\\\small{(\num{120405.29})}}
		& \makecell[t]{\num{121207}\\} \\
\midrule
	& Total
		& \makecell[t]{\num{849}}
		& \makecell[t]{\num{127508}}
		& \makecell[t]{\num{128357}} \\
\lspbottomrule
\end{tabular}
\end{table}
% chisq.test(matrix(c(321,528,6829,120679),ncol=2),corr=FALSE)

The TTR for the first chapter is an impressive 0.3781, that for the rest of the novel is a measly 0.0566, and the difference is highly significant ($\chi^2$ = 1688.7, df = 1, p < 0.001, $\phi$ = 0.1147). But this is not because there is anything special about the first chapter; the TTR for the second chapter is 0.3910, that for the third is 0.3457, that for chapter 4 is 0.3943, and so on. The reason why the first chapter (or any chapter) looks as though it has a significantly higher TTR than the novel as a whole is simply because it the TTR will drop as the size of the text increases.

Therefore, comparing TTRs derived from samples of different sizes will always make the smaller sample look more productive. In other words, we cannot compare such TTRs, let alone evaluate the differences statistically -- the result will simply be meaningless. The same is true for HTRs, with the added problem that, under certain circumstances, it will decrease at some point as we keep increasing the sample size: at some point, all possible words will have been used, so unless new words are added to the language, the number of hapaxes will shrink again and finally drop to zero when all existing types have been used at least twice.

We will encounter the same problem when we compare the TTR or HTR of particular affixes or other linguistic phenomena, rather than that of a text. Consider Figures \ref{fig:izettrhtr}a and \ref{fig:izettrhtr}b, which shows the TTR and the HTR of the verb suffixes \textit{-ize} (occurring in words like \textit{realize}, \textit{maximize} or \textit{liquidize}) and \textit{-ify} (occurring in words like \textit{identify}, \textit{intensify} or \textit{liquify}).

\begin{figure}
\caption{TTRs and HTRs for \textit{-ise} and \textit{-ify} in the LOB corpus}
\label{fig:izettrhtr}
\begin{minipage}{.5\textwidth}
  \centering
  \includegraphics[width=\textwidth]{figures/lobiseifytypes}
\end{minipage}%
\begin{minipage}{.5\textwidth}
  \centering
  \includegraphics[width=\textwidth]{figures/lobiseifyhapaxes}
\end{minipage}
\end{figure}

As we can see, the TTR and HTR of both affixes behaves roughly like that of Jane Austen's vocabulary as a whole as we increase sample size: both of them grow fairly quickly at first before their growth slows down; the latter happens more quickly in the case of the HTR than in the case of the TTR, and, again, we observe that the HTR sometimes decreases as types that were hapaxes up to a particular point in the sample re-occur and cease to be hapaxes.

Taking into account the entire sample, the TTR for \textit{-ise} is $\nicefrac{105}{834} = 0.1259$ and that for \textit{-ify} is $\nicefrac{49}{356} = 0.1376$; it seems that \textit{-ize} is slightly more important to the lexicon of English than \textit{-ify}. A chi-square test suggests that that the difference is not significant (cf. Table \ref{tab:izeifyttr}; $\chi^2$ = 0.3053, df = 1, p > 0.05).

\begin{table}[!htbp]
\caption{Type/token ratios of \textit{-ise/-ize} and \textit{-ify} (LOB)}
\label{tab:izeifyttr}
\begin{tabular}[t]{llccr}
\lsptoprule
 & & \multicolumn{2}{c}{\textvv{Type}} & \\
 & & \textvv{new} & \textvv{seen before} & Total \\
\midrule
\textvv{\makecell[lt]{Affix}}
	& \textvv{-ise} 
		& \makecell[t]{\num{105}\\\small{(\num{107.93})}}
		& \makecell[t]{\num{729}\\\small{(\num{726.07})}}
		& \makecell[t]{\num{834}\\} \\
	& \textvv{-ify}
		& \makecell[t]{\num{49}\\\small{(\num{46.07})}}
		& \makecell[t]{\num{307}\\\small{(\num{309.93})}}
		& \makecell[t]{\num{356}\\} \\
\midrule
	& Total
		& \makecell[t]{\num{154}}
		& \makecell[t]{\num{1036}}
		& \makecell[t]{\num{1190}} \\
\lspbottomrule
\end{tabular}
\end{table}
% chisq.test(matrix(c(105,49,729,307),ncol=2),corr=FALSE)

Likewise, taking into account the entire sample, the HTR for \textit{-ize} is $\nicefrac{47}{834} = 0.0563$ and that for \textit{-ify} is $\nicefrac{17}{365} = 0.0477$; it seems that \textit{-ize} is slightly more productive than \textit{-ify}. However, again, the difference is not significant (cf. Table \ref{tab:izeifyhtr}; $\chi^2$ = 0.3628, df = 1, p > 0.05).

\begin{table}[!htbp]
\caption{Hapax/token ratios of \textit{-ise/-ize} and \textit{-ify} (LOB)}
\label{tab:izeifyhtr}
\begin{tabular}[t]{llccr}
\lsptoprule
 & & \multicolumn{2}{c}{\textvv{Type}} & \\
 & & \textvv{hapax} & \textvv{$\neg$hapax} & Total \\
\midrule
\textvv{\makecell[lt]{Affix}}
	& \textvv{-ise} 
		& \makecell[t]{\num{47}\\\small{(\num{44.85})}}
		& \makecell[t]{\num{787}\\\small{(\num{789.15})}}
		& \makecell[t]{\num{834}\\} \\
	& \textvv{-ify}
		& \makecell[t]{\num{17}\\\small{(\num{19.15})}}
		& \makecell[t]{\num{339}\\\small{(\num{336.85})}}
		& \makecell[t]{\num{356}\\} \\
\midrule
	& Total
		& \makecell[t]{\num{64}}
		& \makecell[t]{\num{1126}}
		& \makecell[t]{\num{1190}} \\
\lspbottomrule
\end{tabular}
\end{table}
% chisq.test(matrix(c(47,17,787,339),ncol=2),corr=FALSE)

However, note that \textit{-ify} has a token frequency that is less than half of that of \textit{-ize}, so the sample is much smaller: as in the example of lexical richness in \textit{Pride and Prejudice}, this means that the TTR and the HTR of this smaller sample are exaggerated and our comparisons in Table \ref{tab:izeifyttr} and Table \ref{tab:izeifyhtr} as well as the accompanying statistics are, in fact, completely meaningless.

The simplest way of solving the problem of different sample sizes is to create samples of equal size for the purposes of comparison. We simply take the size of the smaller of our two samples and draw a random sample of the same size from the larger of the two samples (if our data sets are large enough, it would be even better to draw random samples for both affixes). This means that we lose some data, but there is nothing we can do about this (note that we can still include the discarded data in a qualitative description of the affix in question).\footnote{In studies of lexical richness, a measure called \textit{Mean Segmental Type-Token Ratio} (MSTTR) is sometimes used \citep[cf.][]{johnson_program_1944}. This measure is derived by dividing the texts under investigation into segments of equal size (often segments of 100 words), determining the TTR for each segment, and then calculating an average TTR. This allows us to compare the TTR of texts of different sizes without discarding any data. However, this method is not applicable to the investigation of morphological productivity, as most samples of 100 words (or even 1000 or \num{10000} words) will typically not contain enough cases of a given morpheme to determine a meaningful TTR.}

Figures \ref{fig:izesamplettrhtr}a and \ref{fig:izesamplettrhtr}b show the growth rates of the TTR and the HTR of a sub-sample of 356 tokens of \textit{-ise} in comparison with the total sample of the same size for \textit{-ify} (the sample was derived by first deleting every second hit, then every seventh hit and finally every ninetieth hit, making sure that the remaining hits are spread throughout the corpus).

\begin{figure}
\caption{TTRs and HTRs for \textit{-ise} and \textit{-ify} in the LOB corpus}
\label{fig:izesamplettrhtr}
\begin{minipage}{.5\textwidth}
  \centering
  \includegraphics[width=\textwidth]{figures/lobsampleiseifytypes}
\end{minipage}
%
\begin{minipage}{.5\textwidth}
  \centering
  \includegraphics[width=\textwidth]{figures/lobsampleiseifyhapaxes}
\end{minipage}
\end{figure}

The TTR of \textit{-ize} based on the random sub-sample is $\nicefrac{78}{356} = 0.2191$, that of \textit{-ify} is still $\nicefrac{49}{356} = 0.1376$; the difference between the two suffixes is much clearer now, and a chi-square test shows that it is very significant, although the effect size is weak (cf. Table \ref{tab:izeifyttrsample}; $\chi^2$ = 8.06, df = 1, p < 0.01, $\phi$ = 0.1064).

\begin{table}[!htbp]
\caption{Type/token ratios of \textit{-ize}/\textit{-ise} (sample) and \textit{-ify} (LOB)}
\label{tab:izeifyttrsample}
\begin{tabular}[t]{llccr}
\lsptoprule
 & & \multicolumn{2}{c}{\textvv{Type}} & \\
 & & \textvv{new} & \textvv{seen before} & Total \\
\midrule
\textvv{\makecell[lt]{Affix}}
	& \textvv{-ise} 
		& \makecell[t]{\num{78}\\\small{(\num{63.50})}}
		& \makecell[t]{\num{278}\\\small{(\num{292.50})}}
		& \makecell[t]{\num{356}\\} \\
	& \textvv{-ify}
		& \makecell[t]{\num{49}\\\small{(\num{63.50})}}
		& \makecell[t]{\num{307}\\\small{(\num{292.50})}}
		& \makecell[t]{\num{356}\\} \\
\midrule
	& Total
		& \makecell[t]{\num{127}}
		& \makecell[t]{\num{585}}
		& \makecell[t]{\num{712}} \\
\lspbottomrule
\end{tabular}
\end{table}
% chisq.test(matrix(c(78,49,278,307),ncol=2),corr=FALSE)

Likewise, the HTR of \textit{-ize} based on our sub-sample is  $\nicefrac{41}{356} = 0.1152$, the HTR of \textit{-ify} remains $\nicefrac{17}{365} = 0.0477$. Again, the difference is much clearer, and it, too, is now very significant, again with a weak effect size (cf. Table \ref{tab:izeifyhtrsample}; $\chi^2$ = 10.81, df = 1, p < 0.01, $\phi$ = 0.1232).

\begin{table}[!htbp]
\caption{Hapax/token ratios of \textit{-ize/-ise} (sample) and \textit{-ify} (LOB)}
\label{tab:izeifyhtrsample}
\begin{tabular}[t]{llccr}
\lsptoprule
 & & \multicolumn{2}{c}{\textvv{Type}} & \\
 & & \textvv{hapax} & \textvv{$\neg$hapax} & Total \\
\midrule
\textvv{\makecell[lt]{Affix}}
	& \textvv{-ise} 
		& \makecell[t]{\num{41}\\\small{(\num{29.00})}}
		& \makecell[t]{\num{315}\\\small{(\num{327.00})}}
		& \makecell[t]{\num{356}\\} \\
	& \textvv{-ify}
		& \makecell[t]{\num{17}\\\small{(\num{29.00})}}
		& \makecell[t]{\num{339}\\\small{(\num{327.00})}}
		& \makecell[t]{\num{356}\\} \\
\midrule
	& Total
		& \makecell[t]{\num{58}}
		& \makecell[t]{\num{654}}
		& \makecell[t]{\num{712}} \\
\lspbottomrule
\end{tabular}
\end{table}
% chisq.test(matrix(c(41,17,315,339),ncol=2),corr=FALSE)

In the case of the HTR, decreasing the sample size is slightly more problematic than in the case of the TTR. The proportion of hapax legomena actually resulting from productive rule application becomes smaller as sample size decreases. Take example (\ref{ex:shakspearecinna}) from Shakespeare's Julius Caesar above: the words \textit{directly}, \textit{briefly} and \textit{truly} are all hapaxes in the passage cited, but they are clearly not the result of a productively applied rule-application (all of them have their own entries in the OALD, for example). As we increase the sample, they cease to be hapaxes (\textit{directly} occurs 9 times in the entire play, \textit{briefly} occur 4 times and \textit{truly} 8 times). This means that while we must draw random samples of equal size in order to compare HTRs, we should make sure that these samples are as large as possible.

\section{Case studies}
\label{sec:moprphologycasestudies}

\subsection{Morphemes and stems}
\label{sec:morphemesandstems}

One general question in (derivational) morphology concerns the category of the stem which an affix may be attached to. This is obviously a descriptive issue that can be investigated on the basis of corpora very straightforwardly simply by identifying all types containing the affix in question and describing their internal structure. In the case of affixes with a low productivity, this will typically add little insight over studies based on dictionaries, but for productive affixes, a corpus-analysis will yield more detailed and comprehensive results since corpora will contain spontaneously produced or at least recently created items not (yet) found in dictionaries. Such newly-created words will often offer particularly clear insights into constraints that an affix places on its stems (and obviously, corpus-based approaches are without an alternative in diachronic studies and they yield particularly interesting results when used to study changes in the quality or degree of productivity, cf. for example \citet{dalton-puffer_french_1996}).

In the study of constraints placed by derivational affixes on the stems that they combine with, the combinability of derivational morphemes (in an absolute sense or in terms of preferences) is of particular interest. Again, corpus linguistics is a uniquely useful tool to investigate this.

Finally, there are cases where two derivational morphemes are in direct competition because they are functionally roughly equivalent (e.g. \textit{-ness} and \textit{-ity}, both of which form abstract nouns from typically adjectival bases, \textit{-ize} and \textit{-ify}, which form process verbs from nominal and adjectival bases or \textit{-ic} and \textit{-ical}, which form adjectives from typically nominal bases). Here, too, corpus linguistics provides useful tools, for example to determine whether the choice between affixes is influenced by syntactic, semantic or phonological properties of stems.

\subsubsection{Case study: Phonological constraints of \textit{-ify}}
\label{sec:phonologicalconstraintsofify}

As part of a larger argument that \textit{-ize} and \textit{-ify} should be considered phonologically conditioned allomorphs, \citet{plag_morphological_1999} investigates the phonological constraints that \textit{-ify} places on its stems. First, he summarizes the properties of stems in established words with \textit{-ify} as observed in the literature. Second, he checks these observations against a sample of twenty-three recent (20th-century) coinages from a corpus of neologisms to ensure that the constraints also apply to productive uses of the affix. This is a reasonable question. The affix was first borrowed into English as part of a large number of French loanwords beginning in the late 13th century; two thirds of all non-hapax types and 19 of the twenty most frequent types found in the BNC are older than the 19th century. Thus, it is possible that the constraints observed in the literature are historical remnants not relevant to new coinages.

The most obvious constraint is that the syllable directly preceding \textit{-ify} must carry the main stress of the word. This has a number of consequences, of which we will focus on two: First, monosyllabic stems (as in \textit{falsify}) are preferred, since they always meet this criterion. Second, if a polysyllabic stem ends in an unstressed syllable, the stress must be shifted to that syllable (as in \textit{perSONify} from \textit{PERson});\footnote{This is a simplification: if an unstressed final syllable ends in a vowel, it is simply deleted (as in \textit{SIMple} -- \textit{SIMplify}); stress-shift only occurs with unstressed closed syllables or sequence of two unstressed syllables (\textit{SYLlable} -- \textit{sylLAbify}). Occasionally stem-final consonants are deleted (as in \textit{liquid} -- \textit{liquify}); cf. \citet{plag_morphological_1999} for a more detailed discussion.} since this reduces the transparency of the stem, there should be a preference for those polysyllabic stems which already have the stress on the final syllable.

Plag simply checks his neologisms against the literature, but we will evaluate the claims from the literature quantitatively. Our main hypotheses will be that neologisms with \textit{-ify} do not differ from established types with respect to the fact that the directly preceding the suffix must carry primary stress, with the consequences that (i) they prefer monosyllabic stems, and (ii) if the stem is polysyllabic, they prefer stems that already have the primary stress on the last syllable. Our independent variable is therefore \textsc{Lexical Status} with the values \textsc{established word} vs. \textsc{neologism} (which will be operationalized presently). Our dependent variables are \textsc{Syllabicity} with the values \textsc{monosyllabic} and \textsc{polysyllabic}, and \textsc{Stress Shift} with the values \textsc{required} vs. \textsc{not required} (both of which should be self-explanatory).

Our design compares two predefined groups of types with respect to the distribution that particular properties have in these groups; this means that we do not need to calculate TTRs or HTRs, but that we need operational definitions of the values \textsc{established word} and \textsc{neologism}). Following Plag, let us define \textsc{neologism} as ``coined in the 20th century'', but let us use a large historical dictionary (the Oxford English Dictionary, 3rd edition) and a large corpus (the BNC) in order to identify words matching this definition; this will give us the opportunity to evaluate the idea that hapax legomena are a good way of operationalizing productivity.

Excluding cases with prefixed stems, the OED contains 456 entries or sub-entries for verbs with \textit{-ify}, 31 of which are first documented in the 20th century. Of the latter, 21 do not occur in the BNC at all, and 10 do occur in the BNC, but are not hapaxes (see Table \ref{tab:ifyneologisms} below). The BNC contains 30 hapaxes, of which 13 are spelling errors and 7 are first documented in the OED before the 20th century (\textit{carbonify}, \textit{churchify}, \textit{hornify}, \textit{preachify}, \textit{saponify}, \textit{solemnify}, \textit{townify}). This leaves 10 hapaxes that are plausibly regarded as neologisms, none of which are listed in the OED (again, see Table \ref{tab:ifyneologisms}). In addition, there are four types in the BNC that are not hapax legomena, but that are not listed in the OED; careful cross-checks show that these are also neologisms. Combining all sources, this gives us 45 neologisms.

\begin{table}[!htbp]
\caption{Twentieth century neologisms with \textit{-ify}}
\label{tab:ifyneologisms}
\begin{tabular}[t]{ll}
\lsptoprule
\multicolumn{2}{l}{First documented in the OED in the 20th century} \\
(i) & also occur in the BNC, but not as hapaxes: \\
 & \makecell[tl]{\textit{bourgeoisify}, \textit{esterify}, \textit{gentrify}, \textit{karstify}, \textit{massify}, \textit{Nazify}, \textit{syllabify}, \\ \textit{vinify}, \textit{yuppify}, \textit{zombify}} \\
(ii) & do not occur in the BNC: \\
 & \makecell[tl]{\textit{ammonify}, \textit{aridify}, \textit{electronify}, \textit{glassify}, \textit{humify}, \textit{iconify}, \textit{jazzify}, \\ \textit{mattify}, \textit{metrify}, \textit{mucify}, \textit{nannify}, \textit{passivify}, \textit{plastify}, \textit{probabilify}, \\\textit{Prussify}, \textit{rancidify}, \textit{sinify}, \textit{trendify}, \textit{trustify}, \textit{tubify}, \textit{youthify}} \\
\multicolumn{2}{l}{Hapax Legomena in the BNC} \\
(iii) & not listed in the OED at all: \\
 & \makecell[tl]{\textit{faintify}, \textit{fuzzify}, \textit{lewisify}, \textit{rawify}, \textit{rockify}, \textit{sickify}, \textit{sonify}, \textit{validify}, \\ \textit{yankify}, \textit{yukkify}} \\
\multicolumn{2}{l}{Non-Hapax Legomena in the BNC that are not listed in the OED} \\
(iv) & \makecell[tl]{\textit{commodify}, \textit{desertify}, \textit{extensify}, \textit{geriatrify}} \\
\lspbottomrule
\end{tabular}
\end{table}

Before we turn to the definition and sampling of established types, let us determine the precision and recall of the operational definition of neologism as ``hapax legomenon'' in the BNC, using the formulas introduced in Chapter \ref{ch:retrievalannotation}. Precision is defined as the number of true positives (items that were found and that actually are what they are supposed to be) divided by the number of all positives (all items found); 10 of the 30 hapaxes in the BNC are actually neologisms, so the precision is 10/30 = 0.3333. Recall is defined as the number of true positives divided by the number of true positives and false negatives (i.e., all items that should have been found); 10 of the 45 neologisms were actually found by using the hapax definition, so the recall is 10/45 = 0.2222. In other words, neither precision or recall of the method are very good, at least for moderately productive affixes like \textit{-ify} (the method will presumably give better results with highly productive affixes). Let us also determine the recall of neologisms from the OED (using the definition ``first documented in the 20th century according to the OED''): the OED lists 31 of the 45 neologisms, so the recall is 31/45 = 0.6889; this is much better than the recall of the corpus-based hapax definition, but it also shows that if we combine corpus data and dictionary data, we can increase coverage substantially even for moderately productive affixes.

Let us now turn to the definition of \textsc{established types}. Given our definition of \textsc{neologisms}, established types would first have to be documented before the 20th century, so we could use the 420 types in the OED that meet this criterion (again, excluding prefixed forms). However, these 420 types contain many very rare or even obsolete forms, like \textit{duplify} ``to make double'', \textit{eaglify} ``to make into an eagle'' or \textit{naucify} ``to hold in low esteem''. Clearly, these are not ``established'' in any meaningful sense, so let us add the requirement that a type must occur in the BNC at least twice to count as established. Let us further limit the category to verbs first documented before the 19th century, in order to leave a clear diachronic gap between the established types and the productive types. This leaves the words in Table \ref{tab:ifycontrol}.\footnote{Interestingly, leaving out words coined in the 19th century does not make much of a difference: although the 19th century saw a large number of coinages (with 138 new types it was the most productive century in the history of the suffix), few of these are frequent enough today to occur in the BNC; if anything, we should actually extend our definition of neologisms to include the 19th century.}

\begin{table}[!htbp]
\caption{Control sample of established types with the suffix \textit{-ify}.}
\label{tab:ifycontrol}
\begin{tabular}[t]{l}
\lsptoprule
\makecell[tl]{
\textit{acidify}, \textit{amplify}, \textit{beatify}, \textit{beautify}, \textit{certify}, \textit{clarify}, \textit{classify}, \\ \textit{crucify}, \textit{damnify}, \textit{deify}, \textit{dignify}, \textit{diversify}, \textit{edify}, \textit{electrify}, \\ \textit{exemplify}, \textit{falsify}, \textit{fortify}, \textit{Frenchify}, \textit{fructify}, \textit{glorify}, \textit{gratify}, \\ \textit{identify}, \textit{indemnify}, \textit{justify}, \textit{liquify}/\textit{liquefy}, \textit{magnify}, \textit{modify}, \\ \textit{mollify}, \textit{mortify}, \textit{mummify}, \textit{notify}, \textit{nullify}, \textit{ossify}, \textit{pacify}, \\ \textit{personify}, \textit{petrify}, \textit{prettify}, \textit{purify}, \textit{qualify}, \textit{quantify}, \textit{ramify}, \\ \textit{rarify}, \textit{ratify}, \textit{rectify}, \textit{sacrify}, \textit{sanctify}, \textit{satisfy}, \textit{scarify}, \\ \textit{signify}, \textit{simplify}, \textit{solidify}, \textit{specify}, \textit{stratify}, \textit{stultify}, \textit{terrify}, \\ \textit{testify}, \textit{transmogrify}, \textit{typify}, \textit{uglify}, \textit{unify}, \textit{verify}, \textit{versify}, \\ \textit{vilify}, \textit{vitrify}, \textit{vivify}} \\ 
\lspbottomrule
\end{tabular}
\end{table}

Let us now evaluate the hypotheses. Table \ref{tab:ifysyllabicity} shows the type frequencies for monosyllabic and polysyllabic stems in the two samples. In both cases, there is a preference for monosyllabic stems (as expected), but interestingly, this preference is less strong among the neologisms than among the established types and this difference is very significant ($\chi^2 = 7.37, df = 1, p < 0.01 , \phi = 0.2577$).

\begin{table}[!htbp]
\caption{Monosyllabic and bisyllabic stems with \textit{-ify}}
\label{tab:ifysyllabicity}
\begin{tabular}[t]{llccr}
\lsptoprule
 & & \multicolumn{2}{c}{\textvv{Number of Syllables}} & \\
 & & \textvv{monosyllabic} & \textvv{polysyllabic} & Total \\
\midrule
\textvv{\makecell[lt]{Status}}
	& \textvv{established} 
		& \makecell[t]{\num{57}\\\small{(\num{51.14})}}
		& \makecell[t]{\num{9}\\\small{(\num{14.86})}}
		& \makecell[t]{\num{66}\\} \\
	& \textvv{neologism}
		& \makecell[t]{\num{29}\\\small{(\num{34.86})}}
		& \makecell[t]{\num{16}\\\small{(\num{10.14})}}
		& \makecell[t]{\num{45}\\} \\
\midrule
	& Total
		& \makecell[t]{\num{86}}
		& \makecell[t]{\num{25}}
		& \makecell[t]{\num{111}} \\
\lspbottomrule
\end{tabular}
\end{table}
% chisq.test(matrix(c(57,29,9,16),ncol=2),corr=FALSE)

The fact that there is a significantly higher number of neologisms with polysyllabic stems than expected on the basis of established types, the second hypothesis becomes more interesting: does this higher number of polysyllabic stems correspond with a greater willingness to apply it to stems that then have to undergo stress shift (which would be contrary to our hypothesis, which assumes that there will be no difference between established types and neologisms)?

Table \ref{tab:ifystressshift} shows the relevant data: it seems that there might indeed be such a greater willingness, as the number of neologisms with polysyllabic stems requiring stress shift is higher than expected; however, the difference is not statistically significant ($\chi^2$ = 1.96, df = 1, p > 0.05, $\phi$ = 0.28) (strictly speaking, we cannot use the chi-square test here, since half of the expected frequencies are below 5, but Fisher's exact test confirms that the difference is not significant).

\begin{table}[!htbp]
\caption{Stress-shift with polysyllabic stems with \textit{-ify}}
\label{tab:ifystressshift}
\begin{tabular}[t]{llccr}
\lsptoprule
 & & \multicolumn{2}{c}{\textvv{Shift}} & \\
 & & \textvv{not required} & \textvv{required} & Total \\
\midrule
\textvv{\makecell[lt]{Status}}
	& \textvv{established} 
		& \makecell[t]{\num{3}\\\small{(\num{4.68})}}
		& \makecell[t]{\num{6}\\\small{(\num{4.32})}}
		& \makecell[t]{\num{9}\\} \\
	& \textvv{neologism}
		& \makecell[t]{\num{10}\\\small{(\num{8.32})}}
		& \makecell[t]{\num{6}\\\small{(\num{7.68})}}
		& \makecell[t]{\num{16}\\} \\
\midrule
	& Total
		& \makecell[t]{\num{13}}
		& \makecell[t]{\num{12}}
		& \makecell[t]{\num{25}} \\
\lspbottomrule
\end{tabular}
\end{table}
% chisq.test(matrix(c(3,10,6,6),ncol=2),corr=FALSE)

This case study demonstrates some of the problems and advantages of using corpora to identify neologisms in addition to existing dictionaries. It also constitutes an example of a purely type-based research design; note, again, that such a design is possible here because we are not interested in the type frequency of a particular affix under different conditions (in which case we would have to calculate a TTR to adjust for different sample sizes), but in the distribution of the variables \textsc{Syllable Length} and \textsc{Stress Shift} in two qualitatively different categories of types. Finally, note that the study comes to different conclusions than the impressionistic analysis in \citet{plag_morphological_1999} so it demonstrates the advantages of strictly quantified designs.

\subsubsection{Case study: Semantic differences between \textit{-ic} and \textit{-ical}}
\label{sec:semanticdifferencesbetweenicandical}

Affixes, like words, can be related to other affixes by lexical relations like synonymy, antonymy etc. In the case of (roughly) synonymous affixes, an obvious research question is what determines the choice between them -- for example, whether there are more fine-grained semantic differences that are not immediately apparent.

One way of approaching this question is to focus on stems that occur with both affixes (such as \textit{liqui(d)} in \textit{liquidize} and \textit{liquify}/\textit{liquefy}, \textit{scarce} in \textit{scarceness} and \textit{scarcity} or \textit{electr-} in \textit{electric} and \textit{electrical}) and to investigate the semantic contexts in which they occur -- for example, by categorizing their collocates, analogous to the way \citet{taylor_near_2003} categorizes collocates of \textit{high} and \textit{tall} (cf. Chapter \ref{ch:collocation}, Section \ref{sec:nearsynonyms}).

A good example of this approach is found in \citet{kaunisto_electric/electrical_1999}, who investigates the pairs \textit{electric}/\textit{electrical} and \textit{classic}/\textit{classical} on the basis of the British Newspaper \textit{Daily Telegraph}. Since his corpus is not accessible, let us use the LOB corpus instead to replicate his study for \textit{electric}/\textit{electrical}. It is a study with two nominal variables: \textsc{Affix Variant} (with the values \textsc{-ic} and \textsc{-ical}), and \textsc{Semantic Category} (with a set of values to be discussed presently). Note that this design can be based straightforwardly on token frequency, as we are not concerned with the relationship between the stem and the affix, but with the relationship between the stem-affix combination and the nouns modified by it. Put differently, we are not using the token frequency of a stem-affix combination, but of the collocates of words derived by a particular affix.

Kaunisto uses a mixture of dictionaries and existing literature to identify potentially interesting values for the variable \textsc{Semantic Category}; we will restrict ourselves to dictionaries here. Consider the definitions from six major dictionaries in (\ref{ex:definitionelectric}) and (\ref{ex:definitionelectrical}):

\begin{exe}
\ex \textit{electric}
\begin{xlist} 
\label{ex:definitionelectric}
\ex connected with electricity; using, produced by or producing electricity (OALD)
\ex of or relating to electricity; operated by electricity (MW)
\ex working by electricity; used for carrying electricity; relating to electricity (MD)
\ex of, produced by, or worked by electricity (CALD)
\ex needing electricity to work, produced by electricity, or used for carrying electricity (LDCE)
\ex work$[$ing$]$ by means of electricity; produced by electricity; designed to carry electricity; refer$[$ring$]$ to the supply of electricity (Cobuild)
\end{xlist}
\end{exe}

\begin{exe}
\ex \textit{electrical}
\begin{xlist} 
\label{ex:definitionelectrical}
\ex connected with electricity; using or producing electricity (OALD)
\ex of or relating to electricity; operated by electricity (MW) $[$mentioned as synonym under corresp. sense of electric$]$
\ex working by electricity; relating to electricity (MD)
\ex related to electricity (CALD, LDCE)
\ex work[ing] by means of electricity; supply[ing] or us[ing] electricity; energy ... in the form of electricity; involved in the production and supply of electricity or electrical goods (Cobuild)
\end{xlist}
\end{exe}

MW treats the two words as largely synonymous and OALD distinguishes them only insofar as mentioning for \textit{electric}, but not \textit{electrical}, that it may refer to phenomena ``produced by electricity'' (this is meant to cover cases like \textit{electric current/charge}); however, since both words are also defined as referring to anything ``connected with electricity'', this is not much of a differentiation (the entry for \textit{electrical} also mentions \textit{electrical power/energy}). Macmillan's Dictionary also treats them as largely synonymous, although it is pointed out specifically that \textit{electric} refers to entities ``carrying electricity'' (citing \textit{electric outlet/plug/cord}). CALD and LDCE present \textit{electrical} as a more general word for anything ``related to electricity'', whereas they mention specifically that \textit{electric} is used for things ``worked by electricity'' (e.g. \textit{electric light/appliance}) or ``carrying electricity'' (presumably \textit{cords}, \textit{outlets} etc.) and phenomena produced by electricity (presumably \textit{current}, \textit{charge}, etc.). Collins presents both words as referring to electric(al) appliances, with \textit{electric} additionally referring to things ``produced by electricity'', ``designed to carry electricity'' or being related to the ``supply of electricity'' and \textit{electrical} additionally referring to ``energy'' or entities ``involved in the production and supply of electricity'' (presumably energy companies, engineers, etc.).

Summarizing, we can posit the following four broad values for our variable \textsc{Semantic Category}, with definitions that are hopefully specific enough to serve as an annotation scheme:

\begin{itemize}

\item \textsc{devices} and appliances working by electricity (\textit{light}, \textit{appliance}, etc.)

\item \textsc{energy} in the form of electricity (\textit{power}, \textit{current}, \textit{charge}, \textit{energy}, etc.)

\item the \textsc{industry} researching, producing or supplying energy, i.e. companies and the people working there (\textit{company}, \textit{engineer}, etc.)

\item \textsc{circuits}, broadly defined as entities producing or carrying electricity, including (\textit{cord}, \textit{outlet}, \textit{plug}, but also \textit{power plant} etc.)

\end{itemize}

The definitions are too heterogeneous to base a specific hypothesis on them, but we might broadly expect \textit{electric} to be more typical for the categories \textvv{device} and \textvv{circuit} and \textit{electrical} for the category \textit{industry}.

Table \ref{tab:electricalentitieslob} shows the token frequency with which nouns from these categories are referred to as \textit{electric} or \textit{electrical} in the LOB corpus; in order to understand how these nouns were categorized, it also lists all types found for each category (one example was discarded because it was metaphorical).

\begin{table}[!htbp]
\caption{Entities described as \textit{electric} or \textit{electrical} in the LOB corpus.}
\label{tab:electricalentitieslob}
\resizebox{0.9\textwidth}{!}{%
\begin{tabular}[t]{lccr}
\lsptoprule
 & \multicolumn{2}{c}{\textvv{Adjective}} & \\
\textvv{Noun} & \textvv{electric} & \textvv{electrical} & Total \\
\midrule
\textvv{\makecell[tl]{device}}
	& \makecell[t]{\begin{tabular}[t]{lS[table-format=2.2]} 
		\small{\textit{Obs.:}} & 17 \\ 
		\small{\textit{Exp.:}} & 12.077922 \\ 
		\small{\textit{$\chi^2$:}} & 2.005879 \\
		\multicolumn{2}{l}{
			\begin{minipage}[t]{0.3\textwidth} \raggedright
			\footnotesize{\textit{bulb},  \textit{calculating machine},  \textit{chair},  \textit{cooker},  \textit{dog},  \textit{drill},  \textit{fence},  \textit{fire},  \textit{heating element},  \textit{light switch},  \textit{motor},  \textit{mowing},  \textit{stove},  \textit{torch},  \textit{tricycle}}
			\end{minipage}}
		\end{tabular}}
	& \makecell[t]{\begin{tabular}[t]{lS[table-format=2.2]} 
		\small{\textit{Obs.:}} & 13 \\ 
		\small{\textit{Exp.:}} & 17.922078 \\ 
		\small{\textit{$\chi^2$:}} & 1.351788 \\ 
		\multicolumn{2}{l}{
			\begin{minipage}[t]{0.3\textwidth} \raggedright
			\footnotesize{\textit{amplifier},  \textit{apparatus},  \textit{fire},  \textit{goods},  \textit{machine},  \textit{machinery},  \textit{power unit},  \textit{sign},  \textit{supply},  \textit{system},  \textit{system},  \textit{transmission}}
		\end{minipage}}
		\end{tabular}}
	& 30 \\[3.1cm]
\textvv{\makecell[tl]{energy}}
	& \makecell[t]{\begin{tabular}[t]{lS[table-format=2.2]} 
		\small{\textit{Obs.:}} & 11 \\ 
		\small{\textit{Exp.:}} & 9.662338 \\ 
		\small{\textit{$\chi^2$:}} & 0.185187  \\
		\multicolumn{2}{l}{
			\begin{minipage}[t]{0.3\textwidth} \raggedright
			\footnotesize{\textit{attraction},  \textit{bill},  \textit{blue},  \textit{current},  \textit{effect},  \textit{field},  \textit{force},  \textit{light}, \textit{space constant}}
			\end{minipage}}
		\end{tabular}}
	& \makecell[t]{\begin{tabular}[t]{lS[table-format=2.2]} 
		\small{\textit{Obs.:}} & 13 \\ 
		\small{\textit{Exp.:}} & 14.337662 \\ 
		\small{\textit{$\chi^2$:}} & 0.124800  \\
		\multicolumn{2}{l}{
			\begin{minipage}[t]{0.3\textwidth} \raggedright
			\footnotesize{\textit{accident}, \textit{activity},  \textit{condition},  \textit{load},  \textit{output}, \textit{phenomenon},  \textit{property}, \textit{resistance}}
			\end{minipage}}
		\end{tabular}}
	& 24 \\[2.4cm]
\textvv{\makecell[tl]{circuit}}
	& \makecell[t]{\begin{tabular}[t]{lS[table-format=2.2]} 
		\small{\textit{Obs.:}} & 2 \\ 
		\small{\textit{Exp.:}} & 2.012987 \\ 
		\small{\textit{$\chi^2$:}} & 0.00000008  \\
		\multicolumn{2}{l}{
			\begin{minipage}[t]{0.3\textwidth} \raggedright
			\footnotesize{\textit{battery},  \textit{line}}
			\end{minipage}}
		\end{tabular}}
	& \makecell[t]{\begin{tabular}[t]{lS[table-format=2.2]} 
		\small{\textit{Obs.:}} & 3 \\ 
		\small{\textit{Exp.:}} & 2.987013 \\ 
		\small{\textit{$\chi^2$:}} & 0.564652  \\
		\multicolumn{2}{l}{
			\begin{minipage}[t]{0.3\textwidth} \raggedright
			\footnotesize{\textit{circuit},  \textit{conductivity}}
			\end{minipage}}
		\end{tabular}}
	& 5 \\[1.6cm]
\textvv{\makecell[tl]{industry}}
	& \makecell[t]{\begin{tabular}[t]{lS[table-format=2.2]} 
		\small{\textit{Obs.:}} & 1 \\ 
		\small{\textit{Exp.:}} & 7.246753 \\ 
		\small{\textit{$\chi^2$:}} & 5.384746  \\
		\multicolumn{2}{l}{
			\begin{minipage}[t]{0.3\textwidth} \raggedright
			\footnotesize{\textit{company}}
			\end{minipage}}
		\end{tabular}}
	& \makecell[t]{\begin{tabular}[t]{lS[table-format=2.2]} 
		\small{\textit{Obs.:}} & 17 \\ 
		\small{\textit{Exp.:}} & 10.753247 \\ 
		\small{\textit{$\chi^2$:}} & 3.628851  \\
		\multicolumn{2}{l}{
			\begin{minipage}[t]{0.3\textwidth} \raggedright
			\footnotesize{\textit{communication theory},  \textit{counterpart},  \textit{development},  \textit{engineer}, \textit{industry}, \textit{trade},  \textit{work}}
			\end{minipage}}
		\end{tabular}}
	& 18 \\[2.8cm]
\midrule
Total
	& \makecell[t]{31}
	& \makecell[t]{46}
	& \makecell[t]{77} \\
\lspbottomrule
\end{tabular}}
\end{table}
% query: LOB_LEGACY; [word="electric(al)?"%c][pos="N.*"]

The difference between \textit{electric} and \textit{electrical} is significant overall ($\chi^2$ = 12.68, df = 3, p < 0.01, $\phi$ = 0.2869), suggesting that the two words somehow differ with respect to their preferences for these categories. In order to determine the nature of these preferences, we need to look at the individual $\chi^2$ components,
 Since we are interested in the nature of this difference, it is much more insightful to look at the chi-square components individually. This gives us a better idea where the overall significant difference comes from. In this case, it comes almost exclusively from the fact that \textit{electrical} is indeed associated with the research and supply of electricity (\textvv{industry}), although there is a slight preference for \textit{electric} with nouns referring to devices. Generally, the two words seem to be relatively synonymous, at least in 1960s British English.

Let us repeat the study with the BROWN corpus. Table \ref{tab:electricalentitiesbrown} lists the token frequencies for the individual categories and, again, all types found for each category.

\begin{table}[!htbp]
\caption{Entities described as \textit{electric} or \textit{electrical} in the BROWN corpus}
\label{tab:electricalentitiesbrown}
\resizebox{0.9\textwidth}{!}{%
\begin{tabular}[t]{lccr}
\lsptoprule
 & \multicolumn{2}{c}{\textvv{Adjective}} & \\
\textvv{Noun} & \textvv{electric} & \textvv{electrical} & Total \\
\midrule
\textvv{\makecell[tl]{device}}
	& \makecell[t]{\begin{tabular}[t]{lS[table-format=2.2]}
		\small{\textit{Obs.:}} & 29 \\ 
		\small{\textit{Exp.:}} & 18.285714 \\ 
		\small{\textit{$\chi^2$:}} & 6.2779018 \\
		\multicolumn{2}{l}{
			\begin{minipage}[t]{0.3\textwidth} \raggedright
			\footnotesize{\textit{amplifier}, \textit{blanket}, \textit{bug}, \textit{chair} \textit{computer}, \textit{drive}, \textit{gadget}, \textit{hand tool}, \textit{hand-blower},\textit{heater}, \textit{heater}, \textit{horn}, \textit{icebox}, \textit{lantern}, \textit{model}, \textit{range}, \textit{razor}, \textit{refrigerator}, \textit{signs}, \textit{spit}, \textit{toothbrush}}
			\end{minipage}}
	  \end{tabular}}
	& \makecell[t]{\begin{tabular}[t]{lS[table-format=2.2]}
		\small{\textit{Obs.:}} & 3 \\ 
		\small{\textit{Exp.:}} & 13.714286 \\ 
		\small{\textit{$\chi^2$:}} & 8.3705357 \\
		\multicolumn{2}{l}{
			\begin{minipage}[t]{0.3\textwidth} \raggedright
			\footnotesize{\textit{control}, \textit{display}, \textit{torquers}}
			\end{minipage}}
	  \end{tabular}}
	& 32 \\[3.8cm]
\textvv{\makecell[tl]{energy}}
	& \makecell[t]{\begin{tabular}[t]{lS[table-format=2.2]} 
		\small{\textit{Obs.:}} & 15 \\ 
		\small{\textit{Exp.:}} & 18.285714 \\ 
		\small{\textit{$\chi^2$:}} & 0.5904018 \\
		\multicolumn{2}{l}{
			\begin{minipage}[t]{0.3\textwidth} \raggedright
			\footnotesize{\textit{arc}, \textit{current}, \textit{discharge}, \textit{power}, \textit{rate}, \textit{shock}, \textit{universe}, \textit{utility rate}}
			\end{minipage}}
		\end{tabular}}
	& \makecell[t]{\begin{tabular}[t]{lS[table-format=2.2]}
		\small{\textit{Obs.:}} & 17 \\ 
		\small{\textit{Exp.:}} & 13.714286 \\ 
		\small{\textit{$\chi^2$:}} & 0.7872024 \\
		\multicolumn{2}{l}{
			\begin{minipage}[t]{0.3\textwidth} \raggedright
			\footnotesize{\textit{body}, \textit{characteristic}, \textit{charges}, \textit{distribution}, \textit{energy}, \textit{force}, \textit{form}, \textit{power}, \textit{shock}, \textit{signal}, \textit{stimulation}}
			\end{minipage}}
		\end{tabular}}
	& 32 \\[2.7cm]
\textvv{\makecell[tl]{circuit}}
	& \makecell[t]{\begin{tabular}[t]{lS[table-format=2.2]} 
		\small{\textit{Obs.:}} & 4 \\ 
		\small{\textit{Exp.:}} & 7.428571 \\ 
		\small{\textit{$\chi^2$:}} & 1.5824176 \\
		\multicolumn{2}{l}{
			\begin{minipage}[t]{0.3\textwidth} \raggedright
			\footnotesize{\textit{circuit}, \textit{light plant}, \textit{power plant}}
			\end{minipage}}
		\end{tabular}}
	& \makecell[t]{\begin{tabular}[t]{lS[table-format=2.2]} 
		\small{\textit{Obs.:}} & 9 \\ 
		\small{\textit{Exp.:}} & 5.571429 \\ 
		\small{\textit{$\chi^2$:}} & 2.1098901 \\
		\multicolumn{2}{l}{
			\begin{minipage}[t]{0.3\textwidth} \raggedright
			\footnotesize{\textit{contact} \textit{line}, \textit{outlet}, \textit{pickoff}, \textit{wire}, \textit{wiring}}
			\end{minipage}}
		\end{tabular}}
	& 13 \\[2cm]
\textvv{\makecell[tl]{industry}}
	& \makecell[t]{\begin{tabular}[t]{lS[table-format=2.2]}
		\small{\textit{Obs.:}} & 8 \\ 
		\small{\textit{Exp.:}} & 12.000000 \\ 
		\small{\textit{$\chi^2$:}} & 1.3333333 \\
		\multicolumn{2}{l}{
			\begin{minipage}[t]{0.3\textwidth} \raggedright
			\footnotesize{\textit{company}, \textit{corporation}, \textit{inc.} \textit{utility business}, \textit{utility company}}
			\end{minipage}}
		\end{tabular}}
	& \makecell[t]{\begin{tabular}[t]{lS[table-format=2.2]} 
		\small{\textit{Obs.:}} & 13 \\ 
		\small{\textit{Exp.:}} & 9.000000 \\ 
		\small{\textit{$\chi^2$:}} & 1.7777778 \\
		\multicolumn{2}{l}{
			\begin{minipage}[t]{0.3\textwidth} \raggedright
			\footnotesize{\textit{case}, \textit{company}, \textit{discovery}, \textit{engineer}, \textit{equipment}, \textit{literature}, \textit{manufacturer}, \textit{work}}
			\end{minipage}}
		\end{tabular}}
	& 21 \\[2.7cm]
\midrule
Total
	& \makecell[t]{56}
	& \makecell[t]{42}
	& \makecell[t]{98} \\
\lspbottomrule
\end{tabular}}
\end{table}
% query: BROWN_LEGACY [word="electric(al)?"%c][pos="N.*"]

Again, the overall difference between the two words is significant and the effect is slightly stronger than in the LOB corpus ($\chi^2$ = 22.83, df = 3, p < 0.001, $\phi$ = 0.3413), suggesting a stronger differentiation between them. Again, the most interesting question is where the effect comes from. In this case, devices are much more frequently referred to as \textit{electric} and less frequently as \textit{electrical} than expected, and, as in the LOB corpus, the nouns in the category \textit{industry} are more frequently referred to as \textit{electrical} and less frequently as \textit{electric} than expected (although not significantly so). Again, there is no clear difference with respect to the remaining two categories.

Broadly speaking, then, one of our expectations is borne out by the British English data and one by the American English data. We would now have to look at larger corpora to see whether this is an actual difference between the two varieties or whether it is an accidental feature of the corpora used here. We might also want to look at more modern corpora -- the importance of electricity in our daily lives has changed quite drastically even since the 1960s, so the words may have specialized semantically more clearly in the meantime. Finally, we would look more closely at the categories we have used, to see whether a different or a more fine-grained categorization might reveal additional insights (\citet{kaunisto_electric/electrical_1999} goes on to look at his categories in more detail, revealing more fine-grained differences between the words).

Of course, this type of investigation can also be designed as an inductive study of differential collocates (again, like the study of synonyms such as \textit{high} and \textit{tall}). Let us look at the nominal collocates of \textit{electric} and \textit{electrical} in the BNC. Table \ref{tab:electricalcollocates} shows the results of a differential-collocate analysis, calculated on the basis of all occurrences of \textit{electric}/\textit{al} in the BNC that are directly followed by a noun.

\begin{table}[!htbp]
\caption{Differential nominal collocates of \textit{electric} and \textit{electrical} in the BNC}
\label{tab:electricalcollocates}
\resizebox{.9\textwidth}{!}{%
\begin{tabular}[t]{l *{2}{S[table-format=4]} *{2}{S[table-format=8]} S}
\lsptoprule
\multicolumn{1}{c}{\makecell[tc]{\textvv{Collocate}}} & \multicolumn{1}{c}{\makecell[tc]{Frequency with \\ \textit{electric}}} & \multicolumn{1}{c}{\makecell[tc]{Frequency with \\ \textit{electrical}}} & \multicolumn{1}{c}{\makecell[tc]{Other words \\ with \textit{electric}}} & \multicolumn{1}{c}{\makecell[tc]{Other words \\ with \textit{electrical}}} & \multicolumn{1}{c}{\makecell[tc]{G\textsuperscript{2}}} \\
\midrule
\multicolumn{6}{l}{Most strongly associated with \textvv{electric}} \\
\midrule
\textit{shock} & 140 & 2 & 2692 & 2027 & 136.596184292028 \\
\textit{light} & 122 & 2 & 2710 & 2027 & 116.986764698069 \\
\textit{field} & 191 & 23 & 2641 & 2006 & 104.414114083124 \\
\textit{guitar} & 81 & 0 & 2751 & 2029 & 88.5041462972421 \\
\textit{fire} & 109 & 7 & 2723 & 2022 & 78.6162595911962 \\
\textit{car} & 59 & 2 & 2773 & 2027 & 50.1120604925271 \\
\textit{motor} & 63 & 3 & 2769 & 2026 & 49.4235514184813 \\
\textit{blanket} & 46 & 1 & 2786 & 2028 & 42.0683900773617 \\
\textit{window} & 38 & 0 & 2794 & 2029 & 41.2741838432036 \\
\textit{kettle} & 37 & 0 & 2795 & 2029 & 40.1824906243125 \\
\textit{cooker} & 34 & 0 & 2798 & 2029 & 36.9092161067169 \\
\textit{drill} & 39 & 1 & 2793 & 2028 & 34.7452072463329 \\
\textit{train} & 32 & 0 & 2800 & 2029 & 34.7285356794156 \\
\textit{co} & 27 & 0 & 2805 & 2029 & 29.2820837744337 \\
\textit{vehicle} & 24 & 0 & 2808 & 2029 & 26.01780530035 \\
\textit{fan} & 21 & 0 & 2811 & 2029 & 22.7562157687947 \\
\textit{lighting} & 21 & 0 & 2811 & 2029 & 22.7562157687947 \\
\textit{fence} & 20 & 0 & 2812 & 2029 & 21.6696160169524 \\
\textit{tramway} & 20 & 0 & 2812 & 2029 & 21.6696160169524 \\
\textit{traction} & 19 & 0 & 2813 & 2029 & 20.5833143648945 \\
\midrule
\multicolumn{6}{l}{Most strongly associated with \textvv{electrical}} \\
\midrule
\textit{engineering} & 0 & 108 & 2832 & 1921 & 192.155410246783 \\
\textit{engineer} & 0 & 89 & 2832 & 1940 & 157.841521334624 \\
\textit{equipment} & 6 & 106 & 2826 & 1923 & 147.963093203514 \\
\textit{goods} & 1 & 88 & 2831 & 1941 & 146.119804989242 \\
\textit{activity} & 1 & 85 & 2831 & 1944 & 140.794153692421 \\
\textit{appliance} & 3 & 69 & 2829 & 1960 & 100.175734209633 \\
\textit{conductivity} & 0 & 35 & 2832 & 1994 & 61.5137128862365 \\
\textit{fault} & 0 & 34 & 2832 & 1995 & 59.7462565189791 \\
\textit{signal} & 8 & 53 & 2824 & 1976 & 54.5024482307519 \\
\textit{stimulation} & 0 & 26 & 2832 & 2003 & 45.6277447889228 \\
\textit{union} & 0 & 26 & 2832 & 2003 & 45.6277447889228 \\
\textit{energy} & 2 & 33 & 2830 & 1996 & 44.7814924535317 \\
\textit{impulse} & 2 & 32 & 2830 & 1997 & 43.1354879493866 \\
\textit{retailer} & 0 & 21 & 2832 & 2008 & 36.8226961228705 \\
\textit{property} & 0 & 20 & 2832 & 2009 & 35.0634364130477 \\
\textit{work} & 0 & 19 & 2832 & 2010 & 33.3047590856413 \\
\textit{control} & 1 & 23 & 2831 & 2006 & 33.1002973183347 \\
\textit{system} & 6 & 31 & 2826 & 1998 & 28.0594333955953 \\
\textit{circuit} & 10 & 37 & 2822 & 1992 & 27.063158652186 \\
\textit{recording} & 1 & 19 & 2831 & 2010 & 26.4369853256406 \\
\lspbottomrule
\end{tabular}}
\end{table}
% query: BNC; [word="electric(al)?"%c][pos=".*NN.*"]; count Last by hw

The results largely agree with the preferences also uncovered by the more careful (and more time-consuming) categorization of a complete data set, with one crucial difference: there are members of the category \textsc{device} among the significant differential collocates of both variants. A closer look reveals a systematic difference within this category: the \textsc{device} collocates of \textit{electric} refer to specific devices (such as \textit{light}, \textit{guitar}, \textit{light}, \textit{kettle} etc.); in contrast, the \textsc{device} collocates of \textit{electrical} refer to general classes of devices (\textit{equipment}, \textit{appliance}, \textit{system}). This difference was not discernible in the LOB and BROWN datasets (presumably because they were too small, but it is discernible in the data set used by \citet{kaunisto_electric/electrical_1999}, who posits corresponding subcategories. Of course, the BNC is a much more recent corpus than LOB and BROWN, so, again, a diachronic comparison would be interesting.

There is an additional pattern that would warrant further investigation: there are collocates for both variants that correspond to what some of the dictionaries we consulted refer to as ``produced by energy'': \textit{shock}, \textit{field} and \textit{fire} for \textit{electric} and \textit{signal}, \textit{energy}, \textit{impulse} for \textit{electrical}. It is possible that \textit{electric} more specifically characterizes phenomena that are \textit{caused} by electricity, while \textit{electrical} characterizes phenomena that \textit{manifest} electricity. 

The case study demonstrates, then, that a differential-collocate analysis is a good alternative to the manual categorization and category-wise comparison of all collocates: it allows us to process very large data-sets very quickly and then focus on the semantic properties of those collocates we already know to distinguish significantly between the variants.

We must keep in mind, however, that this kind of study does not primarily uncover differences between affixes, but differences between specific word pairs containing these affixes. They are, as pointed out above, essentially lexical studies of near-synonymy. Of course, it is possible that by performing such analyses for a large number of word pairs containing a particular affix pair, general semantic differences may emerge, but since we are frequently dealing with highly lexicalized forms, there is no guarantee for this. \citet{gries_corpus-linguistic_2001, gries_testing_2003} has shown that \textit{-ic}/\textit{-ical} pairs differ substantially in the extent to which they are synonymous; for example, he finds substantial difference in meaning for \textit{politic}/\textit{political} or \textit{poetic}/\textit{poetical}, but much smaller differences, for example, for \textit{bibliographic}/\textit{bibliographical}, with \textit{electric}/\textit{electrical} somewhere in the middle. Obviously, the two variants have lexicalized independently in many cases, and the specific differences in meaning resulting from this lexicalization process are unlikely to fall into clear general categories.

\subsubsection{Case study: Phonological differences between \textit{-ic} and \textit{-ical}}
\label{sec:phonologicaldifferencesbetweenicandical}

In an interesting but rarely-cited paper, \citet{or_corpus-based_1994} collects a number of hypotheses about semantic and, in particular, phonological factors influencing the distribution of \textit{-ic} and \textit{-ical} that she provides impressionistic corpus evidence for but does not investigate systematically. A simple example is the factor \textsc{Length}: Or hypothesizes that speakers will tend to avoid long words and choose the shorter variant \textit{-ic} for long stems (in terms of number of syllables). She reports that ``a survey of a general vocabulary list'' corroborates this hypothesis but does not present any systematic data.

Let us test this hypothesis using the LOB corpus. Since this is a written corpus, let us define \textvv{Length} in terms of letters and assume that this is a sufficiently close approximation to phonological length. Table \ref{tab:icicallength} lists all types with the two suffixes from LOB, in decreasing order of length (since the point here is to show the influence of length on suffix choice, prefixed stems, compound stems etc. are included in their full form).

\begin{table}[!htbp]
\caption{Adjectives with \textit{-ic} and \textit{-ical} by length (LOB)}
\label{tab:icicallength}
\resizebox*{!}{\textheight}{%
\begin{tabular}[t]{l}
\lsptoprule
Types with \textit{-ic} \\
\midrule
\makecell[tl]{
\begin{minipage}[t]{\textwidth} \raggedright
\textit{function-theoretic}, \textit{non-stoichiometric} (15), \textit{crystallographic}, \textit{politico-economic}, \textit{pseudo-scientific}, \textit{uncharacteristic} (14), \textit{antihaemophilic}, \textit{electro(-)magnetic}, \textit{non-pornographic}, \textit{semi-logarithmic} (13), \textit{characteristic}, \textit{claustrophobic}, \textit{electrographic}, \textit{metallographic}, \textit{non-diastematic}, \textit{part-apologetic}, \textit{potentiometric}, \textit{quasi-acrobatic}, \textit{spectrographic}, \textit{thermoelectric} (12), \textit{architectonic}, \textit{choreographic}, \textit{deterministic}, \textit{electrostatic}, \textit{ferromagnetic}, \textit{hypereutectic}, \textit{idiosyncratic}, \textit{materialistic}, \textit{nationalistic}, \textit{non-parametric}, \textit{philanthropic}, \textit{probabilistic}, \textit{pseudomorphic}, \textit{thermodynamic}, \textit{trans-economic}, \textit{unsympathetic} (11), \textit{aristocratic}, \textit{bureaucratic}, \textit{catastrophic}, \textit{electrolytic}, \textit{enthusiastic}, \textit{evangelistic}, \textit{haemorrhagic}, \textit{hieroglyphic}, \textit{homoeopathic}, \textit{hydrochloric}, \textit{hydrofluoric}, \textit{journalistic}, \textit{kinaesthetic}, \textit{melodramatic}, \textit{meritocratic}, \textit{monosyllabic}, \textit{naturalistic}, \textit{neurasthenic}, \textit{non-alcoholic}, \textit{orthographic}, \textit{paramagnetic}, \textit{philharmonic}, \textit{photographic}, \textit{phylogenetic}, \textit{polysyllabic}, \textit{pornographic}, \textit{positivistic}, \textit{programmatic}, \textit{prophylactic}, \textit{thermometric}, \textit{unscientific} (10), \textit{aerodynamic}, \textit{anaesthetic}, \textit{apocalyptic}, \textit{atmospheric}, \textit{demographic}, \textit{endocentric}, \textit{gastronomic}, \textit{geostrophic}, \textit{gravimetric}, \textit{haemophilic}, \textit{logarithmic}, \textit{macroscopic}, \textit{mechanistic}, \textit{meromorphic}, \textit{microscopic}, \textit{modernistic}, \textit{monotechnic}, \textit{non-catholic}, \textit{non-dogmatic}, \textit{non-dramatic}, \textit{ontogenetic}, \textit{orthopaedic}, \textit{over-drastic}, \textit{pessimistic}, \textit{philosophic}, \textit{phototactic}, \textit{plutocratic}, \textit{prehistoric}, \textit{psychiatric}, \textit{ritualistic}, \textit{socialistic}, \textit{sycophantic}, \textit{syllogistic}, \textit{sympathetic}, \textit{symptomatic}, \textit{syntagmatic}, \textit{telegraphic}, \textit{tetra-acetic}, \textit{therapeutic}, \textit{unaesthetic}, \textit{unpatriotic}, \textit{unrealistic} (9), \textit{aldermanic}, \textit{altruistic}, \textit{antiseptic}, \textit{apologetic}, \textit{asymmetric}, \textit{asymptotic}, \textit{autocratic}, \textit{barometric}, \textit{bimetallic}, \textit{cabalistic}, \textit{concentric}, \textit{corybantic}, \textit{democratic}, \textit{dielectric}, \textit{diplomatic}, \textit{egocentric}, \textit{electronic}, \textit{eulogistic}, \textit{exocentric}, \textit{fatalistic}, \textit{geocentric}, \textit{haemolytic}, \textit{histrionic}, \textit{humanistic}, \textit{hyperbolic}, \textit{hypodermic}, \textit{idealistic}, \textit{legalistic}, \textit{linguistic}, \textit{megalithic}, \textit{monolithic}, \textit{nihilistic}, \textit{novelistic}, \textit{optimistic}, \textit{pentatonic}, \textit{philatelic}, \textit{phosphoric}, \textit{polyphonic}, \textit{scholastic}, \textit{scientific}, \textit{stochastic}, \textit{supersonic}, \textit{systematic}, \textit{telepathic}, \textit{telephonic}, \textit{telescopic}, \textit{theocratic}, \textit{ultrasonic}, \textit{undogmatic}, \textit{undramatic}, \textit{uneconomic}, \textit{volumetric} (8), \textit{acrobatic}, \textit{aesthetic}, \textit{alcoholic}, \textit{algebraic}, \textit{analgesic}, \textit{antigenic}, \textit{apathetic}, \textit{apostolic}, \textit{authentic}, \textit{automatic}, \textit{axiomatic}, \textit{ballistic}, \textit{catalytic}, \textit{chromatic}, \textit{cinematic}, \textit{dualistic}, \textit{eccentric}, \textit{energetic}, \textit{enigmatic}, \textit{fantastic}, \textit{geometric}, \textit{geotactic}, \textit{heuristic}, \textit{homonymic}, \textit{hydraulic}, \textit{inorganic}, \textit{intrinsic}, \textit{kinematic}, \textit{lethargic}, \textit{messianic}, \textit{morphemic}, \textit{neolithic}, \textit{nicotinic}, \textit{nostalgic}, \textit{nucleonic}, \textit{panoramic}, \textit{parabolic}, \textit{paralytic}, \textit{parasitic}, \textit{patriotic}, \textit{pneumatic}, \textit{pragmatic}, \textit{prophetic}, \textit{quadratic}, \textit{realistic}, \textit{sarcastic}, \textit{schematic}, \textit{spasmodic}, \textit{strategic}, \textit{stylistic}, \textit{sub-atomic}, \textit{sulphuric}, \textit{symphonic}, \textit{syntactic}, \textit{synthetic}, \textit{telegenic}, \textit{traumatic} (7), \textit{academic}, \textit{acoustic}, \textit{allergic}, \textit{anarchic}, \textit{aromatic}, \textit{artistic}, \textit{asthenic}, \textit{athletic}, \textit{barbaric}, \textit{carbolic}, \textit{cathodic}, \textit{catholic}, \textit{cherubic}, \textit{climatic}, \textit{cyclonic}, \textit{diabetic}, \textit{dietetic}, \textit{dogmatic}, \textit{domestic}, \textit{dramatic}, \textit{dynastic}, \textit{eclectic}, \textit{economic}, \textit{egoistic}, \textit{electric}, \textit{elliptic}, \textit{emphatic}, \textit{epigonic}, \textit{esoteric}, \textit{forensic}, \textit{galvanic}, \textit{gigantic}, \textit{heraldic}, \textit{historic}, \textit{horrific}, \textit{hygienic}, \textit{hypnotic}, \textit{isotopic}, \textit{magnetic}, \textit{majestic}, \textit{metallic}, \textit{monastic}, \textit{narcotic}, \textit{neurotic}, \textit{operatic}, \textit{pathetic}, \textit{pedantic}, \textit{periodic}, \textit{phonetic}, \textit{platonic}, \textit{podzolic}, \textit{potassic}, \textit{prolific}, \textit{quixotic}, \textit{rhythmic}, \textit{romantic}, \textit{sadistic}, \textit{sardonic}, \textit{semantic}, \textit{specific}, \textit{sporadic}, \textit{syllabic}, \textit{symbolic}, \textit{synaptic}, \textit{synoptic}, \textit{tectonic}, \textit{terrific}, \textit{theistic}, \textit{thematic}, \textit{volcanic} (6), \textit{amoebic}, \textit{angelic}, \textit{anionic}, \textit{aquatic}, \textit{archaic}, \textit{botanic}, \textit{bucolic}, \textit{caustic}, \textit{ceramic}, \textit{chaotic}, \textit{chronic}, \textit{classic}, \textit{cryptic}, \textit{delphic}, \textit{demonic}, \textit{drastic}, \textit{dynamic}, \textit{elastic}, \textit{endemic}, \textit{enteric}, \textit{erratic}, \textit{frantic}, \textit{gastric}, \textit{generic}, \textit{genetic}, \textit{glottic}, \textit{graphic}, \textit{idyllic}, \textit{kinetic}, \textit{laconic}, \textit{lunatic}, \textit{melodic}, \textit{motivic}, \textit{nomadic}, \textit{numeric}, \textit{oceanic}, \textit{oolitic}, \textit{organic}, \textit{pacific}, \textit{phallic}, \textit{politic}, \textit{prosaic}, \textit{psychic}, \textit{spastic} (5), \textit{acetic}, \textit{arctic}, \textit{atomic}, \textit{bardic}, \textit{cosmic}, \textit{cyclic}, \textit{cystic}, \textit{erotic}, \textit{exotic}, \textit{ferric}, \textit{gnomic}, \textit{gothic}, \textit{hectic}, \textit{heroic}, \textit{ironic}, \textit{italic}, \textit{mosaic}, \textit{myopic}, \textit{mystic}, \textit{mythic}, \textit{niobic}, \textit{nitric}, \textit{oxalic}, \textit{photic}, \textit{poetic}, \textit{public}, \textit{rustic}, \textit{scenic}, \textit{static}, \textit{tannic}, \textit{tragic} (4), \textit{basic}, \textit{civic}, \textit{comic}, \textit{cubic}, \textit{ionic}, \textit{lyric}, \textit{magic}, \textit{tonic}, \textit{toxic} (3), \textit{epic} (2)
\end{minipage}} \\
\midrule
Types with \textit{-ical} \\
\midrule
\makecell[tl]{
\begin{minipage}[t]{\textwidth} \raggedright
\textit{autobiographical}, \textit{palaeontological} (12), \textit{anthropological}, \textit{pharmacological}, \textit{physiographical}, \textit{trigonometrical} (11), \textit{archaeological}, \textit{ecclesiastical}, \textit{eschatological}, \textit{haematological}, \textit{iconographical}, \textit{meteorological}, \textit{methodological}, \textit{pharmaceutical}, \textit{quasi-classical} (10), \textit{chronological}, \textit{entomological}, \textit{metallurgical}, \textit{morphological}, \textit{philosophical}, \textit{photochemical}, \textit{physiological}, \textit{psychological}, \textit{radiochemical}, \textit{technological}, \textit{theorematical}, \textit{toxicological}, \textit{unsymmetrical}, \textit{untheological} (9), \textit{aeronautical}, \textit{aetiological}, \textit{alphabetical}, \textit{anti-clerical}, \textit{arithmetical}, \textit{astronomical}, \textit{biographical}, \textit{cosmological}, \textit{etymological}, \textit{genealogical}, \textit{geographical}, \textit{hypocritical}, \textit{hypothetical}, \textit{mathematical}, \textit{metaphorical}, \textit{metaphysical}, \textit{non-technical}, \textit{pathological}, \textit{pre-classical}, \textit{radiological}, \textit{schismatical}, \textit{self-critical}, \textit{sociological}, \textit{systematical}, \textit{uneconomical}, \textit{unrhythmical} (8), \textit{allegorical}, \textit{biochemical}, \textit{categorical}, \textit{cylindrical}, \textit{dialectical}, \textit{egotistical}, \textit{evangelical}, \textit{geometrical}, \textit{grammatical}, \textit{ideological}, \textit{monarchical}, \textit{mycological}, \textit{nonsensical}, \textit{obstetrical}, \textit{paradisical}, \textit{paradoxical}, \textit{pedagogical}, \textit{pedological}, \textit{puranitical}, \textit{purinatical}, \textit{puritanical}, \textit{serological}, \textit{statistical}, \textit{sub(-)tropical}, \textit{subtropical}, \textit{symmetrical}, \textit{synagogical}, \textit{theological}, \textit{theoretical}, \textit{unpractical} (7), \textit{a-political}, \textit{acoustical}, \textit{analytical}, \textit{anatomical}, \textit{biological}, \textit{diabolical}, \textit{economical}, \textit{ecumenical}, \textit{electrical}, \textit{elliptical}, \textit{geological}, \textit{historical}, \textit{hysterical}, \textit{liturgical}, \textit{mechanical}, \textit{methodical}, \textit{oratorical}, \textit{rhetorical}, \textit{symbolical}, \textit{theatrical}, \textit{unbiblical}, \textit{uncritical} (6), \textit{classical}, \textit{empirical}, \textit{fanatical}, \textit{graphical}, \textit{identical}, \textit{juridical}, \textit{numerical}, \textit{political}, \textit{satirical}, \textit{sceptical}, \textit{spherical}, \textit{technical}, \textit{unethical}, \textit{unmusical}, \textit{untypical}, \textit{whimsical} (5), \textit{atypical}, \textit{biblical}, \textit{cervical}, \textit{chemical}, \textit{clerical}, \textit{clinical}, \textit{critical}, \textit{cyclical}, \textit{inimical}, \textit{ironical}, \textit{metrical}, \textit{mystical}, \textit{mythical}, \textit{physical}, \textit{poetical}, \textit{surgical}, \textit{tactical}, \textit{tropical}, \textit{vertical} (4), \textit{comical}, \textit{conical}, \textit{cynical}, \textit{ethical}, \textit{lexical}, \textit{lyrical}, \textit{magical}, \textit{medical}, \textit{optical}, \textit{topical}, \textit{typical} (3)
\end{minipage}} \\
\lspbottomrule
\end{tabular}}
\end{table}

We can test the hypothesis based on the mean length of the two samples using a t-test, or by ranking them by length using the U test. As mentioned in Chapter \ref{ch:significancetesting}, word length (however we measure it) rarely follows a normal distribution, so the U test would probably be the better choice in this case, but let us use the t-test for the sake of practice (the data are there in full, if you want to calculate a U test).

There are 373 stem types occurring with \textit{-ic} in the LOB corpus, with a mean length of 7.32 and a sample variance of 5.72; there are 153 stem types occurring with \textit{-ical}, with a mean length of 6.60 and a sample variance of 4.57. Applying the formula in (\ref{ex:formulawelchst}) from Chapter \ref{ch:significancetesting}, we get a t-value of 2.97. There are 314.31 degrees of freedom in our sample (as calculated using the formula in (\ref{ex:formulawelchdf}), which means that p < 0.001. In other words, length (as measured in letters) seems to have an influence on the choice between the two affixes, with longer stems favoring \textit{-ic}.

This case study has demonstrated the use of a relatively simple operationalization to test a hypothesis about phonological length. We have used samples of types rather than samples of tokens, as we wanted to determine the influence of stem length on affix choice -- in this context, the crucial question is how many stems of a given length occur with a particular affix variant, but it does not matter how \textit{often} a particular stem does so. However, if there was more variation between the two suffixes, the frequency with which a particular stem is used with a particular affix might be interesting, as it would allow us to approach the question by ranking stems in terms of their preference and then correlating this ranking with their length.

\subsubsection{Case study: Affix combinations}
\label{sec:affixcombinations}

It has sometimes been observed that certain derivational affixes show a preference for stems that are already derived by a particular affix. For example, \citet{lindsay_rival_2011} and \citet{lindsay_natural_2013} show that while \textit{-ic} is more productive in general than \textit{-ical}, \textit{-ical} is more productive with stems that contain the affix \textit{-olog-} (for example, \textit{morphological} or \textit{methodological}). Such observations are  interesting from a descriptive viewpoint, as such preferences need to be taken into account, for example, in dictionaries, in teaching word-formation to non-native speakers or when making or assessing stylistic choices. They are also interesting from a theoretical perspective, first, because they need to be modeled and explained within any morphological theory, and second, because they may interact with other factors in a variety of ways.

For example, Case Study \ref{sec:phonologicaldifferencesbetweenicandical} demonstrates that longer stems generally seem to prefer the variant \textit{ic}; however, the mean length of derived stems is necessarily longer than that of non-derived stems, so it is puzzling, at first glance, that stems with the affix \textit{-olog-} should prefer \textit{-ical}. Of course, \textit{-olog-} may be an exception, with derived stems in general preferring the shorter \textit{-ic}; we would still need to account for this exceptional behavior however.

But let us start more humbly by laying the empirical foundations for such discussions and test the observation by \citet{lindsay_rival_2011} and \citet{lindsay_natural_2013}. The authors themselves do so by comparing the ratio of the two suffixes for stems that occur with both suffixes. Let us return to their approach later, and start by looking at the overall distribution of stems with \textit{-ic} and \textit{-ical}. 

First, though, let us see what we can find out by looking at the overall distribution of types, using the four-million-word BNC BABY. Once we remove all prefixes and standardize the spelling, there are 846 types for the two suffixes. There is a clear overall preference for \textit{-ic} (659 types) over \textit{-ical} (187 types) (incidentally, there are only 54 stems that occur with both suffixes). For stems with \textit{-(o)log-}, the picture is drastically different: there is an overwhelming preference for \textit{-ical} (55 types) over \textit{-ic} (3 types). We can evaluate this difference statistically, as shown in Table \ref{tab:ologicalchi}.

\begin{table}[!htbp]
\caption{Preference of stems with \textit{-olog-} for \textit{-ic} and \textit{-ical}}
\label{tab:ologicalchi}
\begin{tabular}[t]{llccr}
\lsptoprule
 & & \multicolumn{2}{c}{\textvv{Suffix Variant}} & \\
 & & \textvv{-ic} & \textvv{-ical} & Total \\
\midrule
\textvv{\makecell[lt]{Stem Type}}
	& \textvv{with -olog-} 
		& \makecell[t]{\num{3}\\\small{(\num{45.18})}}
		& \makecell[t]{\num{55}\\\small{(\num{12.82})}}
		& \makecell[t]{\num{58}\\} \\
	& \textvv{without -olog-}
		& \makecell[t]{\num{656}\\\small{(\num{613.82})}}
		& \makecell[t]{\num{132}\\\small{(\num{174.18})}}
		& \makecell[t]{\num{788}\\} \\
\midrule
	& Total
		& \makecell[t]{\num{659}}
		& \makecell[t]{\num{187}}
		& \makecell[t]{\num{846}} \\
\lspbottomrule
\end{tabular}
\end{table}
% chisq.test(matrix(c(3,656,55,132),ncol=2),corr=FALSE)

Unsurprisingly, the difference between stems with and without the affix \textit{-olog-} is highly significant ($\chi^2$ = 191.27, df = 1, p < 0.001) -- stems with \textit{-olog-} clearly favor the variant \textit{-ical-} against the general trend.

As mentioned above, this could be due specifically to the suffix \textit{-olog}-, but it could also be a general preference of derived stems for \textit{-ical}. In order to determine this, we have to look at derived stems with other affixes. There are a number of other affixes that occur frequently enough to make them potentially interesting, such as \textit{-ist-}, as in \textit{statistic(al)} (\textit{-ic}: 74/\textit{-ical}: 2), \textit{-graph-}, as in \textit{geographic(al)} (19/8), or  \textit{-et-}, as in \textit{arithmetic(al)} (32/9). Note, that all of them have more types with \textit{-ic}, which suggests that derived stems in general, possibly due to their length, prefer \textit{-ic} and that \textit{-olog-} really is an exception.

But there is a methodological issue that we have to address before we can really conclude this. Note that we have been talking of a ``preference'' of particular stems for one or the other suffix, but this is somewhat imprecise: we looked at the total number of stem types with \textit{-ic} and \textit{-ical} with and without additional suffixes. While differences in number are plausibly attributed to ``preferences'', they may also be purely historical leftovers due to the specific history of the two suffixes (which is rather complex, involving borrowing from Latin, Greek and, in the case of \textit{-ical}, French). More convincing evidence for a productive difference in preferences would come from stems that take both \textit{-ic} and \textit{-ical} (such as \textit{electric/al}, \textit{symmetric/al} or \textit{numeric/al}, to take three examples that display a relatively even distribution between the two): for these stems, there is obviously a choice, and we can investigate the influence of additional affixes on that choice.

\citet{lindsay_rival_2011} and \citet{lindsay_natural_2013} focus on precisely these stems, check for each one whether it occurs more frequently with \textit{-ic} or with \textit{-ical} and calculating the preference ratio mentioned above. They then compare the ratio of all stems to that of stems with \textit{-olog-} (see Table \ref{tab:icicallindsay}).

\begin{table}[!htbp]
\caption{Stems favoring -ic or -ical in the COCA \citep[194]{lindsay_rival_2011}}
\label{tab:icicallindsay}
\begin{tabular}[t]{lrr rr}
\lsptoprule
 & \multicolumn{1}{l}{Total Stems} & \multicolumn{1}{l}{Ratio} & \multicolumn{1}{l}{\textit{-olog-} stems} & \multicolumn{1}{l}{Ratio} \\
\midrule
favoring \textit{-ic}	& 1197	& \multirow{2}{*}{{$\nicefrac{4.5}{1}$}}	& 13	& \multirow{2}{*}{{$\nicefrac{1}{5.6}$}} \\
favoring \textit{-ical}	& 268	&                                	& 73	&  \\
\midrule
Total	               & 1465	&	                                 & 86	& \\
\lspbottomrule
\end{tabular}
\end{table}

The ratios themselves are difficult to compare statistically, but they are clearly the right way of measuring the preference of stems for a particular suffix. So let us take Lindsay and Aronoff's approach one step further: Instead of calculating the overall preference of a particular type of stem and comparing it to the overall preference of all stems, let us calculate the preference for each stem individually. This will give us a preference measure for each stem that is, at the very least, ordinal.\footnote{In fact, measures derived in this way are cardinal data, as the value can range from 0 to 1 with every possible value in between; it is safer to treat them as ordinal data, however, because we don't know whether such preference values are normally distributed. In fact, since they are based on word frequency data, which we know \textit{not} to be normally distributed, it is a fair guess that the preference data are not normally distributed.} We can then rank stems containing a particular affix and stems not containing that affix (or containing a specific different affix) by their preference for one or the other of the suffix variants and use the Mann-Whitney U-test to determine whether the stems with \textit{-olog-} tend to occur towards the \textit{-ical} end of the ranking. That way, we can treat preference as the matter of degree that it actually is, rather than as an absolute property of stems.

The BNC BABY does not contain enough derived stems that occur with both suffix variants, so let us focus on two specific suffixes and extract the relevant data from the full BNC. Since \textit{-ist} is roughly equal to \textit{-olog-} in terms of type frequency, let us choose this suffix for comparison. Table \ref{tab:istological} shows the 34 stems containing either of these suffixes, their frequency of occurrence with the variants \textit{-ic} and \textit{-ical}, the preference ratio for \textit{-ic}, and the rank.

\begin{table}[!htbp]
\caption{Preference of stems containing \textit{-ist} and \textit{-olog-} for the suffix variants \textit{-ic} and \textit{-ical} (BNC)}
\label{tab:istological}
%\resizebox*{!}{\textheight}{%
\begin{tabular}[t]{lc *{2}S S[round-mode=places,round-precision=4] S}
\lsptoprule
\multicolumn{1}{l}{Stem} & \multicolumn{1}{c}{Suffix} & \multicolumn{1}{c}{n(-ic)} & \multicolumn{1}{c}{n(-ical)} & \multicolumn{1}{c}{p(-ic)} & \multicolumn{1}{c}{Rank} \\
\midrule
\textit{realist-} & \textit{-ist} & 2435 & 1 & 0.9996 & 1 \\
\textit{atomist-} & \textit{-ist} & 54 & 1 & 0.9818 & 2 \\
\textit{egoist-} & \textit{-ist} & 50 & 1 & 0.9804 & 3 \\
\textit{syllogist-} & \textit{-ist} & 8 & 1 & 0.8889 & 4 \\
\textit{atheist-} & \textit{-ist} & 25 & 9 & 0.7353 & 5 \\
\textit{casuist-} & \textit{-ist} & 2 & 1 & 0.6667 & 6 \\
\textit{logist-} & \textit{-ist} & 148 & 94 & 0.6116 & 7 \\
\textit{pedolog-} & \textit{-olog-} & 2 & 8 & 0.2000 & 8 \\
\textit{hydrolog-} & \textit{-olog-} & 7 & 56 & 0.1111 & 9 \\
\textit{egotist-} & \textit{-ist} & 4 & 54 & 0.0690 & 10 \\
\textit{pharmacolog-} & \textit{-olog-} & 5 & 72 & 0.0649 & 11 \\
\textit{serolog-} & \textit{-olog-} & 3 & 54 & 0.0526 & 12 \\
\textit{haematolog-} & \textit{-olog-} & 2 & 40 & 0.0476 & 13 \\
\textit{litholog-} & \textit{-olog-} & 1 & 24 & 0.0400 & 14 \\
\textit{petrolog-} & \textit{-olog-} & 1 & 28 & 0.0345 & 15 \\
\textit{etymolog-} & \textit{-olog-} & 1 & 29 & 0.0333 & 16 \\
\textit{morpholog-} & \textit{-olog-} & 11 & 347 & 0.0307 & 17 \\
\textit{immunolog-} & \textit{-olog-} & 3 & 107 & 0.0273 & 18 \\
\textit{philolog-} & \textit{-olog-} & 1 & 39 & 0.0250 & 19 \\
\textit{aetiolog-} & \textit{-olog-} & 1 & 40 & 0.0244 & 20 \\
\textit{mineralogic} & \textit{-olog-} & 1 & 41 & 0.0238 & 21 \\
\textit{patholog-} & \textit{-olog-} & 4 & 319 & 0.0124 & 22 \\
\textit{histolog-} & \textit{-olog-} & 5 & 404 & 0.0122 & 23 \\
\textit{radiolog-} & \textit{-olog-} & 2 & 171 & 0.0116 & 24 \\
\textit{epidemiolog-} & \textit{-olog-} & 2 & 186 & 0.0106 & 25 \\
\textit{epistemolog-} & \textit{-olog-} & 2 & 206 & 0.0096 & 26 \\
\textit{physiolog-} & \textit{-olog-} & 7 & 778 & 0.0089 & 27 \\
\textit{ontolog-} & \textit{-olog-} & 2 & 279 & 0.0071 & 28 \\
\textit{geolog-} & \textit{-olog-} & 7 & 999 & 0.0070 & 29 \\
\textit{biolog-} & \textit{-olog-} & 6 & 2093 & 0.0029 & 30 \\
\textit{ecolog-} & \textit{-olog-} & 2 & 746 & 0.0027 & 31 \\
\textit{sociolog-} & \textit{-olog-} & 1 & 808 & 0.0012 & 32 \\
\textit{technolog-} & \textit{-olog-} & 2 & 1666 & 0.0012 & 33 \\
\textit{ideolog-} & \textit{-olog-} & 1 & 1696 & 0.0006 & 34 \\
\lspbottomrule
\end{tabular}%}
\end{table}
% BNC; [word=".*(log|ist)ic(al)?" & pos=".*AJ.*"]; count Last by hw; manually lemmatized

The different preference of stems with \textit{-ist} and \textit{-olog-} are very obvious even from a purely visual inspection of the table: stems with the former occur at the top of the ranking, stems with the latter occur at the bottom and there is almost no overlap. This is reflected clearly in the median ranks of the two stem types: the median for \textit{-ist} is 4.5 (N = 8, rank sum = 38), the median for \textit{-olog-} is 21.5 (N = 26, rank sum = 557). A Mann-Whitney U-test shows that this difference is highly significant (U = 2, N1 = 8, N2 = 26, p < 0.001).

Now that we have established that different suffixes may, indeed, display different preferences for other suffixes (or suffix variants), we could begin to answer the question why this might be the case. In this instance, the explanation is likely found in the complicated history of borrowings containing the suffixes in question. The point of this case study was not to provide such an explanation but to show how an empirical basis can be provided using token frequencies derived from linguistic corpora.

\subsection{Morphemes and demographic variables}
\label{sec:morphemesanddemographicvariables}

There are a few studies investigating the productivity of derivational morphemes across text types (comparing, e.g., written and spoken language or genres), across groups defined by sex, education and/or class, or across varieties. This is an extremely interesting area of research that may offer valuable insights into the very nature of morphological richness and productivity, allowing us, for example, to study potential differences between regular, presumably subconscious applications of derivational rules and the deliberate coining of words. Despite this, it is an area that has not been studied too intensively, so there is much that remains to be discovered.

\subsubsection{Case study: Productivity and genre}
\label{sec:productivityandgenre}

\citet{guz_english_2009} studies the prevalence of different kinds of nominalization across genres. The question whether the productivity of derivational morphemes differs across genres is a very interesting one, and Guz presents a potentially more detailed analysis than previous studies in that he looks at the prevalence of different stem types for each affix, so that qualitative as well as quantitative differences in productivity could, in theory, be studied. In practice, unfortunately, the study offers preliminary insights at best, as it is based entirely on token frequencies, which, as discussed in Section \ref{sec:quantifyingmorphologicalphenomena} above, do not tell us anything at all about productivity.

We will therefore look at a question inspired by Guz's study, and use the TTR and the HTR to study the relative importance and productivity of the nominalizing suffix \textit{-ship} (as in \textit{friendship}, \textit{lordship}, etc.) in newspaper language and in prose fiction. The suffix \textit{-ship} is known to have a very limited productivity, and our hypothesis (for the sake of the argument) will be that it is more productive in prose fiction, since authors of fiction are under pressure to use language creatively (this is not Guz's hypothesis; his study is entirely explorative).

The suffix has a relatively high token frequency: there are 2862 tokens in the fiction section of the BNC, and 7189 tokens in the newspaper section (including all sub-genres of newspaper language, such as reportage, editorial, etc.). This difference is not due to the respective sample sizes: the fiction section in the BNC is much larger than the newspaper section; thus, the difference token frequency would suggest that the suffix is more important in newspaper language than in fiction. However, as extensively discussed in Section \ref{sec:countingmorphemes}, token frequency cannot be used to base such statements on. Instead, we need to look at the Type-Token-Ratio and the Hapax-Token-Ratio.

To get a first impression, consider Figure \ref{fig:genreshipfullttrhtr}, which shows the growth of the TTR (left) and HTR (right) in the full \textvv{fiction} and \textvv{newspaper} sections of the BNC. 

\begin{figure}
\caption{Nouns with the suffix \textit{-ship} in Fiction and Newspapers (BNC)}
\label{fig:genreshipfullttrhtr}
\begin{minipage}{.5\textwidth}
  \centering
  \includegraphics[width=\textwidth]{figures/genreshipfullttr}
\end{minipage}
%
\begin{minipage}{.5\textwidth}
  \centering
  \includegraphics[width=\textwidth]{figures/genreshipfullhtr}
\end{minipage}
\end{figure}

Both the TTR and the HTR suggest that the suffix is more productive in fiction: the ratios rise faster in \textvv{fiction} than in \textvv{newspapers} and remain consistently higher as we go through the two sub-corpora. It is only when the tokens have been exhausted in the fiction subcorpus but not in the newspaper subcorpus, that the ratios in the latter slowly catch up. This broadly supports our hypothesis, but let us look at the genre differences more closely both qualitatively and quantitatively.

In order to compare the two genres in terms of the type-token and hapax-token ratios, they need to have the same size. The following discussion is based on the full data from the fiction subcorpus and a subsample of the newspaper corpus that was arrived at by deleting every second, then every third and finally every 192nd example, ensuring that the hits in the sample are spread through the entire newspaper subcorpus.

Let us begin by looking at the types. Overall, there are 96 different types, 48 of which occur in both samples (some examples of types that frequent in both samples are \textit{relationship} (the most frequent word in the fiction sample), \textit{championship} (the most frequent word in the news sample), \textit{friendship}, \textit{partnership}, \textit{lordship}, \textit{ownership} and \textit{membership}. In addition, there are 36 types that occur only in the prose sample (for example, \textit{churchmanship}, \textit{dreamership}, \textit{librarianship} and \textit{swordsmanship}) and 12 that occur only in the newspaper sample (for example, \textit{associateship}, \textit{draughtsmanship}, \textit{trusteeship} and \textit{sportsmanship}). The number of types exclusive to each genre suggests that the suffix is more important in \textvv{fiction} than in \textvv{newspapers}.

The TTR of the suffix in newspaper language is $\nicefrac{60}{2862} = 0.021$, and the HTR is $\nicefrac{20}{2862} = 0.007$. In contrast, the TTR in fiction is $\nicefrac{84}{2862} = 0.0294$, and the HTR is $\nicefrac{29}{2862} = 0.0101$. Although the suffix, as expected, is generally not very productive, it is more productive in fiction than in newspapers. As Table \ref{tab:shipwords} shows, this difference is statistically significant in the sample ($\chi^2$ = 4.1, df = 1, p < 0.005). This corroborates our hypothesis, but note that it does not tell us whether the higher productivity of \textit{-ship} is something unique about this particular morpheme, or whether fiction generally has more derived words due to a higher overall lexical richness. To determine this, we would have to look at more than one affix.

\begin{table}[!htbp]
\caption{Types with \textit{-ship} in prose fiction and newspapers}
\label{tab:shipwords}
\begin{tabular}[t]{llccr}
\lsptoprule
 & & \multicolumn{2}{c}{\textvv{Type}} & \\
 & & \textvv{new} & \textvv{$\neg$new} & Total \\
\midrule
\textvv{\makecell[lt]{Genre}}
	& \textvv{fiction} 
		& \makecell[t]{\num{84}\\\small{(\num{72.00})}}
		& \makecell[t]{\num{2778}\\\small{(\num{2790.00})}}
		& \makecell[t]{\num{2862}\\} \\
	& \textvv{newspaper}
		& \makecell[t]{\num{60}\\\small{(\num{72.00})}}
		& \makecell[t]{\num{2802}\\\small{(\num{2790.00})}}
		& \makecell[t]{\num{2862}\\} \\
\midrule
	& Total
		& \makecell[t]{\num{144}}
		& \makecell[t]{\num{5580}}
		& \makecell[t]{\num{5724}} \\
\lspbottomrule
\end{tabular}
\end{table}
% chisq.test(matrix(c(84,60,2778,2802),ncol=2),corr=FALSE)

Let us now turn to the hapax legomena. These are so rare in both genres, that the difference in TTR is not statistically significant, as Table \ref{tab:shiphapaxes} ($\chi^2$ = 1.67, df = 1, p = 0.1966). We would need a larger corpus to see whether the difference would at some point become significant.

\begin{table}[!htbp]
\caption{Hapaxes with \textit{-ship} in prose fiction and newspapers}
\label{tab:shiphapaxes}
\begin{tabular}[t]{llccr}
\lsptoprule
 & & \multicolumn{2}{c}{\textvv{Type}} & \\
 & & \textvv{hapax} & \textvv{$\neg$hapax} & Total \\
\midrule
\textvv{\makecell[lt]{Genre}}
	& \textvv{fiction} 
		& \makecell[t]{\num{29}\\\small{(\num{24.50})}}
		& \makecell[t]{\num{2833}\\\small{(\num{2837.50})}}
		& \makecell[t]{\num{2862}\\} \\
	& \textvv{newspaper}
		& \makecell[t]{\num{20}\\\small{(\num{24.50})}}
		& \makecell[t]{\num{2842}\\\small{(\num{2837.50})}}
		& \makecell[t]{\num{2862}\\} \\
\midrule
	& Total
		& \makecell[t]{\num{49}}
		& \makecell[t]{\num{5675}}
		& \makecell[t]{\num{5724}} \\
\lspbottomrule
\end{tabular}
\end{table}
% chisq.test(matrix(c(29,20,2833,2842),ncol=2),corr=FALSE)

To conclude this case study, let us look at a particular problem posed by the comparison of the same suffix in two genres with respect to the HTR. At first glance -- and this is what is shown in Table \ref{tab:shiphapaxes} -- there seem to be 29 hapaxes in fiction and 20 in prose. However, there is some overlap: the words \textit{generalship}, \textit{headship}, \textit{managership}, \textit{ministership} and \textit{professorship} occur as hapax legomena in both samples; other words that are hapaxes in one subsample occur several times in the other, such as \textit{brinkmanship}, which is a hapax in fiction but occurs twice in the newspaper sample, or \textit{acquaintanceship}, which is a hapax in the newspaper sample but occurs 15 times in fiction.

It is not straightforwardly clear whether such cases should be treated as hapaxes. If we think of the two samples as subsamples of the same corpus, it is very counterintuitive to do so. It might be more reasonable to count only those words as hapaxes whose frequency in the combined subsamples is still one. However, the notion ``hapax'' is only an operational definition for neologisms, based on the hope that the number of hapaxes in a corpus (or sub-corpus) is somehow indicative of the number of productive coinages. We saw in Case Study \ref{sec:phonologicalconstraintsofify} that this is a somewhat vain hope, as the correlation between neologisms and hapaxes is not very impressive.

Still, if we want to use this operational definition, we have to stick with it and define hapaxes strictly relative to whatever (sub-)corpus we are dealing with. If we extend the criterion for hapax-ship beyond one subsample to the other, why stop there? We might be even stricter and count only those words as hapaxes that are still hapaxes when we take the entire BNC into account. And if we take the entire BNC into account, we might as well count as hapaxes only those words that occur only once in all accessible archives of the language under investigation. This would mean that the hapaxes in any sample would overwhelmingly cease to be hapaxes -- the larger our corpus, the fewer hapaxes there will be. To illustrate this: just two words from the fiction sample retains their status as a hapax legomenon if we search the Google Books collection: \textit{impress-ship}, which does not occur at all (if we discount linguistic accounts which mention it, such as \citet{trips_lexical_2009}, or this book, once it becomes part of the Google Books archive), and \textit{cloudship}, which does occur, but only referring to water- or airborne vehicles. At the same time, the Google Books archive contains hundreds (if not thousands) of hapax legomena that we never even notice (such as \textit{Johnship} ``the state of being the individual referred to as \textit{John}''). The idea of using hapax legomena is, essentially, that a word like \textit{mageship}, which is a hapax in the fiction sample, but not in the Google Books archive, somehow stands for a word like \textit{Johnship}, which is a true hapax in the English language.

This case study has demonstrated the potential of using the TTR and the HTR not as a means of assessing morphological richness and productivity as such, but as a means of assessing genres with respect to their richness and productivity. It has also demonstrated some of the problems of identifying hapax legomena in the context of such cross-text-type comparisons. As mentioned initially, there are not too many studies of this type, but the study by \citet{plag_morphological_1999} study of productivity across written and spoken language is a good starting point for anyone wanting to fill this gap.

\subsubsection{Case study: Productivity and speaker sex}
\label{sec:productivityandspeakersex}

Morphological productivity has not traditionally been investigated from a sociolinguistic perspective, but a study by \citet{saily_variation_2011} suggests that this may be a promising field of research. S\"{a}ily investigates differences in the productivity of the suffixes \textit{-ness} and \textit{-ity} in the language produced by men and women in the BNC. She finds no difference in productivity for \textit{-ness}, but a higher productivity of \textit{-ity} in the language produced by men (cf. also \citet{saily_comparing_2009} for a diachronic study with very similar results). She uses a sophisticated method involving the comparison of the suffixes' type and hapax growth rates, but let us replicate her study using the simple method used in the preceding case study, beginning with a comparison of type-token ratios.

The BNC contains substantially more speech and writing by male speakers than by female speakers, which is reflected in differences in the number of affix tokens produced by men and women: for \textit{-ity}, there are \num{2562} tokens produced by women and \num{8916} tokens produced by men; for \textit{-ness}, there are 616 tokens produced by women and \num{1154} tokens produced by men (note that unlike S\"{a}ily, I excluded the words \textit{business} and \textit{witness}, since they did not seem to me to be synchronically transparent instances of the affix). To get samples of equal size for each affix, random sub-samples were drawn from the tokens produced by men.

Based on these subsamples, the type-token ratios for \textit{-ity} are 0.0652 for men and 0.0777 for women; as Table \ref{tab:itysex} shows, this difference is not statistically significant ($\chi^2$ = 3.01, df = 1, p < 0.05, $\phi$ = 0.0242).

\begin{table}[!htbp]
\caption{Types with \textit{-ity} in male and female speech (BNC)}
\label{tab:itysex}
\begin{tabular}[t]{llccr}
\lsptoprule
 & & \multicolumn{2}{c}{\textvv{Type}} & \\
 & & \textvv{new} & \textvv{seen before} & Total \\
\midrule
\textvv{\makecell[lt]{Speaker Sex}}
	& \textvv{female} 
		& \makecell[t]{\num{167}\\\small{(\num{183.00})}}
		& \makecell[t]{\num{2395}\\\small{(\num{2379.00})}}
		& \makecell[t]{\num{2562}\\} \\
	& \textvv{male}
		& \makecell[t]{\num{199}\\\small{(\num{183.00})}}
		& \makecell[t]{\num{2363}\\\small{(\num{2379.00})}}
		& \makecell[t]{\num{2562}\\} \\
\midrule
	& Total
		& \makecell[t]{\num{366}}
		& \makecell[t]{\num{4758}}
		& \makecell[t]{\num{5124}} \\
\lspbottomrule
\end{tabular}
\end{table}

The type-token ratios for \textit{-ness} are much higher, namely 0.1981 for women and 0.2597 for men. As Table \ref{tab:nesssex} shows, the difference is statistically significant, although the effect size is weak ($\chi^2$ = 5.37, df = 1, p < 0.05, $\phi$ = 0.066).

\begin{table}[!htbp]
\caption{Types with \textit{-ness} in male and female speech (BNC)}
\label{tab:nesssex}
\begin{tabular}[t]{llccr}
\lsptoprule
 & & \multicolumn{2}{c}{\textvv{Type}} & \\
 & & \textvv{new} & \textvv{seen before} & Total \\
\midrule
\textvv{\makecell[lt]{Speaker Sex}}
	& \textvv{female} 
		& \makecell[t]{\num{122}\\\small{(\num{139.00})}}
		& \makecell[t]{\num{494}\\\small{(\num{477.00})}}
		& \makecell[t]{\num{616}\\} \\
	& \textvv{male}
		& \makecell[t]{\num{156}\\\small{(\num{139.00})}}
		& \makecell[t]{\num{460}\\\small{(\num{477.00})}}
		& \makecell[t]{\num{616}\\} \\
\midrule
	& Total
		& \makecell[t]{\num{278}}
		& \makecell[t]{\num{954}}
		& \makecell[t]{\num{1232}} \\
\lspbottomrule
\end{tabular}
\end{table}
% chisq.test(matrix(c(122,156,494,460),ncol=2),corr=FALSE)

Note that S\"{a}ily investigates spoken and written language separately and she also includes social class in her analysis, so her results differ from the ones presented here; she finds a significantly lower HTR for \textit{-ness} in lower-class women's speech in the spoken subcorpus, but not in the written one, and a significantly lower HTR for \textit{-ity} in both subcorpora. This might be due to the different methods used, or to the fact that I excluded \textit{business}, which is disproportionally frequent in male speech and writing in the BNC and would thus reduce the diversity in the male sample substantially. However, the type-based differences do not have a very impressive effect size in our design and they are unstable across conditions in S\"{a}ily's, so perhaps they are simply not very substantial.

Let us turn to the HTR next. As before, we are defining what counts as a hapax legomenon not with reference to the individual subsamples of male and female speech, but with respect to the combined sample. Table \ref{tab:ityhapaxsexlist} shows the Hapaxes for \textit{-ity} in the male and female samples. The HTRs are very low, suggesting that \textit{-ity} is not a very productive suffix: 0.0099 in female speech and 0.016 in male speech.

\begin{table}[!htbp]
\caption{Hapaxes with \textit{-ity} in sampes of male and female speech (BNC)}
\label{tab:ityhapaxsexlist}
\resizebox{0.9\textwidth}{!}{%
\begin{tabular}[t]{l}
\lsptoprule
\textvv{male speech} \\
\midrule
\makecell[tl]{
\begin{minipage}[t]{\textwidth} \raggedright
\textit{abnormality}, \textit{antiquity}, \textit{applicability}, \textit{brutality}, \textit{civility}, \textit{criminality}, \textit{deliverability}, \textit{divinity}, \textit{duplicity}, \textit{eccentricity}, \textit{eventuality}, \textit{falsity}, \textit{femininity}, \textit{fixity}, \textit{frivolity}, \textit{illegality}, \textit{impurity}, \textit{inexorability}, \textit{infallibility}, \textit{infirmity}, \textit{levity}, \textit{longevity}, \textit{mediocrity}, \textit{obesity}, \textit{perversity}, \textit{predictability}, \textit{rationality}, \textit{regularity}, \textit{reliability}, \textit{scarcity}, \textit{seniority}, \textit{serendipity}, \textit{solidity}, \textit{subsidiarity}, \textit{susceptibility}, \textit{tangibility}, \textit{verity}, \textit{versatility}, \textit{virtuality}, \textit{vitality}, \textit{voracity}
\end{minipage}} \\
\midrule
\textvv{female speech} \\
\midrule
\makecell[tl]{
\begin{minipage}[t]{\textwidth} \raggedright
\textit{absurdity}, \textit{adjustability}, \textit{admissibility}, \textit{centrality}, \textit{complicity}, \textit{effemininity}, \textit{enormity}, \textit{exclusivity}, \textit{gratuity}, \textit{hilarity}, \textit{humility}, \textit{impunity}, \textit{inquisity}, \textit{morbidity}, \textit{municipality}, \textit{originality}, \textit{progility}, \textit{respectability}, \textit{sanity}, \textit{scaleability}, \textit{sincerity}, \textit{spontaneity}, \textit{sterility}, \textit{totality}, \textit{virginity}
\end{minipage}} \\
\lspbottomrule
\end{tabular}}
\end{table}

Although the difference in HTR is relatively small, Table \ref{tab:ityhapaxfrequencies} shows that it is statistically significant, albeit again with a very weak effect size ($\chi^2$ = 3.93, df = 1, p < 0.05, $\phi$ = 0.0277).

\begin{table}[!htbp]
\caption{Hapax legomena with \textit{-ity} in male and female speech (BNC)}
\label{tab:ityhapaxfrequencies}
\begin{tabular}[t]{llccr}
\lsptoprule
 & & \multicolumn{2}{c}{\textvv{Type}} & \\
 & & \textvv{hapax} & \textvv{$\neg$hapax} & Total \\
\midrule
\textvv{\makecell[lt]{Speaker Sex}}
	& \textvv{female} 
		& \makecell[t]{\num{25}\\\small{(\num{33.00})}}
		& \makecell[t]{\num{2537}\\\small{(\num{2529.00})}}
		& \makecell[t]{\num{2562}\\} \\
	& \textvv{male}
		& \makecell[t]{\num{41}\\\small{(\num{33.00})}}
		& \makecell[t]{\num{2521}\\\small{(\num{2529.00})}}
		& \makecell[t]{\num{2562}\\} \\
\midrule
	& Total
		& \makecell[t]{\num{66}}
		& \makecell[t]{\num{5058}}
		& \makecell[t]{\num{5124}} \\
\lspbottomrule
\end{tabular}
\end{table}
% chisq.test(matrix(c(25,41,2537,2521),ncol=2),corr=FALSE)

Table \ref{tab:nesssexlist} shows the hapaxes for \textit{-ness} in the male and female samples. The HTRs are low, but much higher than for \textit{-ity}, 0.0795 for women and 0.1023 for men.

\begin{table}[!htbp]
\caption{Hapaxes with \textit{-ness} in samples of male and female speech (BNC)}
\label{tab:nesssexlist}
\resizebox*{0.9\textwidth}{!}{%
\begin{tabular}[t]{l}
\lsptoprule
\textvv{male speech} \\
\midrule
\makecell[tl]{
\begin{minipage}[t]{\textwidth} \raggedright
\textit{abjectness}, \textit{adroitness}, \textit{aloneness}, \textit{anxiousness}, \textit{awfulness}, \textit{barrenness}, \textit{blackness}, \textit{blandness}, \textit{bluntness}, \textit{carefulness}, \textit{centredness}, \textit{cleansiness}, \textit{clearness}, \textit{cowardliness}, \textit{crispness}, \textit{delightfulness}, \textit{differentness}, \textit{dizziness}, \textit{drowsiness}, \textit{dullness}, \textit{eyewitnesses}, \textit{fondness}, \textit{fulfilness}, \textit{genuineness}, \textit{godliness}, \textit{graciousness}, \textit{headedness}, \textit{heartlessness}, \textit{heinousness}, \textit{keenness}, \textit{lateness}, \textit{likeliness}, \textit{limitedness}, \textit{loudness}, \textit{mentalness}, \textit{messiness}, \textit{narrowness}, \textit{nearness}, \textit{neighbourliness}, \textit{niceness}, \textit{numbness}, \textit{pettiness}, \textit{pleasantness}, \textit{plumpness}, \textit{positiveness}, \textit{quickness}, \textit{reasonableness}, \textit{rightness}, \textit{riseness}, \textit{rudeness}, \textit{sameness}, \textit{sameyness}, \textit{separateness}, \textit{shortness}, \textit{smugness}, \textit{softness}, \textit{soreness}, \textit{springiness}, \textit{steadiness}, \textit{stubbornness}, \textit{timorousness}, \textit{toughness}, \textit{uxoriousness}
\end{minipage}} \\
\midrule
\textvv{female speech} \\
\midrule
\makecell[tl]{
\begin{minipage}[t]{\textwidth} \raggedright
\textit{ancientness}, \textit{appropriateness}, \textit{badness}, \textit{bolshiness}, \textit{chasifness}, \textit{childishness}, \textit{chubbiness}, \textit{clumsiness}, \textit{conciseness}, \textit{eagerness}, \textit{easiness}, \textit{faithfulness}, \textit{falseness}, \textit{feverishness}, \textit{fizziness}, \textit{freshness}, \textit{ghostliness}, \textit{greyness}, \textit{grossness}, \textit{grotesqueness}, \textit{heaviness}, \textit{laziness}, \textit{likeness}, \textit{mysteriousness}, \textit{nastiness}, \textit{outspokenness}, \textit{pinkness}, \textit{plainness}, \textit{politeness}, \textit{prettiness}, \textit{priggishness}, \textit{primness}, \textit{randomness}, \textit{responsiveness}, \textit{scratchiness}, \textit{sloppiness}, \textit{smoothness}, \textit{stiffness}, \textit{stretchiness}, \textit{tenderness}, \textit{tightness}, \textit{timelessness}, \textit{timidness}, \textit{ugliness}, \textit{uncomfortableness}, \textit{unpredictableness}, \textit{untidiness}, \textit{wetness}, \textit{zombieness}
\end{minipage}} \\
\lspbottomrule
\end{tabular}}
\end{table}

As Table \ref{tab:nesssexfrequencies} shows, the difference in HTRs is not statistically significant, and the effect size would be very weak anyway ($\chi^2$ = 1.93, df = 1, p > 0.05, $\phi$ = 0.0395).

\begin{table}[!htbp]
\caption{Hapax legomena with \textit{-ness} in male and female speech (BNC)}
\label{tab:nesssexfrequencies}
\begin{tabular}[t]{llccr}
\lsptoprule
 & & \multicolumn{2}{c}{\textvv{Type}} & \\
 & & \textvv{hapax} & \textvv{$\neg$hapax} & Total \\
\midrule
\textvv{\makecell[lt]{Speaker Sex}}
	& \textvv{female} 
		& \makecell[t]{\num{49}\\\small{(\num{56.00})}}
		& \makecell[t]{\num{567}\\\small{(\num{560.00})}}
		& \makecell[t]{\num{616}\\} \\
	& \textvv{male}
		& \makecell[t]{\num{63}\\\small{(\num{56.00})}}
		& \makecell[t]{\num{553}\\\small{(\num{560.00})}}
		& \makecell[t]{\num{616}\\} \\
\midrule
	& Total
		& \makecell[t]{\num{112}}
		& \makecell[t]{\num{1120}}
		& \makecell[t]{\num{1232}} \\
\lspbottomrule
\end{tabular}
\end{table}
% chisq.test(matrix(c(49,63,567,553),ncol=2),corr=FALSE)

In this case, the results correspond to S\"{a}ily's, who also finds a significant difference in productivity for \textit{-ity}, but not for \textit{-ness}.

This case study was meant to demonstrate, once again, the method of comparing TTRs and HTRs based on samples of equal size. It was also meant to draw attention to the fact that morphological productivity may be an interesting area of research for variationist sociolinguistics; however, it must be pointed out that it would be premature to conclude that men and women differ in their productive use of particular affixes; as S\"{a}ily herself points out, men and women are not only represented unevenly in quantitative terms (with a much larger proportion of male language included in the BNC), but also in qualitative terms (the text types with which they are represented differ quite strikingly). Thus, this may actually be another case of different degrees of productivity in different text types (which we investigated in the preceding case study).

