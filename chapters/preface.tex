\addchap{\lsPrefaceTitle}
 
This book has been a long time in the making (the first version of the first chapter has the date stamp 2005-08-04, 11:30) and as the field of corpus linguistics and my own perspective on this field developed over this time span, many thoughts accumulated, that I intended to put into the preface when the time would come to publish. Now that this time is finally there, I feel that there is not much left to say that I have not already said in the book itself.

However, given that there is, by now, a large number of corpus\hyp{}linguistic textbooks available, ranging from the very decent to the excellent, a few words seem in order to explain why I feel that it makes sense to publish another one. The main reason is that I have found, in my many years of teaching corpus linguistics, that most available textbooks are either too general or too specific. On the one hand, there are textbooks that provide excellent discussions of the history of corpus linguistics or the history of corpus design, or that discuss the epistemological status of corpus data in a field that has been dominated far too long by generative linguistic ideas about what does and does not constitute linguistic evidence. On the other hand, there are textbooks that focus one or more specific corpus\hyp{}based techniques, discussing very specific phenomena (often the research interests of the textbook authors themselves) using a narrow range of techniques (often involving specific software solutions).

What I would have wanted and needed when I took my first steps into corpus linguistic research as a student is an introductory textbook that focuses on methodological issues -- on how to approach the study of language based on usage data and what problems to expect and circumvent. A book that discusses the history and epistemology of corpus linguistics only to the extent necessary to grasp these methodological issues and that presents case studies of a broad range of linguistic phenomena from a coherent methodological perspective. This book is my attempt to write such a textbook.

The first part of the book begins with an almost obligatory chapter on the need for corpus data (a left\hyp{}over from a time when corpus linguistics was still somewhat of a fringe discipline). I then present what I take to be the methodological foundations that distinguish corpus linguistics from other, superficially similar methodological frameworks, and discuss the steps necessary to build concrete research projects on these foundations – formulating the research question, operationalizing the relevant constructs and deriving quantitative predictions, extracting and annotating data, evaluating the results statistically and drawing conclusions. The second part of the book presents a range of case studies from the domains of lexicology, grammar, text linguistics and metaphor, including variationist and diachronic perspectives. These case studies are drawn from the vast body of corpus linguistic research literature published over the last thirty years, but they are all methodologically deconstructed and explicitly reconstructed in terms of the methodological framework developed in the first part of the book.

While I refrain from introducing specific research tools (e.g. in the form of specific concordancing or statistics software), I have tried to base these case studies on publicly available corpora to allow readers to replicate them using whatever tools they have at their disposal. I also provide supplementary online material, including information about the corpora and corpus queries used as well as, in many cases, the full data sets on which the case studies are based. At the time of publication, the supplementary online material is available as a zip file via \url{http://stefanowitsch.net/clm/clm_v01.zip} and \url{https://osf.io/89mgv} and as a repository on GitHub via \url{https://github.com/astefanowitsch/clm_v01}. I hope that it will remain available at least at one of these locations for the foreseeable future. Feel free to host the material in additional locations.

I hope that the specific perspective taken in this book, along with the case studies and the possibility to study the full data sets, will help both beginning and seasoned researchers gain an understanding of the underlying logic of corpus linguistic research. If not -- the book is free as in beer, so at least you will not have wasted any money on it. It is also free as in speech -- the Creative Commons license under which it is published allows you to modify and build on the content, remixing it into the textbook \emph{you} would have wanted and needed.

\begin{flushright}
Berlin, 10th December 2019
\end{flushright}
